\documentclass[xcolor=table,dvipsnames,svgnames,aspectratio=169,fontset=fandol]{ctexbeamer}
% 可以通过 fontset=macnew / fontset=ubuntu / fontset=windows 选项切换字体集
% 如果ubuntu字体集无法使用,可以尝试fandol
\usepackage{tikz}
\usepackage[normalem]{ulem}
\usetikzlibrary{arrows}
\usepackage{amsmath}
\usepackage{mflogo}
\usepackage{graphicx}
\usepackage{ccicons}
\usepackage{hologo}
\usepackage{colortbl}
\usepackage{shapepar}
\usepackage{hyperxmp}
\usepackage{booktabs}
\usepackage{qrcode}
\usepackage{listings}
\usepackage{tipa}
\usepackage{multicol}
\usepackage{datetime2}
\usepackage{fontawesome5}
\usepackage{hyperref}
\usepackage{enumitem}
\usepackage{bm}
\usepackage[backend=biber,style=gb7714-2015]{biblatex}
\addbibresource{thesis.bib}
\graphicspath{{figures/}}
\hypersetup{
  pdfsubject = {上海交通大学图书馆专题培训讲座},
  pdfauthor = {Alexara Wu},
  pdfcopyright = {Licensed under CC-BY-SA 4.0. Some rights reserved.},
  pdflicenseurl = {http://creativecommons.org/licenses/by-sa/4.0/},
  unicode = true,
  psdextra = true,
  pdfdisplaydoctitle = true
}

\DeclareOptionBeamer{en}

\pdfstringdefDisableCommands{
  \let\\\relax
  \let\quad\relax
  \let\hspace\@gobble
}
\renewcommand{\TeX}{\hologo{TeX}}
\renewcommand{\LaTeX}{\hologo{LaTeX}}
\newcommand{\BibTeX}{\hologo{BibTeX}}
\newcommand{\XeTeX}{\hologo{XeTeX}}
\newcommand{\pdfTeX}{\hologo{pdfTeX}}
\newcommand{\LuaTeX}{\hologo{LuaTeX}}
\newcommand{\MiKTeX}{\hologo{MiKTeX}}
\newcommand{\MacTeX}{Mac\hologo{TeX}}
\newcommand{\beamer}{\textsc{beamer}}
\newcommand{\XeLaTeX}{\hologo{Xe}\kern-.13em\LaTeX{}}
\newcommand{\pdfLaTeX}{pdf\LaTeX{}}
\newcommand{\LuaLaTeX}{Lua\LaTeX{}}
\def\TeXLive{\TeX{} Live}
\let\TL=\TeXLive
\newcommand{\SJTUThesis}{\textsc{SJTUThesis}}
\newcommand{\SJTUThesisVersion}{1.1.0}
\newcommand{\SJTUThesisDate}{2023/3/24}
\newcommand{\SJTUBeamer}{\textsc{SJTUBeamer}}
\newcommand{\SJTUBeamerVersion}{3.0.0}
\newcommand{\SJTUBeamerDate}{2022/11/22}
\newcommand\link[1]{\href{#1}{\faLink}}
\newcommand\pkg[1]{\texttt{#1}}
\def\cmd#1{\texttt{\color{structure}\footnotesize $\backslash$#1}}
\def\env#1{\texttt{\color{structure}\footnotesize #1}}
\def\cmdxmp#1#2#3{\small{\texttt{\color{structure}$\backslash$#1}\{#2\}
\hspace{1em}\\ $\Rightarrow$\hspace{1em} {#3}\par\vskip1em}}
\lstset{
  language=[LaTeX]TeX,
  basicstyle=\ttfamily\footnotesize,
  tabsize=2,
  keywordstyle=\bfseries\ttfamily\color{cprimary},
  commentstyle=\sl\ttfamily\color[RGB]{100,100,100},
  stringstyle=\ttfamily\color[RGB]{50,50,50},
  extendedchars=true,
  breaklines=true,
}

\usetheme[maxplus,blue,smoothbars]{sjtubeamer}
% 使用 maxplus/max/min 切换标题页样式
% 使用 red/blue 切换主色调
% 使用 light/dark 切换亮/暗色模式
% 使用外样式关键词以获得不同的边栏样式
%   miniframes infolines  sidebar
%   default    smoothbars split	 
%   shadow     tree       smoothtree
% 使用 topright/bottomright 切换徽标位置
% 使用逗号分隔列表以同时使用多种选项

\author{侯力广  521070910043}
\date{\the\year \,.\the\month \,}
%\subject{LaTeX, 论文排版, SJTUThesis}
\title[流形上的Soblev不等式与嵌入]
{\textbf{General Sobolev Inequalities and \\Embeddings on Riemannian Manifolds}} 

\setbeamercolor{block title alerted}{use=structure,fg=white,bg=structure.fg!88!black}
\setbeamercolor{block body alerted}{parent=normal text,use=block title,bg=block title.bg!10!white}


%=================================================================
%==================================================================

\begin{document}

\maketitle
%-----------------------------------------------------------------------

\begin{frame}{Introduction}
  Our goal in this report is to introduce embeddings of various Sobolev spaces into others. The crucial analytic tools here will be certain so-called "Sobolev inequalities". 

  \vskip 12pt
  Though Riemannian manifolds are natural extensions of Euclidean space, several important questions still puzzle mathematicians today.
\end{frame}

%-----------------------------------------------------------------------

\begin{frame}{Contents}
  \tableofcontents[hideallsubsections]
\end{frame}

%-----------------------------------------------------------------------

\section{Background Materials}

\begin{frame}{Sobolev Spaces on $\mathbf{R^n}$  \footnote{\scalebox{0.8}{Evans, Lawrence C. \textit{Partial differential equations.} Intersxcience Publishers, 1964. }}}
\begin{alertblock}{General Sobolev Inequalities (kp<n)}
  Let $U$ be a bounded open subset of $\mathbb{R}^n$, with a $C^1$ boundary. Assume $u \in W^{k, p}(U)$.
  
  \vskip 5pt
  ~~(i) If $k<\frac{n}{p}, $
  then $u \in L^q(U)$, where
  $$
  \frac{1}{q}=\frac{1}{p}-\frac{k}{n} .
  $$
  ~~We have in addition the estimate
  $$
  \bm
  {||u||_{L^q(U)} \leq C||u||_{W^{k, p}(U)}}
  $$
  ~~the constant $C$ depending only on $k, p, n$ and $U$.
\end{alertblock}
\end{frame}

\begin{frame}{Sobolev Spaces on $\mathbf{R^n}$  \footnote{\scalebox{0.8}{Evans, Lawrence C. \textit{Partial differential equations.} Intersxcience Publishers, 1964. }}}
  \begin{alertblock}{General Sobolev Inequalities (kp>n)}
  ~~(ii) If $k>\frac{n}{p},$ then $u \in C^{k-\left[\frac{n}{p}\right]-1, \gamma}(\bar{U})$, where
  $$
  \gamma=\left\{\begin{array}{l}
  {\left[\frac{n}{p}\right]+1-\frac{n}{p},\, \text { if } \frac{n}{p} \text { is not an integer }} \\
  \text { any positive number}<1,\, \text { if } \frac{n}{p} \text { is an integer. }
  \end{array}\right.
  $$
  ~~We have in addition the estimate
  $$
  \bm
  {||u||_{C^{k-\left[\frac{n}{p}\right]-1, \gamma}(\bar{U})} \leq C||u||_{W^{k, p}(U)}}
  $$
  ~~the constant $C$ depending only on $k, p, n, \gamma$ and $U$.
  \end{alertblock}
\end{frame}

\begin{frame}{Sobolev Spaces on $\mathbf{R^n}$  \footnote{\scalebox{0.8}{Evans, Lawrence C. \textit{Partial differential equations.} Intersxcience Publishers, 1964. }}}
  \begin{alertblock}{Rellich-Kondrachov Compactness Theorem (Compactness)}
    Assume $U$ is a bounded open subset of $\mathbb{R}^n$, and $\partial U$ is $C^1$. Suppose $1 \leq p<n$. Then
    $$
    \bm
    {W^{1, p}(U) \subset \subset L^q(U)}
    $$
    for each $1 \leq q<p^*$, where $1 \leq p<n, p^*=\frac{p n}{n-p}$
    \end{alertblock}
    
    \vskip 5pt
    Before we start on Riemannian manifolds, we may ask:
    
    1. \textit{$\partial U$ is $C^1$} is an important assumption, what about manifolds without boundary?

    2. can we find better embedded spaces which have more compactness properties? 

    3. how to generalize the Sobolev inequalities to Riemannian manifolds?
  \end{frame}

\begin{frame}{Sobolev Spaces on $\mathbf{R^n}$ }
  Let $\Omega$ be some open subset of $\mathbb{R}^n, k$ an integer, $p \geq 1$ real, and $u: \Omega \rightarrow \mathbb{R}$ a smooth, real-valued function. We define then the Sobolev spaces
  $$
  \begin{aligned}
  H_k^p(\Omega)&=\text { the completion of }\left\{u \in C^{\infty}(\Omega) \,,\,||u||_{k, p}<+\infty\right\} \text { for }||\cdot||_{k, p} \\
  W^{k,p}(\Omega)&=\left\{u \in L^p(\Omega) \,,\, \forall|\alpha| \leq k, D_\alpha u \text { exists and belongs to } L^p(\Omega)\right\}\\
  \end{aligned}
  $$

  \vskip 5pt
  \centering{\textbf{[Meyers-Senin] For any} $\bm{\Omega}$, \textbf{any} $\bm{k}$, \textbf{and any} $\bm{p \geq 1, H_k^p(\Omega)=W^{k,p}(\Omega)}$. \footnote{\scalebox{0.8}{Adams, R. A. \textit{Sobolev spaces.} Academic Press, San Diego, 1978.}}}
  
\end{frame}

\begin{frame}{Sobolev Spaces on ($\mathbf{M,g}$) \footnote{\scalebox{0.8}{Emmanuel Hebey. \textit{Nonlinear analysis on manifolds : Sobolev spaces and inequalities.} 1999.}} }
  \begin{alertblock}{Sobolev Spaces on ($\mathbf{M,g}$) }
    Given $(M, g)$ a smooth Riemannian manifold, $k$ an integer, and $p \geq 1$ real, the Sobolev space $\bm{H_k^p(M)}$ \textbf{is the completion of} $\bm{\mathcal{C}_k^p(M)}$ \textbf{with respect to} $\bm{||\cdot||_{H_k^p}}$, where
    \vskip -1pt
    $$
    \mathcal{C}_k^p(M)=\left\{u \in C^{\infty}(M) \,,\, \forall j=0, \ldots, k, \int_M\left|\nabla^j u\right|^p d v(g)<+\infty\right\}
    $$
    When $M$ is compact, one clearly has that $\mathcal{C}_k^p(M)=\mathcal{C}^{\infty}(M)$ for any $k$ and any $p \geq 1$. For $u \in \mathcal{C}_k^p(M)$, set also
    \vskip -1pt
    $$
    ||u||_{H_k^p}=\sum_{j=0}^k\left(\int_M\left|\nabla^j u\right|^p d v(g)\right)^{1 / p}
    $$
  \end{alertblock}
\end{frame}

%-----------------------------------------------------------------------

\section{Compact Manifolds without Boundary}

\begin{frame}{Compact Manifolds without Boundary \footnote{\scalebox{0.8}{Emmanuel Hebey. \textit{Nonlinear analysis on manifolds : Sobolev spaces and inequalities.} 1999.}} }
  \begin{alertblock}{General Sobolev Inequalities on Compact Manifolds without Boundary}
    Let $(M, g)$ be a smooth, compact Riemannian n-manifold.
    Let $p\geq 1$ be real and $0 \leq m<k$ be two integers.
    
    \vskip 5pt
    ~~~~~~\textbf{(A1) If} $\bm{1/q=1 / p-(k-m) / n > 0}$\textbf{, then} $\bm{H_k^p(M) \subset H_m^q(M)}$.
    
    \vskip 5pt
    ~~~~~~\textbf{(A2) If} $\bm{1/q=1 / p-(k-m) / n < 0}$\textbf{, then} $\bm{H_k^p(M) \subset C^m(M)}$.
    
    \vskip 5pt
    where
    $$
    ||u||_{H_k^p}=\sum_{j=0}^k\left(\int_M\left|\nabla^j u\right|^p d v(g)\right)^{1 / p}\,\,\text{and}\,\, ||u||_{C^m}=\sum_{j=0}^m \max _{x \in M}\left|\left(\nabla^j u\right)(x)\right|
    $$
  \end{alertblock}
\end{frame}

\begin{frame}{Compact Manifolds without Boundary \footnote{\scalebox{0.8}{Emmanuel Hebey. \textit{Nonlinear analysis on manifolds : Sobolev spaces and inequalities.} 1999.}} }
  \begin{alertblock}{General Sobolev Embeddings on Compact Manifolds without Boundary}
    Let $(M, g)$ be a smooth, compact Riemannian $n$-manifold.
    Let $p\geq 1$ be real and $0 \leq m<k$ be two integers.
    
    \vskip 5pt
    ~~~~\textbf{(A1)  For any real} $\bm{q}$ \textbf{such that} $\bm{1 \leq q<\frac{n p}{(n-(k-m) p)}}$,\, $\bm{H_{k}^p(M)\subset \subset H_m^q(M)}$.
    
    \vskip 5pt
    Take $k=1,m=0$, for any $p<n$ real and any $1\leq q <\frac{np}{n-p}$, $H_1^p(M) \subset \subset L^q(M)$.
    
    \vskip 5pt
    ~~~~\textbf{(A2) Take} $\bm{k=1,m=0,}$ \textbf{for} $\bm{p>n}$ \textbf{and for any} $\bm{\lambda \in (0,1)}$, \textbf{such that} 
    
    \vskip 5pt
    $\bm{(1-\lambda) p>n}$, $\bm{H_1^p(M)\subset \subset C^\lambda(M)}$\textbf{. Particularly,} $\bm{H_1^p(M)\subset \subset C^0(M)}$.
    $$
    ||u||_{C^\lambda}=\max _{x \in M}|u(x)|+\max _{x \neq y \in M} \frac{|u(y)-u(x)|}{d_g(x, y)^\lambda}
    $$
  \end{alertblock}
\end{frame}

%-----------------------------------------------------------------------

\section{Complete Manifolds without Boundary}

\begin{frame}{Complete Manifolds without Boundary \footnote{\scalebox{0.8}{Emmanuel Hebey. \textit{Nonlinear analysis on manifolds : Sobolev spaces and inequalities.} 1999.}} }
  \begin{alertblock}{General Sobolev Inequalities on Complete Manifolds without Boundary}
  Let $(M, g)$ be a smooth, complete Riemannian $n$-manifold with Ricci curvature bounded from below. Assume that
  $$
  \inf _{x \in M} \operatorname{Vol}_g\left(B_x(1)\right)>0
  $$
  where $\operatorname{Vol}_g\left(B_x(1)\right)$ stands for the volume of $B_x(1)$ with respect to $g$. Then the Sobolev embeddings in their first part (A1) are valid for $(M, g)$. i.e.
  
  \vskip 5pt
  ~~~~~~\textbf{(B1) If} $\bm{1/q=1 / p-(k-m) / n > 0}$\textbf{, then} $\bm{H_k^p(M) \subset H_m^q(M)}$.
\end{alertblock}
The assumption of Ricci curvature is satisfactory but not necessary. e.g. $H_1^p\subset L^q$
\end{frame}

\begin{frame}{Complete Manifolds without Boundary \footnote{\scalebox{0.8}{Emmanuel Hebey. \textit{Nonlinear analysis on manifolds : Sobolev spaces and inequalities.} 1999.}} }
  \begin{alertblock}{General Sobolev Inequalities on Complete Manifolds without Boundary}
    Let $(M, g)$ be a smooth, complete Riemannian $n$-manifold with Ricci curvature bounded from below and positive injectivity radius. 
    
    \vskip 5pt
    For $p \geq 1$ real and $0\leq m<k$ two integers,   
  
    \vskip 8pt
  ~~~~~~\textbf{(B2) If} $\bm{1/q=1 / p-(k-m) / n < 0}$\textbf{, then} $\bm{H_k^p(M) \subset C_B^m(M)}$.
  
    \vskip 8pt
    where $C_B^m(\Omega)$ consists of the functions $u \in C^m(\Omega)$ and $\nabla^ju$ is bounded on $M$ for $0\leq |j|\leq m$. 
    Particularly, for $p \geq 1$ real and $\lambda \in(0,1)$ real, if $1 / p \leq(1-\lambda) / n$, then 
    
    \vskip 3pt
    $\bm{H_1^p(M) \subset C_B^\lambda(M)}$.



\end{alertblock}
\end{frame}

%----------------------------------------------------------------------

\section{Compact Manifolds with Boundary}

\begin{frame}{Compact Manifold with Boundary \footnote{\scalebox{0.8}{Emmanuel Hebey. \textit{Nonlinear analysis on manifolds : Sobolev spaces and inequalities.} 1999.}} }
  \begin{alertblock}{Sobolev Inequalities on Compact Manifolds with Boundary}
  Let $(M, g)$ be a smooth, compact, $n$-dimensional Riemannian manifold with boundary.
  
  For $p<n$ real, set $p^*=n p /(n-p)$. Then for any $q \in[1, p^*],\, \bm{H_1^p(M) \subset L^q(M)}$ 
  \end{alertblock}

  \vskip 8pt
  \begin{alertblock}{Sobolev Embeddings on Compact Manifolds with Boundary}
    If $q \in[1, p^*)$, the embedding above is compact, i.e. $\bm{H_1^p(M) \subset \subset L^q(M)}$.

  \end{alertblock}
\end{frame}


%=======================================================================
\makebottom

%\begin{frame}{Additional Definitions about Riemannian Manifolds  }
%  \textbf{Injectivity Radius :}
  
%  \vskip 5pt
%  Let $(M, g)$ be a Riemannian manifold without boundary.

%  Let $T_p M$ be the tangent space of $M$ at $p \in M$.
%
% Let $\exp _p$ be the restricted exponential map at $p \in M$.

%  Let $B_a(0) \in T_p M$ be an open ball.

%  \vskip 5pt
%  Suppose $A_p$ is the set of all $a \in \mathbb{R}_{>0}$ for which $\exp _p$ is a diffeomorphism from $B_a(0) \subseteq T_p M$ onto its image.
%  Then the supremum of $A_p$ is called the injectivity radius of $M$ at $p$ and is denoted by $\operatorname{inj}(p)$ : $\operatorname{inj}(p)=\sup A_p$ 

% \vskip 5pt
%  One can then define the (global) injectivity radius by
%  $\operatorname{inj}_{(M . g)}=\inf _{x \in M} \operatorname{inj}_{(M . g)}(x)$

%  One has that inj ${ }_{(M . g)}>0$ for a compact manifold, but it may be zero for a complete noncompact manifold. 
%\end{frame}

\end{document}
