%!TeX program = xelatex
\documentclass[12pt,hyperref,a4paper,UTF8]{ctexart}
\usepackage{SJTUReport}
\ctexset{bibname = References}
\renewcommand\contentsname{Contents}


%%-------------------------------正文开始---------------------------%%
\begin{document}

%%-----------------------封面--------------------%%
\cover

%%------------------摘要-------------%%
\begin{abstract}
    This paper is devoted to the study of Sobolev embeddings in different settings of Riemannian manifolds, including compact manifolds without boundary, complete manifolds without boundary and compact manifolds with boundary. Main results of this paper can be refered to \autoref{Sec6}. 
\end{abstract}

\thispagestyle{empty} % 首页不显示页码

%%--------------------------目录页------------------------%%
\newpage
\tableofcontents

%%------------------------正文页从这里开始-------------------%
\newpage
\section{Introduction}
This paper is devoted to the study of Sobolev embeddings in different settings of Riemannian manifolds. 
Though Riemannian manifolds are natural extensions of Euclidean space, the naive idea that what is valid for Euclidean space must be valid for manifolds is false. Several surprising phenomena appear when studying Sobolev spaces on manifolds.

The study of Sobolev spaces on Riemannian manifolds is a field fast developing in the end of the 20th century. 
Nevertheless, several important questions still puzzle mathematicians today. While the theory of Sobolev spaces for noncompact manifolds has its origin in the 1970s with the work of Aubin, Cantor, Hoffman, and Spruck, many of the results presented in this paper have been obtained in the 1980s and 1990s.

The present paper is organized into six sections. Section 1 briefly introduces the background of the subject. Section 2 is a quick introduction to differential and Riemannian geometry.
Section 3 deals with the general theory of Sobolev spaces for compact manifolds without boundary, while Section 4 deals with the general theory of Sobolev spaces for complete, noncompact manifolds without boundary.
Section 5 introduces the results for compact manifolds with boundary. In the end, Section 6 summaries the main results of this paper and gives some perspectives. 
%----------------------------------
\newpage
\section{Elements of Riemannian Geometry}
In this section, we will firstly introduce some basic concepts of Riemannian geometry, including smooth manifolds, Riemannian manifolds, methods for global analysis, and some properties under some special coordinates.

\subsection{Smooth Manifolds}

\subsubsection{Smooth Manifolds}
Let $M$ be a Hausdorff topological space. We say that $M$ is a topological manifold of dimension $n$ if each point of $M$ possesses an open neighborhood that is homeomorphic to some open subset of the Euclidean space $\mathbb{R}^n$. A chart of $M$ is then a couple $(\Omega, \varphi)$ where $\Omega$ is an open subset of $M$, and $\varphi$ is a homeomorphism of $\Omega$ onto some open subset of $\mathbb{R}^n$. For $y \in \Omega$, the coordinates of $\varphi(y)$ in $\mathbb{R}^n$ are said to be the coordinates of $y$ in $(\Omega, \varphi)$. An atlas of $M$ is a collection of charts $\left(\Omega_i, \varphi_i\right), i \in I$, such that $M=\bigcup_{i \in I} \Omega_i$. Given $\left(\Omega_i, \varphi_i\right)_{i \in I}$ an atlas, the transition functions are
$$
\varphi_j \circ \varphi_i^{-1}: \varphi_i\left(\Omega_i \cap \Omega_j\right) \rightarrow \varphi_j\left(\Omega_i \cap \Omega_j\right)
$$
with the obvious convention that we consider $\varphi_j \circ \varphi_i^{-1}$ if and only if $\Omega_i \cap \Omega_j \neq \emptyset$. The atlas is then said to be of class $C^k$ if the transition functions are of class $C^k$, and it is said to be $C^k$-complete if it is not contained in a (strictly) larger atlas of class $C^k$. As one can easily check, every atlas of class $C^k$ is contained in a unique $C^k$-complete atlas.

\vskip 3pt
For our purpose, we will always assume in what follows that $k=+\infty$ and that $M$ is connected. One then gets the following definition of a smooth manifold: A smooth manifold $M$ of dimension $n$ is a connected topological manifold $M$ of dimension $n$ together with a $C^{\infty}$-complete atlas.

\subsubsection{Maps and Ranks}

Given $M$ and $N$ two smooth manifolds, and $f: M \rightarrow N$ some map from $M$ to $N$, we say that $f$ is differentiable (or of class $C^k$ ) if for any charts $(\Omega, \varphi)$ and $(\tilde{\Omega}, \tilde{\varphi})$ of $M$ and $N$ such that $f(\Omega) \subset \tilde{\Omega}$, the map
$$
\tilde{\varphi} \circ f \circ \varphi^{-1}: \varphi(\Omega) \rightarrow \tilde{\varphi}(\tilde{\Omega})
$$
is differentiable (or of class $C^k$ ). In particular, this allows us to define the notion of diffeomorphism and the notion of diffeomorphic manifolds. Independently, one can define the rank $R(f)_x$ of $f$ at some point $x$ of $M$ as the rank of $\tilde{\varphi} \circ f \circ \varphi^{-1}$ at $\varphi(x)$, where $(\Omega, \varphi)$ and $(\tilde{\Omega}, \tilde{\varphi})$ are as above, with the additional property that $x \in \Omega$. This is an intrinsic definition in the sense that it does not depend on the choice of the charts. The map $f$ is then said to be an immersion if, for any $x \in M$, $R(f)_x=m$, where $m$ is the dimension of $M$, and a submersion if for any $x \in M$, $R(f)_x=n$, where $n$ is the dimension of $N$. It is said to be an embedding if it is an immersion that realizes a homeomorphism onto its image.
\vskip 3pt
Up to now, we have adopted the abstract definition of a manifold. As a surface gives the idea of a two-dimensional manifold, a more concrete approach would have been to define manifolds as submanifolds of Euclidean space. Given $M$ and $N$ two manifolds, one will say that $N$ is a submanifold of $M$ if there exists a smooth embedding $f: N \rightarrow M$. According to a well-known result of Whitney, the two approaches (concrete and abstract) are equivalent, at least when dealing with paracompact manifolds, since for any paracompact manifold $M$ of dimension $n$, there exists a smooth embedding $f: M \rightarrow \mathbb{R}^{2 n+1}$. In other words, any paracompact (abstract) manifold of dimension $n$ can be seen as a submanifold of some Euclidean space.

\subsubsection{Tangent Spaces and Tangent Bundles}
Let us now say some words about the tangent space of a manifold. Given $M$ a smooth manifold and $x \in M$, let $\mathcal{F}_x$ be the vector space of functions $f: M \rightarrow \mathbb{R}$ which are differentiable at $x$. For $f \in \mathcal{F}_x$, we say that $f$ is flat at $x$ if for some chart $(\Omega, \varphi)$ of $M$ at $x, D\left(f \circ \varphi^{-1}\right)_{\varphi(x)}=0$. Let $\mathcal{N}_x$ be the vector space of such functions. A linear form $X$ on $\mathcal{F}_x$ is then said to be a tangent vector of $M$ at $x$ if $\mathcal{N}_x \subset \operatorname{Ker} X$. We let $T_x(M)$ be the vector space of such tangent vectors. Given
$(\Omega, \varphi)$ some chart at $x$, of associated coordinates $x^i$, we define $\left(\frac{\partial}{\partial x_i}\right)_x \in T_x(M)$ by: for any $f \in \mathcal{F}_x$,
$$
\left(\frac{\partial}{\partial x_i}\right)_x \cdot(f)=D_i\left(f \circ \varphi^{-1}\right)_{\varphi(x)}
$$
As a simple remark, one gets that the $\left(\frac{\partial}{\partial x_i}\right)_x$ 's form a basis of $T_{{x}}(M)$. Now, one defines the tangent bundle of $M$ as the disjoint union of the $T_x(M)$ 's, $x \in M$. If $M$ is $n$-dimensional, one can show that $T(M)$ possesses a natural structure of a $2 n$-dimensional smooth manifold. Given $(\Omega, \varphi)$ a chart of $M$,
$$
\left(\; \bigcup_{x \in \Omega} T_x(M), \Phi\; \right)
$$
is a chart of $T(M)$, where for $X \in T_{{x}}(M), x \in \Omega$,
$$
\Phi(X)=\left(\varphi^{\prime}(x), \ldots, \varphi^n(x), X\left(\varphi^1\right), \ldots, X\left(\varphi^n\right)\right)
$$
(the coordinates of $x$ in $(\Omega, \varphi)$ and the components of $X$ in $(\Omega, \varphi)$, that is, the coordinates of $X$ in the basis of $T_x(M)$ associated to $(\Omega, \varphi)$ by the process described above). By definition, a vector field on $M$ is a map $X: M \rightarrow T(M)$ such that for any $x \in M, X(x) \in T_x(M)$. Since $M$ and $T(M)$ are smooth manifolds, the notion of a vector field of class $C^k$ makes sense.

\vskip 3pt
Given $M, N$ two smooth manifolds, $x$ a point of $M$, and $f: M \rightarrow N$ differentiable at $x$, the tangent linear map of $f$ at $x$ (or the differential map of $f$ at $x)$, denoted by $f_*(x)$, is the linear map from $T_{{x}}(M)$ to $T_{f({x})}(N)$ defined by: For $X \in T_x(M)$ and $g: N \rightarrow \mathbb{R}$ differentiable at $f(x)$,
$$
\left(f_*(x) \cdot(X)\right) \cdot(g)=X(g \circ f)
$$
By extension, if $f$ is differentiable on $M$, one gets the tangent linear map of $f$, denoted by $f_*$. That is the map $f_*: T(M) \rightarrow T(N)$ defined by: For $X \in T_x(M)$, $f_*(X)=f_*(x) .(X)$. As one can easily check, $f_*$ is $C^{k-1}$ if $f$ is $C^k$. For $f: M_1 \rightarrow$ $M_2, g: M_2 \rightarrow M_3$, and $x \in M_1,(g \circ f)_*(x)=g_*(f(x)) \circ f_*(x)$.

\vskip 3pt
Similar to the construction of the tangent bundle, one can define the cotangent bundle of a smooth manifold $M$. For $x \in M$, let $T_{{x}}(M)^*$ be the dual space of $T_x(M)$. If $(\Omega, \varphi)$ is a chart of $M$ at $x$ of associated coordinates $x^i$, one gets a basis of $T_x(M)^*$ by considering the $d x_{{x}}^i$ 's defined by $d x_{{x}}^i \cdot\left(\frac{\partial}{\partial {x}_j}\right)_x=\delta_j^i$. As for the tangent bundle, the cotangent bundle of $M$, denoted by $T^*(M)$, is the disjoint union of the $T_x(M)^*$ 's, $x \in M$. Here again, if $M$ is $n$-dimensional, $T^*(M)$ possesses a natural structure of $2 n$-dimensional smooth manifold. Given $(\Omega, \varphi)$ a chart of $M$,
$$
\left(\;\bigcup_{x \in \Omega} T_x(M)^*, \Phi\;\right)
$$
is a chart of $T(M)$, where for $\eta \in T_x(M)^*, x \in \Omega$,
$$
\Phi(\eta)=\left(\varphi^{\prime}(x), \ldots, \varphi^n(x), \eta\left(\frac{\partial}{\partial x_1}\right)_x, \ldots, \eta\left(\frac{\partial}{\partial x_n}\right)_x\right)
$$
(the coordinates of $x$ in $(\Omega, \varphi)$ and the components of $\eta$ in $(\Omega, \varphi)$, that is, the coordinates of $\eta$ in the basis of $T_x(M)^*$ associated to $(\Omega, \varphi)$ by the process described
above). By definition, a 1 -form on $M$ is a map $\eta: M \rightarrow T^{\star}(M)$ such that for any $x \in M, \eta(x) \in T_x(M)^*$. Here again, since $M$ and $T^*(M)$ are smooth manifolds, the notion of a 1 -form of class $C^k$ makes sense. For $f$ a function of class $C^k$ on $M$, let $d f$ be defined by: For $x \in M$ and $X \in T_x(M), d f(x) \cdot X=X(f)$. Then $d f$ is a 1-form of class $C^{k-1}$.

\subsubsection{$q-$Forms and $(p,q)-$Tensors}
Given $M$ a smooth $n$-manifold, and $1 \leq q \leq n$ an integer, let $\wedge^q T_x(M)^*$ be the space of skew-symmetric $q$-linear forms on $T_x(M)$. If $(\Omega, \varphi)$ is a chart of $M$ at $x$, of associated coordinates $x^i,\left\{d x_x^{i_1} \wedge \cdots \wedge d x_x^{i_q}\right\}_{i_1<\cdots<i_q}$ is a basis of $\bigwedge^q T_x(M)^*$. With similar constructions to the ones made above, one gets that $\bigwedge^q(M)$, the disjoint union of the $\bigwedge^q T_x(M)^{\star}$'s, possesses a natural structure of a smooth manifold. Its dimension is $n+C_n^q$, where $C_n^q=n!/(q!(n-q)!)$. Some map $\eta: M \rightarrow \bigwedge^q(M)$ is then said to be an exterior form of degree $q$, or just an exterior $q$-form, if for any $x \in M, \eta(x) \in \bigwedge^q T_x(M)^*$. Here again, the notion of an exterior $q$-form of class $C^k$ makes sense. Given $(\Omega, \varphi)$ some chart of $M$, and $\eta$ a $q$-form of class $C^k$ whose expression in $(\Omega, \varphi)$ is
$$
\eta=\sum_{i_1<\cdots<i_q} \eta_{i_1 \ldots i_q} d x^{i_1} \wedge \cdots \wedge d x^{i_q}
$$
the exterior derivative of $\eta$, denoted by $d \eta$, is the exterior $(q+1)$-form of class $C^{k-1}$ whose expression in $(\Omega, \varphi)$ is
$$
d \eta=\sum_{i_1<\cdots<i_q} d \eta_{i_1 \ldots i_q} \wedge d x^{i_1} \wedge \cdots \wedge d x^{i_q}
$$
One then gets that for any exterior $q$-form $\eta, d(d \eta)=0$. Conversely, by the Poincaré lemma, if $\eta$ is an exterior $q$-form such that $d \eta=0$, that is, a closed exterior $q$-form, around any point in $M$, there exists an exterior $(q-1)$-form $\tilde{\eta}$ such that $d \tilde{\eta}=\eta$. One says that a closed exterior form is locally exact.

\vskip 3pt
As another generalization, given $M$ a smooth $n$-manifold, $x$ some point of $M$, and $p, q$ two integers, one can define $T_p^q\left(T_x(M)\right)$ as the space of $(p, q)$-tensors on $T_x(M)$, that is, the space of $(p+q)$-linear forms
$$
\eta: \underbrace{T_x(M) \times \cdots \times T_x(M)}_p \times \underbrace{T_x(M)^{\star} \times \cdots \times T_x(M)^{\star}}_q \rightarrow \mathbb{R}
$$
An element of $T_p^q\left(T_x(M)\right)$ is said to be $p$-times covariant and $q$-times contravariant. If $(\Omega, \varphi)$ is a chart of $M$ at $x$, of associated coordinates $x^i$, the family
$$
\qquad\qquad\left\{d x_x^{i_1} \otimes \cdots \otimes d x_x^{i_p} \otimes\left(\frac{\partial}{\partial x_{j_1}}\right)_x \otimes \cdots \otimes\left(\frac{\partial}{\partial x_{j_q}}\right)_x\right\}_{i_1 \ldots \ldots i_p, j_1 \ldots \ldots j_q}
$$
is a basis of $T_p^q\left(T_x(M)\right)$. Here again, one gets that the disjoint union $T_p^q(M)$ of the $T_p^q\left(T_x(M)\right)$ 's possesses a natural structure of a smooth manifold. Its dimension is $n\left(1+n^{p+q-1}\right)$. A map $T: M \rightarrow T_p^q(M)$ is then said to be a $(p, q)$-tensor field on $M$ if for any $x \in M, T(x) \in T_p^q\left(T_x(M)\right)$. It is said to be of class $C^k$ if it is of class $C^k$ from the manifold $M$ to the manifold $T_p^q(M)$. Given $(\Omega, \varphi)$ and $(\Omega, \psi)$ two charts of $M$ of associated coordinates $x^i$ and $y^i$, and $T$ a $(p, q)$-tensor field, let us
denote by $T_{i_1 \ldots i_p}^{j_1 \ldots j_q}$ and $\tilde{T}_{i_1 \ldots i_p}^{j_1 \ldots j_q}$ its components in $(\Omega, \varphi)$ and $(\Omega, \psi)$. Then, for any $i_1, \ldots, i_p, j_1, \ldots, j_q$, and any $x \in \Omega$,
$$
\tilde{T}_{i_1, i_p}^{j_1 \ldots j_q}(x)=T_{\alpha_1 \ldots \alpha_p}^{\beta_1 \ldots \beta_q}(x)\left(\frac{\partial x^{\alpha_1}}{\partial y_{i_1}}\right)_x \ldots\left(\frac{\partial x^{\alpha_p}}{\partial y_{i_p}}\right)_x\left(\frac{\partial y^{j_1}}{\partial x_{\beta_1}}\right)_x \ldots\left(\frac{\partial y^{j_q}}{\partial x_{\beta_4}}\right)_x
$$
As a remark, given $M$ and $N$ two manifolds, $f: M \rightarrow N$ a map of class $C^{k+1}$, and $T$ a $(p, 0)$-tensor field of class $C^k$ on $N$, one can define the pullback $f^{\star} T$ of $T$ by $f$, that is, the $(p, 0)$-tensor field of class $C^k$ on $M$ defined by: For $x \in M$ and $X_1, \ldots, X_p \in T_r(M)$,
$$
\left(f^* T\right)(x) \cdot\left(X_1, \ldots, X_p\right)=T(f(x)) \cdot\left(f_*(x) \cdot X_1, \ldots, f_*(x) \cdot X_p\right)
$$
As one can easily check, for $f: M_1 \rightarrow M_2$ and $g: M_2 \rightarrow M_3,(g \circ f)^{\star}=f^{\star} \circ g^*$.

\subsubsection{Linear Connections}
Let us now define the notion of a linear connection. Denote by $\Gamma(M)$ the space of differentiable vector fields on $M$. A linear connection $D$ on $M$ is a map
$$
D: T(M) \times \Gamma(M) \rightarrow T(M)
$$
such that

1. $\forall x \in M, \forall X \in T_x(M), \forall Y \in \Gamma(M), D(X, Y) \in T_x(M)$,

2. $\forall x \in M, D: T_x(M) \times \Gamma(M) \rightarrow T_x(M)$ is bilinear,

3. $\forall x \in M, \forall X \in T_x(M), \forall f: M \rightarrow \mathbb{R}$ differentiable, $\forall Y \in \Gamma(M)$, $D(X, f Y)=$

$\quad\, X(f) Y(x)+f(x) D(X, Y)$, and

4. $\forall X, Y \in \Gamma(M)$, and $\forall k$ integer, if $X$ is of class $C^k$ and $Y$ is of class $C^{k+1}$, then 

$\quad\, D(X, Y)$ is of class $C^k$, where $D(X, Y)$ is the vector field $x \rightarrow$ $D(X(x), Y)$

\vskip 3pt
\noindent 
Given $D$ a linear connection, the usual notation for $D(X, Y)$ is $D_X(Y)$. One says that $D_X(Y)$ is the covariant derivative of $Y$ with respect to $X$. Let $(\Omega, \varphi)$ be a chart of $M$ of associated coordinates $x^i$. Set
$$
\nabla_i=D_{\left(\frac{\partial}{\partial x_i}\right)}
$$
As one can easily check, there exist $n^3$ smooth functions $\Gamma_{i j}^k: \Omega \rightarrow \mathbb{R}$ such that for any $i, j$, and any $x \in \Omega$,
$$
\nabla_i\left(\frac{\partial}{\partial x_j}\right)(x)=\Gamma_{i j}^k(x)\left(\frac{\partial}{\partial x_k}\right)_x
$$
Such functions, the Christoffel symbols of $D$ in $(\Omega, \varphi)$, characterize the connection in the sense that for $X \in T_x(M), x \in \Omega$, and $Y \in \Gamma(M)$,
$$
\qquad D_X(Y)=X^i\left(\nabla_i Y\right)(x)=X^i\left(\left(\frac{\partial Y^j}{\partial x_i}\right)_x+\Gamma_{i \alpha}^j(x) Y^\alpha(x)\right)\left(\frac{\partial}{\partial x_j}\right)_x
$$
where the $X^i$ 's and $Y^i$ 's denote the components of $X$ and $Y$ in the chart $(\Omega, \varphi)$, and for $f: M \rightarrow \mathbb{R}$ differentiable at $x$,
$$
\quad \left(\frac{\partial f}{\partial x_i}\right)_x=D_i\left(f \circ \varphi^{-1}\right)_{\varphi(x)}
$$
As one can easily check, since (1.1) is not satisfied by the $\Gamma_{i j}^k$ 's, the $\Gamma_{i j}^k$ 's are not the components of a $(2,1)$-tensor field. An important remark is that linear connections
have natural extensions to differentiable tensor fields. Given $T$ a differentiable $(p, q)$-tensor field, $x$ a point of $M, X \in T_x(M)$, and $(\Omega, \varphi)$ a chart of $M$ at $x$, $D_X(T)$ is the $(p, q)$-tensor on $T_x(M)$ defined by $D_X(T)=X^i\left(\nabla_i T\right)(x)$, where
$$
\begin{aligned}
\left(\nabla_i T\right)(x)_{i_1 \ldots i_p}^{j_1 \ldots j_q}= & \left(\frac{\partial T_{i_1 \ldots i_p}^{j_1 \ldots j_q}}{\partial x_i}\right)_x-\sum_{k=1}^p \Gamma_{ii_k}^\alpha(x) T(x)_{i_1 \ldots i_{-1} \alpha i_{k+1} \ldots i_p}^{j_1 \ldots j_q} \\
& +\sum_{k=1}^q \Gamma_{i \alpha}^{j_k}(x) T(x)_{i_1 \ldots i_p}^{j_1 \ldots j_{k-1} \alpha j_{k+1} \ldots j_q}
\end{aligned}
$$
The covariant derivative commutes with the contraction in the sense that
$$
D_X\left(C_{k_1}^{k_2} T\right)=C_{k_1}^{k_2} D_X(T)
$$
where $C_{k_1}^{k_2} T$ stands for the contraction of $T$ of order $\left(k_1, k_2\right)$. More, for $X \in T_{{x}}(M)$, and $T$ and $\tilde{T}$ two differentiable tensor fields, one has that
$$
D_X(T \otimes \tilde{T})=\left(D_X(T)\right) \otimes \tilde{T}(x)+T(x) \otimes\left(D_X(\tilde{T})\right)
$$
Given $T$ a $(p, q)$-tensor field of class $C^{k+1}$, we let $\nabla T$ be the $(p+1, q)$-tensor field of class $C^k$ whose components in a chart are given by
$$
(\nabla T)_{i_1 \ldots i_{p+1}}^{j_1 \ldots j_q}=\left(\nabla_{i_1} T\right)_{i_2 \ldots i_{p+1}}^{j_1 \ldots j_q}
$$
By extension, one can then define $\nabla^2 T, \nabla^3 T$, and so on. For $f: M \rightarrow \mathbb{R}$ a smooth function, one has that $\nabla f=d f$ and, in any chart $(\Omega, \varphi)$ of $M$,
$$
\left(\nabla^2 f\right)(x)_{i j}=\left(\frac{\partial^2 f}{\partial x_i \partial x_j}\right)_x-\Gamma_{i j}^k(x)\left(\frac{\partial f}{\partial x_k}\right)_x
$$
where
$$
\left(\frac{\partial^2 f}{\partial x_i \partial x_j}\right)_x=D_{i j}^2\left(f \circ \varphi^{-1}\right)_{\varphi(x)}
$$
In the Riemannian context, $\nabla^2 f$ is called the Hessian of $f$ and is sometimes denoted by Hess $(f)$.

\subsubsection{Torsion and Curvature}
Finally, let us define the torsion and the curvature of a linear connection $D$. The torsion $T$ of $D$ can be seen as the smooth $(2,1)$-tensor field on $M$ whose components in any chart are given by the relation $T_{i j}^k=\Gamma_{i j}^k-\Gamma_{j i}^k$. One says that the connection is torsion-free if $T \equiv 0$. The curvature $R$ of $D$ can be seen as the smooth $(3,1)$-tensor field on $M$ whose components in any chart are given by the relation
$$
R_{i j k}^l=\frac{\partial \Gamma_{k i}^l}{\partial x_j}-\frac{\partial \Gamma_{j i}^l}{\partial x_k}+\Gamma_{j \alpha}^l \Gamma_{k i}^\alpha-\Gamma_{k \alpha}^l \Gamma_{j i}^\alpha
$$
As one can easily check, $R_{i j k}^l=-R_{i k j}^l$. Moreover, when the connection is torsionfree, one has that
$$
\begin{array}{l}
R_{i j k}^l+R_{k i j}^l+R_{j k i}^l=0 \\
\left(\nabla_i R\right)_{m j k}^l+\left(\nabla_k R\right)_{m i j}^l+\left(\nabla_j R\right)_{m k i}^l=0
\end{array}
$$
Such relations are referred to as the first Bianchi's identity, and the second Bianchi's identity.

%============================================================================

\subsection{Riemannian Manifolds}

\subsubsection{Riemannian Manifolds}
Let $M$ be a smooth manifold. A Riemannian metric $g$ on $M$ is a smooth $(2,0)$ tensor field on $M$ such that for any $x \in M, g(x)$ is a scalar product on $T_x(M)$. A smooth Riemannian manifold is a pair $(M, g)$ where $M$ is a smooth manifold and $g$ a Riemannian metric on $M$. According to Whitney, for any paracompact smooth $n$-manifold there exists a smooth embedding $f: M \rightarrow \mathbb{R}^{2 n+1}$. One then gets that any smooth paracompact manifold possesses a Riemannian metric. Just think to $g=f^* e, e$ the Euclidean metric. Two Riemannian manifolds $\left(M_1, g_1\right)$ and $\left(M_2, g_2\right)$ are said to be isometric if there exists a diffeomorphism $f: M_1 \rightarrow M_2$ such that $f^* g_2=g_1$.

\vskip 3pt
Given $(M, g)$ a smooth Riemannian manifold, and $\gamma:[a, b] \rightarrow M$ a curve of class $C^{1}$, the length of $\gamma$ is
$$
L(\gamma)=\int_a^b \sqrt{g(\gamma(t)) \cdot\left(\left(\frac{d \gamma}{d t}\right)_t,\left(\frac{d \gamma}{d t}\right)_t\right)} d t
$$
where $\left(\frac{d \gamma}{d t}\right)_t \in T_{\gamma(t)}(M)$ is such that $\left(\frac{d \gamma}{d t}\right)_t \cdot f=(f \circ \gamma)^{\prime}(t)$ for any $f: M \rightarrow \mathbb{R}$ differentiable at $\gamma(t)$. If $\gamma$ is piecewisc $C^{\prime}$, the length of $\gamma$ may be defined as the sum of the lengths of its $C^1$ pieces. For $x$ and $y$ in $M$, let $\mathcal{C}_1$. be the space of piecewise $C^1$ curves $\gamma:[a, b] \rightarrow M$ such that $\gamma(a)=x$ and $\gamma(b)=y$. Then
$$
d_g(x, y)=\inf _{\gamma \in \mathcal{C}_{xx}} L(\gamma)
$$
defines a distance on $M$ whose topology coincides with the original one of $M$. In particular, by Stone's theorem, a smooth Riemannian manifold is paracompact. By definition, $d_g$ is the distance associated to $g$.

\subsubsection{Levi-Civita Connections}
Let $(M, g)$ be a smooth Riemannian manifold. There exists a unique torsionfree connection on $M$ having the property that $\nabla g=0$. Such a connection is the Levi-Civita connection of $g$. In any chart $(\Omega, \varphi)$ of $M$, of associated coordinates $x^i$, and for any $x \in \Omega$, its Christoffel symbols are given by the relations
$$
\Gamma_{i j}^k(x)=\frac{1}{2}\left(\left(\frac{\partial g_{m j}}{\partial x_i}\right)_x+\left(\frac{\partial g_{m i}}{\partial x_j}\right)_x-\left(\frac{\partial g_{i j}}{\partial x_m}\right)_x\right) g(x)^{m k}
$$
where the $g^{i j}$ 's are such that $g_{i m} g^{m j}=\delta_i^j$. Let $R$ be the curvature of the Levi-Civita connection as introduced above. One defines:

1. the Riemann curvature $\operatorname{Rm}_{(M, g)}$ of $g$ as the smooth $(4,0)$-tensor field on $M$ whose 

$\quad\,$components in a chart are $R_{i j k l}=g_{i \alpha} R_{j k l}^\alpha$,

2. the Ricci curvature $\operatorname{R c}_{(M, g)}$ of $g$ as the smooth $(2,0)$-tensor ficld on $M$ whose 

$\quad\,$components in a chart are $R_{i j}=R_{\alpha i \beta j} g^{\alpha \beta}$, and

3. the scalar curvature $\operatorname{Scal}_{(M, g)}$ of $g$ as the smooth real-valued function on $M$ whose 

$\quad\,$expression in a chart is $\operatorname{Scal}_{(M, g)}=R_{i j} g^{i j}$.

\noindent
As one can check, in any chart,
$$
R_{i j k l}=-R_{j i k l}=-R_{i j l k}=R_{k l i j}
$$
and the two Bianchi identities are
$$
\begin{array}{l}
R_{i j k l}+R_{i l j k}+R_{i k l j}=0 \\
\left(\nabla_i \operatorname{Rm}_{(M \cdot g)}\right)_{j k l m}+\left(\nabla_m \operatorname{Rm}_{(M \cdot g)}\right)_{j k i l}+\left(\nabla_l \operatorname{Rm}_{(M . g)}\right)_{j k m i}=0
\end{array}
$$
In particular, the Ricci curvature $\operatorname{Rc}_{(M . g)}$ of $g$ is symmetric, so that in any chart $R_{i j}=R_{j i}$. For $x \in M$, let $G_x^2(M)$ be the 2-Grassmannian of $T_x(M)$. The sectional curvature $K_{(M . g)}$ of $g$ is the real-valued function defined on $\bigcup_{x \in M} G_x^2(M)$ by: For $P \in G_x^2(M)$
$$
K_{(M, g)}(P)=\frac{\operatorname{Rm}_{(M . g)}(x)(X, Y, X, Y)}{g(x)(X, X) g(x)(Y, Y)-g(x)(X, Y)^2}
$$
where $(X, Y)$ is a basis of $P$. As one can easily check, such a definition does not depend on the choice of the basis. Moreover, one can prove that the sectional curvature determines the Riemann curvature.

\subsubsection{Geodesics and Exponential Maps}
Given $(M, g)$ a smooth Riemannian manifold, and $D$ its Levi-Civita connection, a smooth curve $\gamma:[a, b] \rightarrow M$ is said to be a geodesic if for all $t$,
$$
D_{\left(\frac{d \gamma}{d t}\right),}\left(\frac{d \gamma}{d t}\right)=0
$$
This means again that in any chart, and for all $k$,
$$
\left(\gamma^k\right)^{\prime \prime}(t)+\Gamma_{i j}^k(\gamma(t))\left(\gamma^i\right)^{\prime}(t)\left(\gamma^j\right)^{\prime}(t)=0
$$
For any $x \in M$, and any $X \in T_x(M)$, there exists a unique geodesic $\gamma:[0, \varepsilon] \rightarrow$ $M$ such that $\gamma(0)=x$ and $\left(\frac{d \gamma}{d t}\right)_0=X$. Let $\gamma_{x, X}$ be this geodesic. For $\lambda>0$ real, $\gamma_{x, \lambda X}(t)=\gamma_{x, X}(\lambda t)$. Hence, for $\|X\|$ sufficiently small, where $\|\cdot\|$ stands for the norm in $T_x(M)$ associated to $g(x)$, one has that $\gamma_{x, X}$ is defined on $[0,1]$. The exponential map at $x$ is the map from a neighborhood of 0 in $T_x(M)$, with values in $M$, defined by $\exp _x(X)=\gamma_{x, X}(1)$. If $M$ is $n$-dimensional and up to the assimilation of $T_x(M)$ to $\mathbb{R}^n$ via the choice of an orthonormal basis, one gets a chart $\left(\Omega, \exp _x^{-1}\right)$ of $M$ at $x$. This chart is normal at $x$ in the sense that the components $g_{i j}$ of $g$ in this chart are such that $g_{i j}(x)=\delta_{i j}$, with the additional property that the Christoffel symbols $\Gamma_{i j}^k$ of the Levi-Civita connection in this chart are such that $\Gamma_{i j}^k(x)=0$. The coordinates associated to this chart are referred to as geodesic normal coordinates.

\subsubsection{Hopf-Rinow Theorem}
Let $(M, g)$ be a smooth Riemannian manifold. The Hopf-Rinow theorem states that the following assertions are equivalent:

1. the metric space $\left(M, d_g\right)$ is complete,

2. any closed-bounded subset of $M$ is compact,

3. there exists $x \in M$ for which $\exp _x$ is defined on the whole of $T_x(M)$, and

4. for any $x \in M, \exp _x$ is defined on the whole of $T_x(M)$.

\noindent
Moreover, one gets that any of the above assertions implies that any two points in $M$ can be joined by a minimizing geodesic. Here, a curve $\gamma$ from $x$ to $y$ is said to be minimizing if $L(\gamma)=d_g(x, y)$.

\subsubsection{Measures and Integration}
Given $(M, g)$ a smooth Riemannian $n$-manifold, one can define a natural positive Radon measure on $M$. In particular, the theory of the Lebesgue integral can be applied. For $\left(\Omega_i, \varphi_i\right)_{i \in I}$ some atlas of $M$, we shall say that a family $\left(\Omega_j, \varphi_j, \alpha_j\right)_{j \in J}$ is a partition of unity subordinate to $\left(\Omega_i, \varphi_i\right)_{i \in I}$ if the following holds:

1. $\left(\alpha_j\right)_j$ is a smooth partition of unity subordinate to the covering $\left(\Omega_i\right)_i$,

2. $\left(\Omega_j, \varphi_j\right)_j$ is an atlas of $M$, and

3. for any $j$, supp $\alpha_j \subset \Omega_j$.

\vskip 3pt
\noindent
As one can easily check, for any atlas $\left(\Omega_i, \varphi_i\right)_{i \in I}$ of $M$, there exists a partition of unity $\left(\Omega_j, \varphi_j, \alpha_j\right)_{j \in J}$ subordinate to $\left(\Omega_i, \varphi_i\right)_{i \in I}$. One can then define the Riemannian measure as follows: Given $f: M \rightarrow \mathbb{R}$ continuous with compact support, and given $\left(\Omega_i, \varphi_i\right)_{i \in I}$ an atlas of $M$,
$$
\int_M f d v(g)=\sum_{j \in J} \int_{\varphi_j\left(\Omega_i\right)}\left(\alpha_j \sqrt{|g|} f\right) \circ \varphi_j^{-1} d x
$$
where $\left(\Omega_j, \varphi_j, \alpha_j\right)_{j \in J}$ is a partition of unity subordinate to $\left(\Omega_i, \varphi_i\right)_{i \in I},|g|$ stands for the determinant of the matrix whose elements are the components of $g$ in $\left(\Omega_j, \varphi_j\right)$, and $d x$ stands for the Lebesgue volume element of $\mathbb{R}^n$. One can prove that such a construction does not depend on the choice of the atlas $\left(\Omega_i, \varphi_i\right)_{i \in I}$ and the partition of unity $\left(\Omega_j, \varphi_j, \alpha_j\right)_{j \in J}$.

\subsubsection{Laplacian Acting}
The Laplacian acting on functions of a smooth Riemannian manifold $(M, g)$ is the operator $\Delta_g$ whose expression in a local chart of associated coordinates $x^i$ is

$$
\Delta_g u=-g^{i j}\left(\frac{\partial^2 u}{\partial x_i \partial x_j}-\Gamma_{i j}^k \frac{\partial u}{\partial x_k}\right)
$$
For $u$ and $v$ of class $C^2$ on $M$, on then has the following integration hy parts formula
$$
\int_M\left(\Delta_g u\right) v d v(g)=\int_M\langle\nabla u, \nabla v\rangle d v(g)=\int_M u\left(\Delta_g v\right) d v(g)
$$
where $\langle\cdot, \cdot\rangle$ is the scalar product associated with $g$ for 1 -forms.


\subsubsection{Injectivity Radius and Cut Locus}
Coming back to geodesics, one can define the injectivity radius of $(M, g)$ at some point $x$, denoted by $\operatorname{inj}_{(M, g)}(x)$, as the largest positive real number $r$ for which any geodesic starting from $x$ and of length less than $r$ is minimizing. One can then define the (global) injectivity radius by
$$
\operatorname{inj}_{(M . g)}=\inf _{x \in M} \operatorname{inj}_{(M . g)}(x)
$$
One has that $\operatorname{inj}_{(M \cdot g)}>0$ for a compact manifold, but it may be zero for a complete noncompact manifold. More generally, one can define the cut locus $\operatorname{Cut}(x)$ of $x$ as a subset of $M$ and prove that $\operatorname{Cut}(x)$ has measure zero, that $\operatorname{inj}_{(M, g)}(x)=$ $d_g(x, \operatorname{Cut}(x))$, and that $\exp _x$ is a diffeomorphism from some star-shaped domain of $T_x(M)$ at 0 onto $M \backslash \operatorname{Cut}(x)$. In particular, one gets that the distance function $r$
to a given point is differentiable almost everywhere, with the additional property that $|\nabla r|=1$ almost everywhere.

\subsection{From Local to Global Analysis}

We prove here a packing lemma that will be used many times in the sequel. Such a lemma was first proved by Calabi under the assumptions that the sectional curvature of the manifold is bounded and that the injectivity radius of the manifold is positive.\cite{H1} By Croke's result\cite{Croke}, it was then possible to replace the assumption on the sectional curvature by a lower bound on the Ricci curvature. 
Finally, by an ingenious use of Gromov's theorem, \autoref{thm21} below, one obtains the result under the more general form of \autoref{lem21}. When we discuss Sobolev inequalities on complete manifolds, this lemma will be an important tool in the process of passing from local to global inequalities.

\subsubsection{Gromov's Volume Comparison Theorem}
As a starting point, we mention the following result, generally referred to as Gromov's volume comparison theorem. 
Details on the proof of this theorem can be refered to the excellent references of Chavel\cite{Chavel} and Gallot-Hulin-Lafontaine\cite{Gallot}.

\begin{Theorem}
    Let $(M, g)$ be a smooth, complete Riemannian n-manifold whose Ricci curvature satisfies $\operatorname{Rc}_{(M . g)} \geq(n-1) k g$ as bilinear forms, for some $k$ real. Then, for any $0<r<R$ and any $x \in M$.
    $$
    \operatorname{Vol}_g\left(B_{x}(R)\right) \leq \frac{V_k(R)}{V_k(r)} \operatorname{Vol}_g\left(B_x(r)\right)
    $$
    where $\operatorname{Vol}_g\left(B_x(t)\right)$ denotes the volume of the geodesic ball of center $x$ and radius $t$, and where $V_k(t)$ denotes the volume of a ball of radius $t$ in the complete simply connected Riemannian $n$-manifold of constant curvature $k$. In particular, for any $r>0$ and any $x \in M$, $\operatorname{Vol}_g\left(B_x(r)\right) \leq V_k(r)$.
    \label{thm21}
\end{Theorem}

\noindent
\textbf{Remark :} Let $b_n$ be the volume of the Euclidean ball of radius one. It is well-known that for any $t>0$,
$$
V_{-1}(t)=n b_n \int_0^t(\sinh s)^{n-1} d s
$$
where, according to the notation of \autoref{thm21}, $ V_{-1}(t)$ denotes the volume of a ball of radius $t$ in the simply connected hyperbolic space of dimension $n$. It is then easy to prove that for any $k \geq 0$ and any $t>0$,
$$
b_n t^n \leq V_{-k}(t) \leq b_n t^n e^{(n-1) \sqrt{k t}}
$$
One just has to note here that for $s \geq 0, s \leq \sinh s \leq s e^s$, and that if $g^{\prime}=\alpha^2 g$ are Riemannian metrics on a $n$-manifold $M$, where $\alpha$ is some positive real number,
then for any $x \in M$ and any $t>0$,
$$
\operatorname{Vol}_{g^{\prime}}\left(B_x^{\prime}(t)\right)=\alpha^n \operatorname{Vol}_g\left(B_x(t / \alpha)\right)
$$
As a consequence, by \autoref{thm21} and what we just said, we get that if $(M, g)$ is a complete Riemannian $n$-manifold whose Ricci curvature satisfies $\mathrm{R c}_{(M, g)} \geq k g$ for some $k$ real, then for any $x \in M$ and any $0<r<R$,
$$
\operatorname{Vol}_g\left(B_x(R)\right) \leq e^{\sqrt{(n-1)|k|} R}\left(\frac{R}{r}\right)^n \operatorname{Vol}_g\left(B_x(r)\right)
$$
Such an explicit inequality will be used occasionally in the sequel.

\subsubsection{Uniformly Locally Finite Covering}
Given $(M, g)$ a Riemannian manifold, we say that a family $\left(\Omega_k\right)$ of open subsets of $M$ is a uniformly locally finite covering of $M$ if the following holds: $\left(\Omega_k\right)$ is a covering of $M$, and there exists an integer $N$ such that each point $x \in M$ has a neighborhood which intersects at most $N$ of the $\Omega_k$ 's. One then has the following result:

\begin{Lemma}
    Let $(M, g)$ be a smooth, complete Riemannian $n$-manifold with Ricci curvature bounded from below by some $k$ real, and let $\rho>0$ be given. There exists a sequence $\left(x_i\right)$ of points of $M$ such that for any $r \geq \rho$ :
\vskip 3pt
    (i) the family $\left(B_{x_i}(r)\right)$ is a uniformly locally finite covering of $M$, and there is an 
    
    $\quad\;$upper bound for $N$ in terms of $n, \rho, r$, and $k$

    (ii) for any $i \neq j, B_{x_i}\left(\frac{\rho}{2}\right) \cap B_{x_j}\left(\frac{\rho}{2}\right)=\emptyset$
\vskip 3pt
\noindent where, for $x \in M$ and $r>0, B_x(r)$ stands for the geodesic ball of center $x$ and radius $r$.
\label{lem21}
\end{Lemma}

\begin{Proof}
    By \autoref{thm21} and the remark following this theorem, for any $x \in$ $M$ and any $0<r<R$,
    \begin{equation}
    \operatorname{Vol}_g\left(B_x(r)\right) \geq e^{-\sqrt{(n-1)|k|} R}\left(\frac{r}{R}\right)^n \operatorname{Vol}_g\left(B_x(R)\right)
    \label{eq1}
    \end{equation}
    Independently, we claim there exists a sequence $\left(x_i\right)$ of points of $M$ such that
    \begin{equation}
    \begin{array}{l}
    M=\bigcup_i B_{x_i}(\rho) \\
    \forall i \neq j, B_{x_i}\left(\frac{\rho}{2}\right) \cap B_{x_j}\left(\frac{\rho}{2}\right)=\emptyset
    \end{array}
    \label{eq2}
    \end{equation}
    Let
    $$
    X_\rho=\left\{\left(x_i\right)_I, x_i \in M \text {, s.t. } I \text { is countable and } \forall i \neq j, d_g\left(x_i, x_j\right) \geq \rho\right\}
    $$
    where $d_g$ is the Riemannian distance associated to $g$. As one can easily check, $X_\rho$ is partially ordered by inclusion and every chain in $X_\rho$ has an upper bound. Hence, by Zorn's lemma, $X_\rho$ contains a maximal element $\left(x_i\right)$, and $\left(x_i\right)$ satisfies \autoref{eq2}. This proves the above claim. From now on, let $\left(x_i\right)$ be such that \autoref{eq2} is satisfied. For $r>0$ and $x \in M$ we define
    $$
    I_r(x)=\left\{i \text { s.t. } x \in B_{x_i}(r)\right\}
    $$
    By \autoref{eq1} we get that for $r \geq \rho$
    $$
    \begin{aligned}
    \operatorname{Vol}_g\left(B_x(r)\right) & \geq \frac{1}{2^n} e^{-2 \sqrt{(n-1)|k|} r} \operatorname{Vol}_g\left(B_x(2 r)\right) \\
    & \geq \frac{1}{2^n} e^{-2 \sqrt{(n-1)|k|} r} \sum_{i \in I_{r}(x)} \operatorname{Vol}_g\left(B_{x_i}\left(\frac{\rho}{2}\right)\right)
    \end{aligned}
    $$
    since
    $$
    \begin{array}{l}
    \bigcup_{i \in I_r(x)} B_{x_i}\left(\frac{\rho}{2}\right) \subset B_x(2 r) \\
    B_{x_i}\left(\frac{\rho}{2}\right) \cap B_{x_j}\left(\frac{\rho}{2}\right)=\emptyset \quad \text { if } i \neq j
    \end{array}
    $$
    But, again by \autoref{eq1},
    $$
    \operatorname{Vol}_g\left(B_{x_i}(\rho / 2)\right) \geq e^{-2 \sqrt{(n-1)|k|} r}\left(\frac{\rho}{4 r}\right)^n \operatorname{Vol}_g\left(B_{x_i}(2 r)\right)
    $$
    and since for any $i \in I_r(x), B_x(r) \subset B_{x_i}(2 r)$, we get that
    $$
    \operatorname{Vol}_g\left(B_x(r)\right) \geq\left(\frac{\rho}{8 r}\right)^n e^{-4 \sqrt{(n-1)|k|} r} \operatorname{Card} I_r(x) \operatorname{Vol}_g\left(B_x(r)\right)
    $$
    where Card stands for the cardinality. As a consequence, for any $r \geq \rho$ there exists $C=C(n, \rho, r, k)$ such that for any $x \in M$, Card $I_r(x) \leq C$. Now, let $B_{x_i}(r)$ be given, $r \geq \rho$, and suppose that $N$ balls $B_{x_j}(r)$ have a nonempty intersection with $B_{x_i}(r), j \neq i$. Then, obviously, Card $I_{2 r}\left(x_i\right) \geq N+1$. Hence,
    $$
    N \leq C(n, \rho, 2 r, k)-1
    $$
    and this proves the lemma.
\end{Proof}

\subsection{Special Coordinates}

\subsubsection{Harmonic Coordinates}
Given $(M, g)$ a smooth Riemannian manifold, some chart $(\Omega, \varphi)$ of $M$ of associated coordinates $x^i$ is said to be harmonic if for any $i, \Delta_g x^i=0$, where $\Delta_g$ is the Laplacian of $g$. As one can easily check from the expression of $\Delta_g$, this means again that for any $k, g^{i j} \Gamma_{i j}^k=0$, where the $\Gamma_{i j}^k$ 's stand for the Christoffel symbols of the Levi-Civita connection in the chart. A simple assertion to prove is that for any $x$ in $M$, there exists a harmonic chart $(\Omega, \varphi)$ at $x$. This comes from the classical fact that there always exists a smooth solution of $\Delta_g u=0$ with $u(x)$ and $\partial_i u(x)$ prescribed. The solutions $y^j$ of
$$
\left\{\begin{array}{l}
\Delta_g y^j=0 \\
y^j(x)=0 \\
\partial_i y^j(x)=\delta_i^j
\end{array}\right.
$$
are then the desired harmonic coordinates. Furthermore, since composing with linear transformations do not affect the fact that coordinates are harmonic, one easily sees that we can choose the harmonic coordinate system such that $g_{i j}(x)=$ $\delta_{i j}$ for any $i, j$.

A key idea when dealing with harmonic coordinates, is that they simplify the formula for the Ricci tensor. In harmonic coordinates, one has that
$$
\quad R_{i j}=-\frac{1}{2} g^{\alpha \beta} \frac{\partial^2 g_{i j}}{\partial x_\alpha \partial x_\beta}+\cdots
$$
where the dots indicate lower-order terms involving at most one derivative of the metric. 
For our purpose, let us now define the concept of harmonic radius.
\begin{Definition}
Let $(M, g)$ be a smooth Riemannian $n$-manifold and let $x \in M$. Given $Q>1, k \in \mathbb{N}$, and $\alpha \in(0,1)$, we define the $C^{k, \alpha}$ harmonic radius at $x$ as the largest number $r_H=r_H(Q, k, \alpha)(x)$ such that on the geodesic ball $B_x\left(r_H\right)$ of center $x$ and radius $r_H$, there is a harmonic coordinate chart such that the metric tensor is $C^{k . \alpha}$ controlled in these coordinates. Namely, if $g_{i j}, i, j=1, \ldots, n$, are the components of $g$ in these coordinates, then

\vskip 3pt
1. $Q^{-1} \delta_{i j} \leq g_{i j} \leq Q \delta_{i j}$ as bilinear forms

\vskip 3pt
2. $\sum\limits_{1 \leq|\beta| \leq k} r_H^{|\beta|} \sup\limits_{y}\left|\partial_\beta g_{i j}(y)\right|+\sum\limits_{|\beta|=k} r_H^{k+\alpha} \sup\limits_{y \neq z} \frac{\left|\partial_\beta g_{i j}(z)-\partial_\beta g_{i j}(y)\right|}{d_g(y, z)^\alpha} \leq Q-1$

\vskip 3pt
\noindent
where $d_g$ is the distance associated to $g$. We now define the (global) harmonic radius $r_H(Q, k, \alpha)(M)$ of $(M, g)$ by
$$
r_H(Q, k, \alpha)(M)=\inf _{x \in M} r_H(Q, k, \alpha)(x)
$$
where $r_H(Q, k, \alpha)(x)$ is as above.
\label{def21}
\end{Definition}
As one can easily check, the function
$$
x \rightarrow r_H(Q, k, \alpha)(x)
$$
is 1-Lipschitz on $M$, since by definition, for any $x, y \in M$,
$$
r_H(Q, k, \alpha)(y) \geq r_H(Q, k, \alpha)(x)-d_g(x, y)
$$
One then gets that the harmonic radius is positive for any fixed, smooth, compact Riemannian manifold. The purpose of \autoref{thm22} below is to show that one obtains lower bounds on the harmonic radius in terms of bounds on the Ricci curvature and the injectivity radius. Roughly speaking, when changing from geodesic normal coordinates to harmonic coordinates, one controls the components of the metric in terms of the Ricci curvature instead of the whole Riemann curvature. 
Concerning its proof, let us just say that the general idea is to construct a sequence of Riemannian $n$-manifolds with harmonic radius less than or equal to 1 to prove that such a sequence converges to the Euclidean space $\mathbb{R}^n$, and to get the contradiction by noting that this would imply that the harmonic radius of $\mathbb{R}^n$ is less than or equal to 1 . (Obviously, $\mathbb{R}^n$ has an infinite harmonic radius). Key steps in such a proof are the above formula for the Ricci tensor
in harmonic coordinates, and properties of the harmonic radius when passing to the limit in a converging sequence of metrics.

\begin{Theorem}
Let $\alpha \in(0,1), Q>1, \delta>0$. Let $(M, g)$ be a smooth Riemannian n-manifold, and $\Omega$ an open subset of $M$. Set
$$
\Omega(\delta)=\left\{x \in M \text { s.t. } d_g(x, \Omega)<\delta\right\}
$$
where $d_g$ is the distance associated to $g$. Suppose that for some $\lambda$ real and some $i>0$ real, we have that for all $x \in \Omega(\delta)$,
$$
\operatorname{Rc}_{(M, g)}(x) \geq \lambda g(x) \quad \text { and } \quad \operatorname{inj}_{(M, g)}(x) \geq i
$$
Then there exists a positive constant $C=C(n, Q, \alpha, \delta, i, \lambda)$, depending only on $n$, $Q, \alpha, \delta, i$, and $\lambda$, such that for any $x \in \Omega, r_H(Q, 0, \alpha)(x) \geq C$. In addition, if instead of the bound $\operatorname{Rc}_{(M . g)}(x) \geq \lambda g(x)$ we assume that for some $k$ integer, and some positive constants $C(j)$,
$$
\left|\nabla^j \operatorname{Rc}_{(M . g)}(x)\right| \leq C(j) \quad \text { for all } j=0, \ldots, k \text { and all } x \in \Omega(\delta)
$$
then, there exists a positive constant $C=C\left(n, Q, k, \alpha, \delta, i, C(j)_{0 \leq j \leq k}\right)$, depending only on $n, Q, k, \alpha, \delta, i$, and the $C(j) ' s, 0 \leq j \leq k$, such that for any $x \in \Omega$, $r_H(Q, k+1, \alpha)(x) \geq C$.
\label{thm22}
\end{Theorem}

Let $(M, g)$ be a smooth, complete Riemannian $n$-manifold, $\alpha \in(0,1)$ real, and $Q>1$ real. Suppose that for $\lambda$ real and some $i>0$ real,
$$
\operatorname{Rc}_{(M, g)} \geq \lambda g \quad \text { and } \quad \operatorname{inj}_{(M, g)} \geq i
$$
on $M$. As an immediate consequence of \autoref{thm22}, one gets that there exists a positive constant $C=C(n, Q, \alpha, i, \lambda)$, depending only on $n, Q, \alpha, i$, and $\lambda$, such that the (global) harmonic radius of $(M, g)$ satisfies $r_H(Q, 0, \alpha)(M) \geq C$. Similarly, if instead of the bound $\operatorname{Rc}_{(M . g)} \geq \lambda g$ we assume that for some $k$ integer and some positive constants $C(j)$,
$$
\left|\nabla^j \operatorname{Rc}_{(M, g)}\right| \leq C(j) \quad \text { for all } j=0, \ldots, k
$$
then there exists a positive constant $C=C\left(n, Q, k, \alpha, i, C(j)_{0 \leq j \leq k}\right)$, depending only on $n, Q, k, \alpha, i$, and the $C(j)$ 's, $0 \leq j \leq k$, such that the (global) harmonic radius of $(M, g)$ satisfies $r_H(Q, k+1, \alpha)(M) \geq C$.

\subsubsection{Geodesic Normal Coordinates}
Coming back to geodesic normal coordinates, analogous estimates to those of \autoref{thm22} are available. Such estimates are rougher. On the one hand, they involve the Riemann curvature instead of the Ricci curvature. On the other hand, one recovers the type of phenomena: Changing from harmonic coordinates to geodesic normal coordinates involves loss of derivatives. Nevertheless, such results are sometimes useful, because of special properties that geodesic normal coordinates have with respect to harmonic coordinates. For the sake of clarity, when dealing with geodesic normal coordinates, we will restrict ourselves to the following result.
\begin{Theorem}
    Let $(M, g)$ be a smooth Riemannian n-manifold. Suppose that for some point $x \in M$ there exist positive constants $\Lambda_1$ and $\Lambda_2$ such that
$$
\left|\mathrm{Rm}_{(M, g)}\right| \leq \Lambda_1 \quad \text { and } \quad\left|\nabla \operatorname{Rm}_{(M, g)}\right| \leq \Lambda_2
$$
on the geodesic ball $B_x\left(\operatorname{inj}_{(M, g)}(x)\right)$ of center $x$ and radius $\operatorname{inj}_{(M, g)}(x)$. Then there exist positive constants $K=K\left(n, \Lambda_1, \Lambda_2\right)$ and $\delta=\delta\left(n, \Lambda_1, \Lambda_2\right)$, depending only on $n, \Lambda_1$, and $\Lambda_2$, such that the components $g_{i j}$ of $g$ in geodesic normal coordinates at $x$ satisfy: For any $i, j, k=1, \ldots, n$ and any $y \in B_0\left(\min \left(\delta, \operatorname{inj}_{(M . g)}(x)\right)\right)$,

\vskip 3pt
(i) $\frac{1}{4} \delta_{i j} \leq g_{i j}\left(\exp _x(y)\right) \leq 4 \delta_{i j}$ (as bilinear forms) and

\vskip 3pt
(ii) $\left|g_{i j}\left(\exp _x(y)\right)-\delta_{i j}\right| \leq K|y|^2$ and $\left|\partial_k g_{i j}\left(\exp _x(y)\right)\right| \leq K|y|$

\vskip 3pt
\noindent
where for $t>0, B_0(t)$ denotes the Euclidean ball of $\mathbb{R}^n$ with center 0 and radius $t$, and $|y|$ is the Euclidean distance from 0 to $y$. In addition, one has that
$$
\lim _{\Lambda \rightarrow 0} \delta\left(n, \Lambda_1, \Lambda_2\right)=+\infty \quad \text { and } \quad \lim _{\Lambda \rightarrow 0} K\left(n, \Lambda_1, \Lambda_2\right)=0
$$
where $\Lambda=\left(\Lambda_1, \Lambda_2\right)$.
\label{thm23}
\end{Theorem}

\begin{Proof}
Let $B$ be the Euclidean ball of $\mathbb{R}^n$ of radius inj $(M, g)(x)$ and centered at 0 . We still denote by $g$ the metric when transported on $B$ by $\exp _x$. Let $S$ be a segment in $B$ joining 0 to some point $P$ on $\partial B$. Then, $S$ is a geodesic for $g$. Let ( $\rho, \theta_1, \ldots, \theta_{n-1}$ ) be a polar coordinate system defined in a neighborhood of $S$, and let $Q \in S^{n-1}$, the unit sphere of $\mathbb{R}^n$, be such that $\overrightarrow{O Q}=\lambda \overrightarrow{O P}$ for some $\lambda>0$. We choose $\left(\theta_1, \ldots, \theta_{n-1}\right)$ such that it is a normal coordinate system at $Q$ for the standard metric of $S^{n-1}$. By the Gauss lemma,
$$
g=d \rho^2+\rho^2 h_{i j}(\rho, \theta) d \theta^i d \theta^j
$$
We let $g_{i j}=\rho^2 h_{i j}$. It is then easy to see that for any $i, j$,
\begin{equation}
R_{i \rho j \rho}=-\frac{1}{2} \partial_\rho \partial_\rho g_{i j}+\frac{1}{4} g^{\alpha \beta} \partial_\rho g_{\alpha i} \partial_\rho g_{\beta j}
\label{eq3}
\end{equation}
where obvious notation is used in this relation. Independently, there exist positive constants $\delta_1\left(n, \Lambda_1\right)$ and $C_1\left(n, \Lambda_1\right)$, satisfying
$$
\left\{\begin{array}{l}
\lim _{\Lambda_1 \rightarrow 0} \delta_1\left(n, \Lambda_1\right)=+\infty \\
\lim _{\Lambda_1 \rightarrow 0} C_1\left(n, \Lambda_1\right)=0
\end{array}\right.
$$
and such that for any $\rho<\delta_1\left(n, \Lambda_1\right)$, and any $i, j$,
\begin{equation}
\left|\partial_\rho h_{i j}\right| \leq C_1\left(n, \Lambda_1\right) \rho
\label{eq4}
\end{equation}
Since, when passing to the limit along $S, h_{i j}(0)=\delta_{i j}$, we get that for any $\rho<$ $\delta_1\left(n, \Lambda_1\right)$, and any $i, j$,
\begin{equation}
\left|h_{i j}-\delta_{i j}\right| \leq C_1\left(n, \Lambda_1\right) \rho^2
\label{eq5}
\end{equation}
on $S$. There exists then a positive constant
$$
\delta_2\left(n, \Lambda_1\right)=\min \left(\delta_1\left(n, \Lambda_1\right),\left(2 C_1\left(n, \Lambda_1\right)\right)^{-\frac{1}{2}}\right)
$$
satisfying
$$
\left\{\begin{array}{l}
\delta_2\left(n, \Lambda_1\right) \leq \delta_1\left(n, \Lambda_1\right) \\
\lim _{\Lambda_1 \rightarrow 0} \delta_2\left(n, \Lambda_1\right)=+\infty
\end{array}\right.
$$
and such that for any $\rho<\delta_2\left(n, \Lambda_1\right)$,
$$
\frac{1}{2} \delta_{i j} \leq\left(1-\frac{n}{2} C_1\left(n, \Lambda_1\right) \rho^2\right) \delta_{i j} \leq h_{i j} \leq\left(1+\frac{n}{2} C_1\left(n, \Lambda_1\right) \rho^2\right) \delta_{i j} \leq \frac{3}{2} \delta_{i j}
$$
as bilinear forms, and on $S$. Independently, it is easy to see that there exists a positive constant $A$ such that for any $i, j, k,\left|\partial_\rho \partial_k g_{i j}\right| \leq A \rho^3$ on $S$. Hence, for any $i, j, k,\left|\partial_k g_{i j}\right| \leq(A / 4) \rho^4$ on $S$. In the following, we show that $A$ can be chosen such that it depends only on $n, \Lambda_1$, and $\Lambda_2$. First, by the derivation of \autoref{eq3}, we get that for any $i, j, k$,
\begin{equation}
\partial_k R_{i \rho j \rho}=-\frac{1}{2} \partial_\rho \partial_\rho \partial_k g_{i j}+\frac{1}{4} \partial_k\left(g^{\alpha \beta} \partial_\rho g_{\alpha i} \partial_\rho g_{\beta j}\right)
\label{eq6}
\end{equation}
Independently, since $\left|\operatorname{Rm}_{(M . g)}\right| \leq \Lambda_1$ and $\left|\nabla \operatorname{Rm}_{(M . g)}\right| \leq \Lambda_2$, we get that there exist positive constants $\delta_3\left(n, \Lambda_1, \Lambda_2\right) \leq \delta_2\left(n, \Lambda_1\right), C_2\left(n, \Lambda_1\right)$, and $C_3\left(n, \Lambda_1\right)$, such that
$$
\left\{\begin{array}{l}
\lim _{\Lambda \rightarrow 0} \delta_3\left(n, \Lambda_1, \Lambda_2\right)=+\infty \\
\lim _{\Lambda_1 \rightarrow 0} C_2\left(n, \Lambda_1\right)=0 \\
\lim _{\Lambda_1 \rightarrow 0} C_3\left(n, \Lambda_1\right)=0
\end{array}\right.
$$
and such that for any $\rho<\delta_3\left(n, \Lambda_1, \Lambda_2\right)$, and any $i, j, k$,
$$
\left|\partial_k R_{i \rho j \rho}\right| \leq C_2\left(n, \Lambda_1\right) \rho^2+C_3\left(n, \Lambda_1\right) A \rho^4
$$
on $S$, where $\Lambda=\left(\Lambda_1, \Lambda_2\right)$. On the other hand, it is possible to prove that there exist positive constants $\delta_4\left(n, \Lambda_1, \Lambda_2\right) \leq \delta_3\left(n, \Lambda_1, \Lambda_2\right)$ and $C_4\left(n, \Lambda_1\right)$ such that
$$
\left\{\begin{array}{l}
\lim _{\Lambda \rightarrow 0} \delta_4\left(n, \Lambda_1, \Lambda_2\right)=+\infty \\
\lim _{\Lambda_1 \rightarrow 0} C_4\left(n, \Lambda_1\right)=0
\end{array}\right.
$$
and such that for any $\rho<\delta_4\left(n, \Lambda_1, \Lambda_2\right)$, and any $i, j, k$,
$$
\left|\partial_k\left(g^{\alpha \beta} \partial_\rho g_{\alpha i} \partial_\rho g_{\beta j}\right)\right| \leq 5 A \rho^2+C_4\left(n, \Lambda_1\right) A \rho^4
$$
on $S$. Now, combining these estimates with \autoref{eq6}, we get that there exist positive constants $C_5\left(n, \Lambda_1\right)$ and $C_6\left(n, \Lambda_1\right)$, such that
$$
\left\{\begin{array}{l}
\lim _{\Lambda_1 \rightarrow 0} C_5\left(n, \Lambda_1\right)=0 \\
\lim _{\Lambda_1 \rightarrow 0} C_6\left(n, \Lambda_1\right)=0
\end{array}\right.
$$
and such that for any $\rho<\delta_4\left(n, \Lambda_1, \Lambda_2\right)$, and any $i, j, k$,
$$
\left|\partial_\rho \partial_\rho \partial_k g_{i j}\right| \leq \frac{5}{2} A \rho^2+C_5\left(n, \Lambda_1\right) \rho^2+C_6\left(n, \Lambda_1\right) A \rho^4
$$
on $S$. Hence,
$$
\left|\partial_\rho \partial_k g_{i j}\right| \leq \frac{5}{6} A \rho^3+\frac{1}{3} C_5\left(n, \Lambda_1\right) \rho^3+\frac{1}{5} C_6\left(n, \Lambda_1\right) A \rho^5 \quad \text { on } S
$$
and there exist positive constants $\delta_5\left(n, \Lambda_1, \Lambda_2\right) \leq \delta_4\left(n, \Lambda_1, \Lambda_2\right)$, and $C_7\left(n, \Lambda_1\right)$, such that
$$
\left\{\begin{array}{l}
\lim _{\Lambda \rightarrow 0} \delta_5\left(n, \Lambda_1, \Lambda_2\right)=+\infty \\
\lim _{\Lambda_1 \rightarrow 0} C_7\left(n, \Lambda_1\right)=0
\end{array}\right.
$$
and such that for any $\rho<\delta_5\left(n, \Lambda_1, \Lambda_2\right)$, and any $i, j, k$,
$$
\left|\partial_\rho \partial_k g_{i j}\right| \leq \frac{6}{7} A \rho^3+C_7\left(n, \Lambda_1\right) \rho^3
$$
on $S$. Therefore, by induction, we get that for any $\rho<\delta_5\left(n, \Lambda_1, \Lambda_2\right)$, and any $i, j$, $k$,
$$
\left|\partial_\rho \partial_k g_{i j}\right| \leq 7 C_7\left(n, \Lambda_1\right) \rho^3
$$
on $S$. As a consequence, for any $\rho<\delta_5\left(n, \Lambda_1, \Lambda_2\right)$, and any $i, j, k$,
\begin{equation}
\left|\partial_k h_{i j}\right| \leq \frac{7}{4} C_7\left(n, \Lambda_1\right) \rho^2
\label{eq8}
\end{equation}
on $S$. Rewriting the inequalities \autoref{eq4}, \autoref{eq5}, and \autoref{eq8} in the Euclidean coordinate system of $\mathbb{R}^n$ ends the proof of the result.
\end{Proof}





%========================================
\newpage



\section{Compact Manifolds without Boundary}
We start in this section with the theory of Sobolev spaces on Riemannian manifolds. Section 3.1 recalls some elementary facts about Sobolev spaces for open subsets of the Euclidean space. 
Section 3.2 introduces Sobolev spaces on Riemannian manifolds. Here, in these sections, the compactness of the manifold is not assumed to hold. In Section 3.3, we start dealing with Sobolev embeddings and Sobolev inequalities. General results are proved there. Here again, the compactness of the manifold is not assumed to hold. 
Sections 3.4 deals with the validity of such embeddings and such inequalities for compact manifolds, while Section 3.5 deals with the compactness of these embeddings, still for compact manifolds. 

\subsection{Background Materials}

\subsubsection{Weak Derivatives}
Let $\Omega$ be an open subset of $\mathbb{R}^{\prime \prime}, \alpha$ a multi-index of length $|\alpha|$, and $u \in L_{\text {loc }}^1(\Omega)$ a locally integrable, real-valued function on $\Omega$. A function $v_\alpha \in L_{\text {loc }}^{1}(\Omega)$ is said to be the $\alpha^{\text {th }}$ weak (or distributional) derivative of $u$; we write $v_\alpha=D_\alpha u$, if, for any $\varphi \in \mathcal{D}(\Omega)$
$$
\int_{\Omega} u\left(D_\alpha \varphi\right) d x=\int_{\Omega} v_\alpha \varphi d x
$$
where $\mathcal{D}(\Omega)$ denotes the space of smooth functions with compact support in $\Omega$, and $d x$ is the Lebesgue's volume element. If such a $v_\alpha$ exists, it is unique up to sets of measure zero. When all the first weak derivatives of $u$ exist, namely, when $D_\alpha u$ exists for any $\alpha$ such that $|\alpha|=1, u$ is said to be weakly differentiable on $\Omega$. It is said to be $k$ times weakly differentiable if all its weak derivatives $D_\alpha u$ exist for $|\alpha| \leq k$

\subsubsection{Weakly Differentiable}
Let us now recall what we mean when speaking of an absolutely continuous function. Given $u: \mathbb{R} \rightarrow \mathbb{R}$, and $a<b$ real, we shall say that $u$ is absolutely continuous on $[a, b]$ if for all $\varepsilon>0$, there exists $\delta>0$ such that for any finite sequence
$$
a \leq x_1<y_1 \leq x_2<y_2 \leq \cdots \leq x_m<y_m \leq b,
$$
one has that
$$
\sum_{j=1}^m\left(y_j-x_j\right) \leq \delta \Rightarrow \sum_{j=1}^m\left|u\left(y_j\right)-u\left(x_j\right)\right| \leq \varepsilon
$$
As one can easily check, $u$ is absolutely continuous on $[a, b]$ if and only if there exists $v$ integrable on $[a, b]$ such that for any $a \leq x \leq b$,
$$
u(x)-u(a)=\int_a^x v(t) d t
$$
In particular, $u$ is differentiable almost everywhere and $u^{\prime}=v$. By extension, given $\Omega$ some open subset of $\mathbb{R}^n$, and $u: \Omega \rightarrow \mathbb{R}$ a real-valued function, we shall say that $u$ is absolutely continuous on all (resp. almost all) line segments in $\Omega$ parallel to the coordinate axes, if for all (resp. almost all) $x=\left(x_1, \ldots, x_n\right)$ in $\Omega$, all $i=1, \ldots, n$, and all $a<x_i<b$ such that
$$
\left\{\left(x_1, \ldots, x_{i-1}, x, x_{i+1}, \ldots, x_n\right), x \in[a, b]\right\} \subset \Omega
$$
the function
$$
x \rightarrow u\left(x_1, \ldots, x_{i-1}, x, x_{i+1}, \ldots, x_n\right)
$$
is absolutely continuous on $[a, b]$. According to what has been said above, if $u$ is absolutely continuous on almost all line segments in $\Omega$ parallel to the coordinate axes, then $u$ possesses partial derivatives of first order almost everywhere. We recall here the well-known following result, \autoref{thm31}. For its proof, one can look at the celebrated book of Schwartz \cite{Schwartz}.

\begin{Theorem}
    Let $\Omega$ be an open subset of $\mathbb{R}^n$ and $u \in L_{\text {loc }}^1(\Omega)$. Then $u$ is weakly differentiable on $\Omega$ if and only if (up to modifications on a set of measure zero):

    (i) $u$ is absolutely continuous on almost all line segments in $\Omega$ parallel to the coordinate 
    
    $\quad\;$axes, and

    (ii) the first partial derivatives of $u$ (which exist almost everywhere) belong to $L_{\text {loc }}^1(\Omega)$.
\label{thm31}
\end{Theorem}

\subsubsection{Sobolev Spaces on ${\mathbb{R}^n}$}
Let us now recall some material on what concerns the theory of Sobolev spaces in the Euclidean context. Let $\Omega$ be some open subset of $\mathbb{R}^n, k$ an integer, $p \geq 1$ real, and $u: \Omega \rightarrow \mathbb{R}$ a smooth, real-valued function. We let
$$
\|u\|_{k, p}=\sum_{0 \leq|\alpha| \leq k}\left(\int_{\Omega}\left|D_\alpha u\right|^p d x\right)^{1 / p}
$$
and we define then the Sobolev spaces
$$
\begin{array}{l}
H_k^p(\Omega)=\text { the completion of }\left\{u \in C^{\infty}(\Omega) /\|u\|_{k, p}<+\infty\right\} \text { for }\|\cdot\|_{k, p} \\
W_k^p(\Omega)=\left\{u \in L^p(\Omega) / \forall|\alpha| \leq k, D_\alpha u \text { exists and belongs to } L^p(\Omega)\right\}
\end{array}
$$
where $D_\alpha u$ denotes the $\alpha^{\text {ln }}$ weak partial derivative of $u$ as defined above. 
For many years, there has been considerable confusion in the mathematical literature about the relationship between these spaces. 
The following result of Meyers-Senin, \autoref{thm32}, dispelled such confusion, which can be found in the book of Adams \cite{A}. 
\begin{Theorem}
    For any $\Omega$, any $k$, and any $p \geq 1, H_k^p(\Omega)=W_k^p(\Omega)$
\label{thm32}
\end{Theorem}

In the end of this section, we recall basic properties of Sobolev spaces with respect to Lipschitz functions. This is the purpose of the following result.\cite{A} 

\begin{Theorem}
    ~~(i) If $\Omega$ is a bounded, open subset of $\mathbb{R}^n$, and if $u: \Omega \rightarrow \mathbb{R}$ is Lipschitz, then $u \in H_1^p(\Omega)$ for all $p \geq 1$.
    ~~~~(ii) Let $\Omega$ be an open subset of $\mathbb{R}^n, h: \mathbb{R} \rightarrow \mathbb{R}$ a Lipschitz function, and $u \in H_1^p(\Omega)$ for some $p \geq 1$. If $h \circ u \in L^p(\Omega)$, then $h \circ u \in H_1^p(\Omega)$ and
    $$
    D_i(h \circ u)(x)=h^{\prime}(u(x)) D_i u(x)
    $$
    for all $i=1, \ldots, n$, and almost all $x \in \Omega$.
\label{thm33}    
\end{Theorem}

\subsection{Sobolev Spaces on Riemannian Manifolds}

\subsubsection{Definition of Sobolev Spaces on (M,g)}
Let $(M, g)$ be a smooth Riemannian manifold. For $k$ integer, and $u: M \rightarrow \mathbb{R}$ smooth, we denote by $\nabla^k u$ the $k^{\text {th }}$ covariant derivative of $u$, and $\left|\nabla^k u\right|$ the norm of $\nabla^k u$ defined in a local chart by
$$
\left|\nabla^k u\right|=g^{i_1 j_1} \ldots g^{i_k j_k}\left(\nabla^k u\right)_{i_1 \ldots i_k}\left(\nabla^k u\right)_{j_1 \ldots j_k}
$$
Recall that $(\nabla u)_i=\partial_i u$, while
$$
\left(\nabla^2 u\right)_{i j}=\partial_{i j} u-\Gamma_{i j}^k \partial_k u
$$
Given $k$ an integer, and $p \geq 1$ real, set
$$
\mathcal{C}_k^p(M)=\left\{u \in C^{\infty}(M)\,,\, \forall j=0, \ldots, k, \int_M\left|\nabla^j u\right|^p d v(g)<+\infty\right\}
$$
When $M$ is compact, one clearly has that $\mathcal{C}_k^p(M)=C^{\infty}(M)$ for any $k$ and any $p \geq 1$. For $u \in \mathcal{C}_k^p(M)$, set also
$$
\|u\|_{H_k^p}=\sum_{j=0}^k\left(\int_M\left|\nabla^j u\right|^p d v(g)\right)^{1 / p}
$$
We define the Sobolev space $H_k^p(M)$ as follows:

\begin{Definition}
    Given $(M, g)$ a smooth Riemannian manifold, $k$ an integer, and $p \geq 1$ real, the Sobolev space $H_k^p(M)$ is the completion of $\mathcal{C}_k^p(M)$ with respect to $\|\cdot\|_{H_k^p}$.
\label{def31}
\end{Definition}

Note here that one can look at these spaces as subspaces of $L^p(M)$. Let $\|\cdot\|_p$ be the norm of $L^p(M)$ defined by
$$
\|u\|_p=\left(\int_M|u|^p d v(g)\right)^{1 / p}
$$
As one can easily check:

1. any Cauchy sequence in $\left(\mathcal{C}_k^p(M),\|\cdot\|_{H_k^p}\right)$ is a Cauchy sequence in the Lebesgue 

$\quad\;$space $\left(L^p(M),\|\cdot\|_p\right)$ and

2. any Cauchy sequence in $\left(\mathcal{C}_k^p(M),\|\cdot\|_{H_k^p}\right)$ that converges to 0 in the Lebesgue 

$\quad\;$space $\left(L^p(M),\|\cdot\|_p\right)$, also converges to 0 in $\left(\mathcal{C}_k^p(M),\|\cdot\|_{H_k^p}\right)$.

\vskip 3pt
\noindent
As a consequence, one can look at $H_k^p(M)$ as the subspace of $L^p(M)$ made of functions $u \in L^p(M)$ which are limits in $\left(L^p(M),\|\cdot\|_p\right)$ of a Cauchy sequence $\left(u_m\right)$ in $\left(\mathcal{C}_k^p(M),\|\cdot\|_{H_k^p}\right)$, and define $\|u\|_{H_k^p}$ as above, where $\left|\nabla^j u\right|, 0 \leq j \leq k$, is now the limit in $\left(L^p(M),\|\cdot\|_p\right)$ of the Cauchy sequence $\left(\left|\nabla^j u_m\right|\right)$.

\subsubsection{Properties of Sobolev Spaces on (M,g)}
Coming back to \autoref{def31} , one can replace $\|\cdot\|_{H_k^p}$ by any other equivalent norm. In particular, the following holds.

\begin{Proposition}
    For any $k$ integer, $H_k^2(M)$ is a Hilbert space when equipped with the equivalent norm
$$
\|u\|=\sqrt{\sum_{j=0}^k \int_M\left|\nabla^j u\right|^2 d v(g)}
$$
The scalar product $\langle\cdot, \cdot)$ associated to $\|\cdot\|$ is defined by
$$
\langle u, v\rangle=\sum_{j=0}^k \int_M\left\langle\nabla^j u, \nabla^j v\right\rangle d v(g)
$$
where, in such an expression, $\langle\cdot, \cdot\rangle$ is the scalar product on covariant tensor fields associated to $\mathrm{g}$.
\label{prp31}
\end{Proposition}

In the same order of ideas, let $M$ be a compact manifold endowed with two Riemannian metrics $g$ and $\tilde{g}$. As one can easily check, there exists $C>1$ such that
$$
\frac{1}{C} g \leq \tilde{g} \leq C g
$$
on $M$, where such inequalities have to be understood in the sense of bilinear forms. This leads to the following:

\begin{Proposition}
    If $M$ is compact, $H_k^p(M)$ does not depend on the metric.
\label{prp32}
\end{Proposition}

Such a proposition is of course not anymore true if the manifold is not assumed to be compact. Let, for instance, $g$ and $\tilde{g}$ be two Riemannian metrics on $\mathbb{R}^n,\left(\mathbb{R}^n, g\right)$ being of finite volume, $\left(\mathbb{R}^n, \tilde{g}\right)$ being of infinite volume. As an example, one can take
$$
g=\frac{4}{\left(1+|x|^2\right)^2} e
$$
(the standard metric of $S^n$ after stereographic projection), and $\tilde{g}=e$, where $e$ is the Euclidean metric of $\mathbb{R}^n$. Then the constant function $u=1$ belongs to the Sobolev spaces associated to $g$, while it does not belong to the Sobolev spaces associated to $\tilde{g}$. This proves the claim. Independently, noting that $\left(L^p(M),\|\cdot\|_p\right)$ is reflexive if $p>1$, one gets the following:

\begin{Proposition}
    If $p>1, H_k^p(M)$ is reflexive.
    \label{prp33}
\end{Proposition}

Still when dealing with general results, let us now prove the following one. Given $(M, g)$ a Riemannian manifold, $u: M \rightarrow \mathbb{R}$ is said to be Lipschitz on $M$ if there exists $\Lambda>0$ such that for any $x, y \in M$,
$$
|u(y)-u(x)| \leq \Lambda d_g(x, y)
$$
where $d_g$ is the distance associated to $g$.

\begin{Proposition}
    Let $(M, g)$ be a smooth Riemannian manifold, and $u: M \rightarrow \mathbb{R}$ a Lipschitz function on $M$ with compact support. Then $u \in H_1^p(M)$ for any $p \geq 1$. In particular, if $M$ is compact, any Lipschitz function on $M$ belongs to the Sobolev spaces $H_1^p(M), p \geq 1$.
\label{prp34}
\end{Proposition}

\begin{Proof}
    Let $u: M \rightarrow \mathbb{R}$ be a Lipschitz function on $M$ such that $u=0$ outside a compact subset $K$ of $M$. Let also $\left(\Omega_k, \varphi_k\right)_{k=1 \ldots . N}$ be a family of charts such that $K \subset \bigcup_{k=1}^N \Omega_k$ and such that for any $k=1, \ldots, N$,
$$
\varphi_k\left(\Omega_k\right)=B_0(1) \text { and } \frac{1}{C} \delta_{i j} \leq g_{i j}^k \leq C \delta_{i j}
$$
as bilinear forms, where $C>1$ is given, $B_0(1)$ denotes the Euclidean ball of $\mathbb{R}^n$ of center 0 and radius 1 , and where the $g_{i j}^k$ 's stand for the components of $g$ in $\left(\Omega_k, \varphi_k\right)$. Consider $\left(\eta_k\right)_{k=1, \ldots . N+1}$ a smooth partition of unity subordinate to the covering
$$
\left(\Omega_k\right)_{k=1 \ldots . . N} \cup(M \backslash K)
$$
For $k \in\{1, \ldots, N\}$, it is clear that the function
$$
u_k=\left(\eta_k u\right) \circ \varphi_k^{-1}
$$
is Lipschitz on $B_0(1)$ for the Euclidean metric. According to \autoref{thm33} one then gets that $u_k \in H_1^p\left(B_0(1)\right)$ for any $p \geq 1$. Clearly, this implies that $\eta_k u \in H_1^p(M)$. Since
$$
u=\sum_{k=1}^N \eta_k u
$$
this ends the proof of the proposition.
\end{Proof}

On what concerns \autoref{prp34} , note that given $(M, g)$ a smooth Riemannian manifold, a differentiable function $u: M \rightarrow \mathbb{R}$ for which $|\nabla u|$ is bounded, is Lipschitz on $M$. In order to fix ideas, suppose that $(M, g)$ is complete. Let $x$ and $y$ be two points on $M$, and let $\gamma:[0,1] \hat{\mathrm{E}} \rightarrow M$ be the minimizing geodesic from $x$ to $y$. One has that there exists $t \in(0,1)$ such that
$$
\begin{aligned}
|u(y)-u(x)| & =|u(\gamma(1))-u(\gamma(0))| \\
& =\left|(u \circ \gamma)^{\prime}(t)\right|
\end{aligned}
$$
Hence, if $d_g$ denotes the distance associated to $g$,
$$
\begin{aligned}
|u(y)-u(x)| & =\left|(u \circ \gamma)^{\prime}(t)\right| \\
& =\left|d u(\gamma(t)) \cdot\left(\frac{d \gamma}{d t}\right)_t\right| \\
& \leq|\nabla u(\gamma(t))| \times\left|\left(\frac{d \gamma}{d t}\right)_t\right| \\
& \leq\left(\sup _M|\nabla u|\right) d_g(x, y)
\end{aligned}
$$
This proves the claim. Independently, one has the following result:

\begin{Proposition}
    Let $(M, g)$ be a smooth complete Riemannian manifold, $h$ : $\mathbb{R} \rightarrow \mathbb{R}$ a Lipschitz function, and $u \in H_1^p(M), p \geq 1$. If $h \circ u \in L^p(M)$, then $h \circ u \in H_1^p(M)$ and
$$
|(\nabla(h \circ u))(x)|=\left|h^{\prime}(u(x))\right| \cdot|(\nabla u)(x)|
$$
for almost all $x$ in $M$. In particular, for any $u \in H_1^p(M),|u| \in H_1^p(M)$, and $|\nabla| u||=|\nabla u|$ almost everywhere.
\label{prp35}
\end{Proposition}

\begin{Proof}
 Let $x \in M$ be given. Let also $v=h \circ u$. With similar arguments to those used in the proof of \autoref{prp34}, one can easily get that $v \in H_1^p\left(B_x(r)\right)$ for all $r>0$. Moreover, coming back to \autoref{thm33}, one sees that
$$
|\nabla v(y)|=\left|h^{\prime}(u(y))\right| \cdot|\nabla u(y)|
$$
for almost all $y$ in $M$. In particular, and by assumption, $v$ and $|\nabla v|$ both belong to $L^p(M)$. One must still check that $v \in H_1^p(M)$. Let $f: \mathbb{R} \rightarrow \mathbb{R}$ be the function defined by
$$
f(t)=1 \quad \text { if } t \leq 0, \quad f(t)=1-t \quad \text { if } 0 \leq t \leq 1, \quad f(t)=0 \quad \text { if } t \geq 1
$$
For $j$ an integer, and if $d_g$ denotes the distance associated to $g$, we let $f_j$ be the function defined by
$$
f_j(y)=f\left(d_g(x, y)-j\right)
$$
Clearly, $f_j$ is Lipschitz with compact support the closure of $B_x(j+1)$. Moreover, since the cut-locus of $x$ is negligible, one has that $f_j$ is differentiable almost everywhere, with the additional property that $\left|\nabla f_j\right| \leq 1$. We claim here that $v_j=f_j v$ belongs to $H_1^p(M)$. Indeed, given $r>j+1$, let $\left(v_m\right)$ be a sequence in $C^{\infty}\left(B_x(r)\right)$ that converges to $v$ in $H_1^p\left(B_x(r)\right)$. Clearly, $f_j v_m$ is Lipschitz with compact support in $M$, so that by \autoref{prp34}, $f_j v_m \in H_1^p(M)$. Moreover, since $f_j$ and $\left|\nabla f_j\right|$ are bounded, and since
$$
\nabla v_j=\left(\nabla f_j\right) v+f_j(\nabla v)
$$
almost everywhere, one easily gets that $\left(f_j v_m\right)$ converges, in $H_1^p(M)$ and as $m$ goes to $+\infty$, to $v_j$. Hence, $v_j \in H_1^p(M)$ and this proves the claim. Here, one easily checks that for any $j$
$$
\left(\int_M\left|v_j-v\right|^p d v(g)\right)^{1 / p} \leq\left(\int_{M \backslash B_x(j)}|v|^p d v(g)\right)^{1 / p}
$$
and
$$
\left(\int_M\left|\nabla\left(v_j-v\right)\right|^p d v(g)\right)^{1 / p} \leq\left(\int_{M \backslash B_x(j)}|\nabla v|^p d v(g)\right)^{1 / p}+\left(\int_{M \backslash B_x(j)}|v|^p d v(g)\right)^{1 / p}
$$
As an easy consequence of such inequalities, one gets that $v \in H_1^p(M)$, and that $\left(v_j\right)$ converges to $v$ in $H_1^p(M)$ as $j$ goes to $+\infty$. This proves the proposition.
\end{Proof}

\vskip 3pt
In the end of this section, we now prove the following. Similar density questions for higher-order Sobolev spaces will be treated in the sequel.

\begin{Theorem}
    Given $(M, g)$ a smooth, complete Riemannian manifold, the set $D(M)$ of smooth functions with compact support in $M$ is dense in $H_1^p(M)$ for any $p \geq 1$.
\label{thm34}
\end{Theorem}

\begin{Proof}
Let $f: \mathbb{R} \rightarrow \mathbb{R}$ be detined by
$$
f(t)=1 \quad \text { if } t \leq 0, \quad f(t)=1-t \quad \text { if } 0 \leq t \leq 1, \quad f(t)=0 \quad \text { if } t \geq 1
$$
and let $u \in \mathcal{C}_1^p(M)$ where $p \geq 1$ is some given real number. Let $x$ be some point of $M$ and set
$$
u_j(y)=u(y) f\left(d_g(x, y)-j\right)
$$
where $d_g$ is the distance associated to $g, j$ is an integer, and $y \in M$. By \autoref{prp34}, $u_j \in H_1^p(M)$ for any $j$, and since $u_j=0$ outside a compact subset of $M$, one easily gets that for any $j, u_j$ is the limit in $H_1^p(M)$ of some sequence of functions in $\mathcal{D}(M)$. One just has to note here that if $\left(u_m\right) \in \mathcal{C}_1^p(M)$ converges to $u_j$ in $H_1^p(M)$, and if $\alpha \in \mathcal{D}(M)$, then $\left(\alpha u_m\right)$ converges to $\alpha u_j$ in $H_1^p(M)$. Then, one can choose $\alpha \in D(M)$ such that $\alpha=1$ where $u_j \neq 0$. Independently, one clearly has that for any $j$,
$$
\left(\int_M\left|u_j-u\right|^p d v(g)\right)^{1 / p} \leq\left(\int_{M \backslash B_x(j)}|u|^p d v(g)\right)^{1 / p}
$$
and
$$
\qquad\left(\int_M\left|\nabla\left(u_j-u\right)\right|^p d v(g)\right)^{1 / p}\leq\left(\int_{M \backslash B_x(j)}|\nabla u|^p d v(g)\right)^{1 / p}+\left(\int_{M \backslash B_x(j)}|u|^p d v(g)\right)^{1 / p}
$$
Hence, $\left(u_j\right)$ converges to $u$ in $H_1^p(M)$ as $j$ goes to $+\infty$. According to what has been said above, one then gets that $u$ is the limit in $H_1^p(M)$ of some sequence in $D(M)$. This ends the proof of the theorem.
\end{Proof}

\subsection{Sobolev Embeddings: General Results}

\subsubsection{General Results of Sobolev Embeddings}
As a starting point, let us fix a convention that will be used in the sequel. Given $\left(E,\|\cdot\|_E\right)$ and $\left(F,\|\cdot\|_F\right)$ two normed vector spaces with the property that $E$ is a subspace of $F$, we write that $E \subset F$ and say that the inclusion is continuous if there exists $C>0$ such that for any $x \in E$,
$$
\|x\|_F \leq C\|x\|_E
$$
Now, let $(M, g)$ be a smooth Riemannian $n$-manifold. By Sobolev embeddings, at least in their first part, we refer to the following: 

\vskip 5pt
\noindent
\textbf{Given $\bm{p, q}$ two real numbers with $\bm{1 \leq q<p}$, and given $\bm{k, m}$ two integers with $\bm{0 \leq m<k}$, if $\bm{1 / p=1 / q-(k-m) / n}$, then $\bm{H_k^q(M) \subset H_m^p(M)}$.}

\vskip 5pt
\noindent
As mentioned above, the notation $H_k^q(M) \subset H_m^p(M)$ includes the continuity of the embedding, namely, the existence of a positive constant $C$ such that for any $u \in H_k^q(M),\|u\|_{H_m^p} \leq C\|u\|_{H_k^q}$. The validity of such embeddings, which may or may not hold in the general case of a Riemannian manifold, is often referred to as the Sobolev embedding theorem. We will see later on in this section that the Sobolev embedding theorem does hold for compact manifolds, while we will see in the next section that the situation is more intricate on what concems complete noncompact manifolds. 

\subsubsection{General Embeddings when k=1 and m=0}
Note here that for $k=1$, and hence $m=0$, the Sobolev embeddings reduce to the assertion that for any $q \in[1, n), H_1^q(M) \subset L^p(M)$ with $p=n q /(n-q)$. Note also that the exponents in the Sobolev embeddings are optimal. Think, for instance, of the Euclidean space $\left(\mathbb{R}^n, e\right)$; let $k=1$, and let $\varphi \in D\left(\mathbb{R}^n\right), \varphi \not\equiv 0$, be smooth with compact support in $\mathbb{R}^n$. For $\lambda \geq 1$, set $\varphi_\lambda(x)=\varphi(\lambda x)$. Then, as one can easily check,

$$
\begin{aligned}
\left\|\varphi_\lambda\right\|_p & =\lambda^{-n / p}\|\varphi\|_p \\
\left\|\varphi_\lambda\right\|_{H_1^q} & \leq \lambda^{1-\frac{n}{q}}\|\varphi\|_{H_1^q}
\end{aligned}
$$
By passing to the limit $\lambda \rightarrow+\infty$, one sees that the existence of $C>0$ such that for any $u \in D\left(\mathbb{R}^n\right)$,
$$
\|u\|_p \leq C\|u\|_{H_1^q}
$$
leads to the inequality $1 / p \geq 1 / q-1 / n$. This proves the claim. For convenience, all the manifolds in what follows will be assumed to be at least complete. We start here by proving the following result:

\begin{Lemma}
    Let $(M, g)$ be a smooth, complete Riemannian n-manifold. Suppose that $H_1^{1}(M) \subset L^{n /(n-1)}(M)$. Then for any real numbers $1 \leq q<p$ and any integers $0 \leq m<k$ such that $1 / p=1 / q-(k-m) / n, H_k^q(M) \subset H_m^p(M)$.
\label{lem31}
\end{Lemma}

\begin{Proof}
    We prove that if $H_1^1(M) \subset L^{n /(n-1)}(M)$, then for any $q \in[1, n)$, $H_1^q(M) \subset L^p(M)$ where $1 / p=1 / q-1 / n$. We refer to \cite{Aubin}, Proposition 2.11, for the proof that the other embeddings are also valid. Let $C>0$ be such that for any $u \in H_1^{1}(M)$
$$
\left(\int_M|u|^{n /(n-1)} d v(g)\right)^{(n-1) / n} \leq C \int_M(|\nabla u|+|u|) d v(g)
$$
Let also $q \in(1, n), p=n q /(n-q)$, and $u \in \mathcal{D}(M)$. Set $\varphi=|u|^{p(n-1) / n}$. By Hölder's inequalities we get that

$$
\begin{aligned}
    \left(\int_M|u|^p d v(g)\right)^{(n-1) / n} &=\left(\int_M|\varphi|^{n /(n-1)} d v(g)\right)^{(n-1) / n} \leq C \int_M(|\nabla \varphi|+|\varphi|) d v(g)\\
    &= \frac{C p(n-1)}{n} \int_M|u|^{p^{\prime}}|\nabla u| d v(g)+C \int_M|u|^{p(n-1) / n} d v(g) \\
    &\leq  \frac{C p(n-1)}{n}\left(\int_M|u|^{p^{\prime} q^{\prime}} d v(g)\right)^{1 / q^{\prime}}\left(\int_M|\nabla u|^q d v(g)\right)^{1 / q} \\
    & +C\left(\int_M|u|^{p^{\prime} q^{\prime}} d v(g)\right)^{1 / q^{\prime}}\left(\int_M|u|^q d v(g)\right)^{1 / q}
\end{aligned}
$$
where $1 / q+1 / q^{\prime}=1$ and $p^{\prime}=p(n-1) / n-1$. But $p^{\prime} q^{\prime}=p$ since $1 / p=$ $1 / q-1 / n$. As a consequence, for any $u \in \mathcal{D}(M)$,
$$
\left(\int_M|u|^p d v(g)\right)^{1 / p} \leq \frac{C p(n-1)}{n}\left(\left(\int_M|\nabla u|^q d v(g)\right)^{1 / q}+\left(\int_M|u|^q d v(g)\right)^{1 / q}\right)
$$
By \autoref{thm34} this ends the proof of the lemma.
\end{Proof}

\vskip 3pt
Note here that with the same arguments as those developed in the proof of \autoref{lem31}, one gets a hierarchy for Sobolev embeddings. More precisely, one can prove that if for some $q \in[1, n), H_1^q(M) \subset L^{n q /(n-q)}(M)$, then $H_1^{q^{\prime}}(M) \subset$ $L^{n q^{\prime} /\left(n-q^{\prime}\right)}(M)$ for any $q^{\prime} \in[q, n)$. Indeed, let $C>0$ be such that for any $u \in$ $H_1^q(M)$
$$
\left(\int_M|u|^p d v(g)\right)^{1 / p} \leq C\left(\left(\int_M|\nabla u|^q d v(g)\right)^{1 / q}+\left(\int_M|u|^q d v(g)\right)^{1 / q}\right)
$$
where $1 / p=1 / q-1 / n$. Given $q^{\prime} \in(q, n)$, and $u \in \mathcal{D}(M)$, let also $\varphi=$ $|u|^{p^{\prime}(n-q) / n q}$ where $p^{\prime}$ is such that $1 / p^{\prime}=1 / q^{\prime}-1 / n$. Then, as in the proof of \autoref{lm31}, one gets with Hölder's inequalities that
$$
\begin{aligned}
    &\left(\int_M|u|^{p^{\prime}} d v(g)\right)^{1 / p} =\left(\int_M|\varphi|^p d v(g)\right)^{1 / p} \\
    &\leq C\left(\left(\int_M|\nabla \varphi|^q d v(g)\right)^{1 / q}+\left(\int_M|\varphi|^q d v(g)\right)^{1 / q}\right) \\
    &=C(s+1)\left(\int_M|u|^{s q}|\nabla u|^q d v(g)\right)^{1 / q}+C\left(\int_M|u|^{p^{\prime}(n-q) / n} d v(g)\right)^{1 / q} \\
    & \leq C(s+1)\left(\int_M|u|^{q s q^{\prime} /\left(q^{\prime}-q\right)} d v(g)\right)^{\left(q^{\prime}-q\right) / q q^{\prime}}\left(\int_M|\nabla u|^{q^{\prime}} d v(g)\right)^{1 / q^{\prime}} \\
    &+C\left(\int_M|u|^{q s q^{\prime} /\left(q^{\prime}-q\right)} d v(g)\right)^{\left(q^{\prime}-q\right) / q q^{\prime}}\left(\int_M|u|^{q^{\prime}} d v(g)\right)^{1 / q^{\prime}}\\
\end{aligned}
$$
where $s=\frac{p^{\prime}(n-q)}{n q}-1$. But
$$
\frac{1}{p}-\frac{q^{\prime}-q}{q q^{\prime}}=\frac{1}{p^{\prime}} \quad \text { and } \quad \frac{q s q^{\prime}}{q^{\prime}-q}=p^{\prime}
$$
Hence, for any $u \in \mathcal{D}(M)$,
$$
\left(\int_M|u|^{p^{\prime}} d v(g)\right)^{1 / p^{\prime}} 
\quad \leq \frac{p^{\prime}(n-q) C}{n q}\left(\left(\int_M|\nabla u|^{q^{\prime}} d v(g)\right)^{1 / q^{\prime}}+\left(\int_M|u|^{q^{\prime}} d v(g)\right)^{1 / q^{\prime}}\right)
$$
By \autoref{thm34}, this proves the above claim.

\subsubsection{Volume of Balls when k=1 and m=0}
Let us now discuss an important consequence of the validity of Sobolev embeddings. As a starting point, suppose that
$$
H_1^{1}(M) \subset L^{n /(n-1)}(M)
$$
Let $C>0$ be such that for any $u \in H_1^{1}(M)$
$$
\left(\int_M|u|^{n /(n-1)} d v(g)\right)^{(n-1) / n} \leq C \int_M(|\nabla u|+|u|) d v(g)
$$
From such an inequality, one gets that for any $x \in M$, and almost all $r>0$,
$$
\operatorname{Vol}_g\left(B_x(r)\right)^{(n-1) / n} \leq C \frac{d}{d r} \operatorname{Vol}_g\left(B_x(r)\right)+C \operatorname{Vol}_g\left(B_x(r)\right)
$$
where $B_x(r)$ is the ball of center $x$ and radius $r$ in $M$, and $\operatorname{Vol}_g\left(B_x(r)\right)$ stands for its volume with respect to $g$. From now on, let $R>0$ be given. Either $\operatorname{Vol}_g\left(B_x(R)\right) \geq$ $(1 / 2 C)^n$, or $\operatorname{Vol}_g\left(B_x(R)\right) \leq(1 / 2 C)^n$. In the last case, one gets that for almost all $r \in(0, R]$
$$
\frac{1}{2 C} \operatorname{Vol}_g\left(B_x(r)\right)^{1-1 / n} \leq \frac{d}{d r} \operatorname{Vol}_g\left(B_x(r)\right)
$$
Integrating this last inequality one then gets that for any $x \in M$ and any $R>0$,
$$
\operatorname{Vol}_g\left(B_x(R)\right) \geq \min \left(\left(\frac{1}{2 C}\right)^n,\left(\frac{R}{2 n C}\right)^n\right)
$$
In other words, the fact that $H_1^{1}(M) \subset L^{n /(n-1)}(M)$ implies that there is a lower bound for the volume of balls with respect to their center. The following important lemma, due to Carron\cite{Carron}, extends this result to the other embeddings $H_1^q(M) \subset$ $L^p(M), q \in[1, n), 1 / p=1 / q-1 / n$

\begin{Lemma}
    Let $(M, g)$ be a smooth, complete Riemannian n-manifold. Suppose that $H_1^q(M) \subset L^p(M)$ for some $q \in[1, n)$, where $1 / p=1 / q-1 / n$. Then for any $r>0$ there exists a positive constant $v=v(M, q, r)$ such that for any $x \in M$, $\operatorname{Vol}_g\left(B_x(r)\right) \geq v$.
\label{lem32}
\end{Lemma}

\begin{Proof}
Let $q \in[1, n)$, and suppose that $H_1^q(M) \subset L^p(M)$ where $1 / p=$ $1 / q-1 / n$. One then gets the existence of $A>0$ such that for any $u \in H_1^q(M)$,
$$
\left(\int_M|u|^p d v(g)\right)^{1 / p} \leq A\left(\left(\int_M|\nabla u|^q d v(g)\right)^{1 / q}+\left(\int_M|u|^q d v(g)\right)^{1 / q}\right)
$$
Let $r>0$, let $x$ be some point of $M$, and let $v \in H_1^q(M)$ be such that $v=0$ on $M \backslash B_x(r)$. By Hölder's inequality,
$$
\left(\int_M|v|^q d v(g)\right)^{1 / q} \leq \operatorname{Vol}_g\left(B_x(r)\right)^{1 / n}\left(\int_M|v|^p d v(g)\right)^{1 / p}
$$
Hence,
$$
\frac{1}{\operatorname{Vol}_g\left(B_x(r)\right)^{1 / n}}-A \leq A \frac{\left(\int_M|\nabla v|^q d v(g)\right)^{1 / q}}{\left(\int_M|v|^q d v(g)\right)^{1 / q}}
$$
Fix $x \in M$ and let $R>0$ be given. Then, either $\operatorname{Vol}_g\left(B_x(R)\right)>(1 / 2 A)^n$ or $\operatorname{Vol}_g\left(B_x(R)\right) \leq(1 / 2 A)^n$, in which case we get that for any $r \in(0, R]$,
$$
\frac{1}{\operatorname{Vol}_g\left(B_x(r)\right)^{1 / n}}-A \geq \frac{1}{2 \mathrm{Vol}_g\left(B_x(r)\right)^{1 / n}}
$$
Suppose that $\operatorname{Vol}_g\left(B_x(R)\right) \leq(1 / 2 A)^n$. We then have that for any $r \in(0, R]$ and any $v \in H_1^q(M)$ such that $v=0$ on $M \backslash B_x(r)$,
$$
\frac{1}{(2 A)^q} \operatorname{Vol}_g\left(B_x(r)\right)^{-q / n} \leq \frac{\int_M|\nabla v|^q d v(g)}{\int_M|v|^q d v(g)}
$$
From now on, let
$$
\begin{array}{ll}
v(y)=r-d_g(x, y) & \text { if } d_g(x, y) \leq r \\
v(y)=0 & \text { if } d_g(x, y) \geq r
\end{array}
$$
where $d_g$ is the distance associated to $g$. Clearly, $v$ is Lipschitz and $v=0$ on $M \backslash B_x(r)$. Hence (see \autoref{prp34}), $v$ belongs to $H_1^q(M)$. As a consequence,
$$
\frac{1}{(2 A)^q} \operatorname{Vol}_g\left(B_x(r)\right)^{-q / n} \leq \frac{\operatorname{Vol}_g\left(B_x(r)\right)}{\int_{B_{{x}}(r / 2)} v^q d v(g)} \leq \frac{2^q \operatorname{Vol}_g\left(B_x(r)\right)}{r^q \operatorname{Vol}_g\left(B_x(r / 2)\right)}
$$
and we get that for any $r \leq R$,
$$
\operatorname{Vol}_g\left(B_x(r)\right) \geq\left(\frac{r}{4 A}\right)^{n q /(n+q)} \operatorname{Vol}_g\left(B_x(r / 2)\right)^{n /(n+q)}
$$
By induction we then get that for any $m \in \mathbb{N} \backslash\{0\}$,
\begin{equation}
\operatorname{Vol}_g\left(B_x(R)\right) \geq\left(\frac{R}{2 A}\right)^{q \alpha(m)}\left(\frac{1}{2}\right)^{q \beta(m)} \operatorname{Vol}_g\left(B_{x}\left(R / 2^m\right)\right)^{\gamma(m)}
\label{eq7}
\end{equation}
where
$$
\alpha(m)=\sum_{i=1}^m\left(\frac{n}{n+q}\right)^i, \beta(m)=\sum_{i=1}^m i\left(\frac{n}{n+q}\right)^i,  \gamma(m)=\left(\frac{n}{n+q}\right)^m
$$
However,
$$
\operatorname{Vol}_g\left(B_x(r)\right)=b_n r^n(1+o(r))
$$
where $b_n$ is the volume of the Euclidean ball of radius one. Hence,
$$
\lim _{m \rightarrow \infty} \operatorname{Vol}_g\left(B_x\left(R / 2^{m}\right)\right)^{\gamma(m)}=1
$$
In addition, we have that
$$
\sum_{i=1}^{\infty}\left(\frac{n}{n+q}\right)^i=\frac{n}{q} \text { and } \sum_{i=1}^{\infty} i\left(\frac{n}{n+q}\right)^i=\frac{n(n+q)}{q^2}
$$
As a consequence, letting $m \rightarrow \infty$ in \autoref{eq7} we get that
$$
\operatorname{Vol}_g\left(B_x(R)\right) \geq\left(\frac{1}{2^{(n+2 q) / q} A}\right)^n R^n
$$
Finally, for any $x \in M$ and any $R>0$,
$$
\operatorname{Vol}_g\left(B_x(R)\right) \geq \min \left(1 / 2 A, R / 2^{(n+2 q) / q} A\right)^n
$$
and this ends the proof of the lemma.
\end{Proof}

\vskip 3pt
Note here that one gets from the above proof the exact dependence of $v$. Namely, $v$ depends on $n, q, r$, and the constant $C$ of the embedding of $H_1^q(M)$ in $L^p(M)$. Independently, we used in the above proof the fact that $\operatorname{Vol}_g\left(B_x(r)\right)=$ $b_n r^n(1+o(r))$ where $b_n$ is the volume of the Euclidean ball of radius one. More precisely (see, for instance, Gallot-Hulin-Lafontaine \cite{Gallot}), one has that
$$
\operatorname{Vol}_g\left(B_x(r)\right)=b_n r^n\left(1-\frac{1}{6(n+2)} \operatorname{Scal}_{(M, g)}(x) r^2+o\left(r^2\right)\right)
$$
where $\operatorname{Scal}_{(M, g)}$ stands for the scalar curvature of $(M, g)$.

\subsection{Embeddings for Compact Manifolds without Boundary}

\subsubsection{When 1/q-(k-m)/n>0}
We prove in this section that Sobolev embeddings in their first part do hold for compact manifolds. This is the subject of the following theorem:
\begin{Theorem}
    Let $(M, g)$ be a smooth, compact Riemannian n-manifold. The Sobolev embeddings in their first part do hold on $(M, g)$ in the sense that for any real numbers $1 \leq q<p$ and any integers $0 \leq m<k$, \textbf{if $\bm{1 / p=1 / q-(k-m) / n}$, then $\bm{H_k^q(M) \subset H_m^p(M)}$.} In particular, for any $q \in[1, n)$ real, $H_1^q(M) \subset L^p(M)$ where $1 / p=1 / q-1 / n$.
\label{thm35}
\end{Theorem}

\begin{Proof}
    By \autoref{lem31} we have only to prove that the embedding $H_1^1(M) \subset$ $L^{n /(n-1)}(M)$ is valid. Since $M$ is compact, $M$ can be covered by a finite number of charts
$$
\left(\Omega_m, \varphi_m\right)_{m=1, \ldots, N}
$$
such that for any $m$ the components $g_{i j}^m$ of $g$ in $\left(\Omega_m, \varphi_m\right)$ satisfy
$$
\frac{1}{2} \delta_{i j} \leq g_{i j}^m \leq 2 \delta_{i j}
$$
as bilinear forms. Let $\left(\eta_m\right)$ be a smooth partition of unity subordinate to the covering $\left(\Omega_m\right)$. For any $u \in C^{\infty}(M)$ and any $m$, one has that
$$
\int_M\left|\eta_m u\right|^{n /(n-1)} d v(g) \leq 2^{n / 2} \int_{R^n}\left|\left(\eta_m u\right) \circ \varphi_m^{-1}(x)\right|^{n /(n-1)} d x
$$
and
$$
\int_M\left|\nabla\left(\eta_m u\right)\right| d v(g) \geq 2^{-(n+1) / 2} \int_{R^n}\left|\nabla\left(\left(\eta_m u\right) \circ \varphi_m^{-1}\right)(x)\right| d x
$$
Independently, by the Sobolev embedding for $({\mathbb{R}^n},e)$, \autoref{lem33}, \cite{H1} Theorem 2.5.
\begin{Lemma}
Let $q \in[1, n)$ and let $p$ be such that $1 / p=1 / q-1 / n$. Then for any $u \in H_1^q\left(\mathbb{R}^n\right)$,
$$
\left(\int_{R^n}|u|^p d x\right)^{1 / p} \leq \frac{p(n-1)}{2 n}\left(\int_{R^n}|\nabla u|^q d x\right)^{1 / q}
$$
In particular, for any real numbers $1 \leq q<p$ and any integers $0 \leq m<k$ satisfying $1 / p=1 / q-(k-m) / n, H_k^q\left(\mathbb{R}^n\right) \subset H_m^p\left(\mathbb{R}^n\right)$.
\label{lem33}
\end{Lemma}

\noindent
we can get
$$
\left(\int_{R^n}\left|\left(\eta_m u\right) \circ \varphi_m^{-1}(x)\right|^{n /(n-1)} d x\right)^{(n-1) / n} \leq \frac{1}{2} \int_{R^n}\left|\nabla\left(\left(\eta_m u\right) \circ \varphi_m^{-1}\right)(x)\right| d x
$$
for any $m$. As a consequence, for any $u \in C^{\infty}(M)$,
$$
\begin{aligned}
\left(\int_M|u|^{n /(n-1)} d v(g)\right)^{(n-1) / n} & \leq \sum_{m=1}^N\left(\int_M\left|\eta_m u\right|^{n /(n-1)} d v(g)\right)^{(n-1) / n} \\
&\leq  2^{n-1} \sum_{m=1}^N \int_M\left|\nabla\left(\eta_m u\right)\right| d v(g) \\
&\leq  2^{n-1} \int_M|\nabla u| d v(g) \\
& +2^{n-1}\left(\max _M \sum_{m=1}^N\left|\nabla \eta_m\right|\right) \int_M|u| d v(g)
\end{aligned}
$$
Hence, for any $u \in C^{\infty}(M)$,
$$
\left(\int_M|u|^{n /(n-1)} d v(g)\right)^{(n-1) / n} \leq A\left(\int_M|\nabla u| d v(g)+\int_M|u| d v(g)\right)
$$
where
$$
A=2^{n-1}\left(1+\max _M \sum_{m=1}^N\left|\nabla \eta_m\right|\right)
$$
This ends the proof of the theorem.
\end{Proof}

\vskip 3pt
As an immediate corollary to \autoref{thm35} , one has the following: Just note here that since $M$ is assumed to be compact, $(M, g)$ has finite volume. Hence, for $1 \leq q \leq q^{\prime}, L^{q^{\prime}}(M) \subset L^q(M)$

\begin{Corollary}
    Let $(M, g)$ be a smooth, compact Riemannian n-manifold, and let $q$ and $p_0$ real be such that $q \in[1, n)$ and $1 / p_0=1 / q-1 / n$. Then $H_1^q(M) \subset$ $L^p(M)$ for any $p \in\left[1, p_0\right]$.
\label{cor31}
\end{Corollary}

\subsubsection{When 1/q-(k-m)/n<0}

The purpose of this section is to discuss Sobolev embeddings in their second part. For dimension reasons and the sake of clarity, we will be brief on the subject.
Given $(M, g)$ a smooth, compact Riemannian manifold, we define the norm $\|\cdot\|_{C^m}$ on $C^m(M)$ by
$$
\|u\|_{C^m}=\sum_{j=0}^m \max _{x \in M}\left|\left(\nabla^j u\right)(x)\right|
$$
One then has the following:
\begin{Theorem}
    Let $(M, g)$ be a smooth, compact Riemannian ${n}$-manifold, ${q \geq 1}$ real, and ${m<k}$ two integers. \textbf{If $\bm{1 / q<(k-m) / n}$, then $\bm{H_k^q(M) \subset C^m(M)}$.}
\label{thm36}
\end{Theorem}

\begin{Proof}
First we prove that for $q>n, H_1^q(M) \subset C^0(M)$. Since $M$ is compact, $M$ can be covered by a finite number of charts
$$
\left(\Omega_s, \varphi_s\right)_{s=1 \ldots ., N}
$$
such that for any $s$ the components $g_{i j}^s$ of $g$ in $\left(\Omega_s, \varphi_s\right)$ satisfy
$$
\frac{1}{2} \delta_{i j} \leq g_{i j}^s \leq 2 \delta_{i j}
$$
as bilinear forms. Let $\left(\eta_s\right)$ be a smooth partition of unity subordinate to the covering $\left(\Omega_s\right)$. Given $u \in C^{\infty}(M)$,
$$
\left\|\eta_s u\right\|_{C^0}=\left\|\left(\eta_s u\right) \circ \varphi_s^{-1}\right\|_{C^0}
$$
for all $s$. Independently, starting from the inequalities satisfied by the $g_{i j}^s$ 's, one easily gets that there exists $C>0$ such that for any $s$ and any $u \in C^{\infty}(M)$,
$$
\left\|\left(\eta_s u\right) \circ \varphi_s^{-1}\right\|_{H_1^q} \leq C\left\|\eta_s u\right\|_{H_1^q}
$$
where the $H_1^q$-norm in the left-hand side of this inequality is with respect to the Euclidean space. Since $H_1^q\left(\mathbb{R}^n\right) \subset C_B^0\left(\mathbb{R}^n\right)$, this leads to the existence of some $A>0$ such that for any $s$ and any $u \in C^{\infty}(M)$,
$$
\left\|\eta_s u\right\|_{C^0} \leq A\left\|\eta_s u\right\|_{H_1^q}
$$
Clearly, there exists $B>0$ such that for any $u \in C^{\infty}(M)$,
$$
\sum_{s=1}^N\left\|\eta_s u\right\|_{H_1^q} \leq B\|u\|_{H_1^q}
$$
For instance, one can take
$$
B=\sum_{s=1}^N\left(\max _{x \in M} \eta_s+\max _{x \in M}\left|\nabla \eta_s\right|\right)
$$
Hence,
$$
\|u\|_{C^0} \leq \sum_{s=1}^N\left\|\eta_s u\right\|_{C^0} \leq A \sum_{s=1}^N\left\|\eta_s u\right\|_{H_1^q} \leq A B\|u\|_{H_1^q}
$$
This proves the above claim, i.e., that for ( $M, g$ ) compact and $q>n, H_1^q(M) \subset$ $C^0(M)$. Let us now prove that for $q, k$, and $m$ as in the theorem, $H_k^q(M) \subset C^m(M)$. Given $u \in C^{\infty}(M)$, one has by Kato's inequality that for any integer $s$,
$$
|\nabla| \nabla^s u|| \leq\left|\nabla^{s+1} u\right|
$$
Let $s \in\{0, \ldots, m\}$. According to the first part of the Sobolev embedding theorem, \autoref{thm35}, one has that $H_{k-s}^q(M) \subset H_1^{p_s}(M)$ where
$$
\frac{1}{p_s}=\frac{1}{q}-\frac{k-s-1}{n}
$$
In particular, $p_s>n$, so that, according to what has been said above, $H_1^{p_s}(M) \subset$ $C^0(M)$. Hence, for any $s \in\{0, \ldots, m\}$, and any $u \in C^{\infty}(M)$,
$$
\left\|\nabla^s u\right\|_{C^0} \leq C_1(s)\left\|\nabla^s u\right\|_{H_1^{p_s}} \leq C_2(s)\left\|\nabla^s u\right\|_{H_{k-s}^q} \leq C_2(s)\|u\|_{H_k^q}
$$
by Kato's inequality, and where $C_1(s)$ and $C_2(s)$ do not depend on $u$. As an immediate consequence of such inequalities, one gets that $H_k^q(M) \subset C^m(M)$ for $k, q$, and $m$ as above. This ends the proof of the theorem.
\end{Proof}

\vskip 3pt
What's more, some improvements of the above result involving Hölder spaces can be obtained, which will be discussed now. For the sake of clarity, we restrict ourselves to the case $k=1$ and $m=0$. Given $(M, g)$ a smooth, compact Riemannian manifold, and $\lambda \in(0,1)$, let $C^\lambda(M)$ be the set of continuous functions $u: M \rightarrow \mathbb{R}$ for which the norm
$$
\|u\|_{C^\lambda}=\max _{x \in M}|u(x)|+\max _{x \neq y \in M} \frac{|u(y)-u(x)|}{d_g(x, y)^\lambda}
$$
is finite, where $d_g$ denotes the distance associated to $g$. One then has the following:

\begin{Theorem}
    Let $(M, g)$ be a smooth, compact Riemannian $n$-manifold, $q \geq 1$ real, and $\lambda \in(0,1)$ real. If $1 / q \leq(1-\lambda) / n$, then $H_1^q(M) \subset C^\lambda(M)$.
\label{thm37}
\end{Theorem}

\begin{Proof}
    Let $C_B^\lambda \left(\mathbb{R}^u\right)$ be the space of smooth functions $u: \mathbb{R}^n \rightarrow \mathbb{R}$ for which the norm
$$
\|u\|_{C^\lambda}=\max _{x \in R^n}|u(x)|+\max _{x \neq y \in R^n} \frac{|u(y)-u(x)|}{|y-x|^\lambda}
$$
is finite. By classical results of Morrey, see Adams \cite{A}, one has that for $q \geq 1$ real such that $1 / q \leq(1-\lambda) / n, H_1^q\left(\mathbb{R}^n\right) \subset C_B^\lambda\left(\mathbb{R}^n\right)$. Consider now a smooth, compact Riemannian $n$-manifold $(M, g)$, and $q \geq 1$ as abovc. Since $M$ is compact, one can once more assume that $M$ is covered by a finite number of charts
$$
\left(\Omega_s, \varphi_s\right)_{s=1 \ldots . . N}
$$
such that for any $s$ the components $g_{i j}^s$ of $g$ in $\left(\Omega_{\mathrm{s}}, \varphi_{\mathrm{s}}\right)$ satisfy
$$
\frac{1}{2} \delta_{i j} \leq g_{i j}^s \leq 2 \delta_{i j}
$$
as bilinear forms. Without loss of generality, one can also assume that the $\Omega_s$ 's are convex with respect to $g$. Let $\left(\eta_s\right)$ be a smooth partition of unity subordinate to the
covering $\left(\Omega_s\right)$. Starting from the inequalities satisfied by the $g_{i j}^s$ 's, one clearly gets that there exist $C_1>0$ and $C_2>0$ such that for any $s$ and any $u \in C^{\infty}(M)$,
$$
\begin{aligned}
\left\|\eta_s u\right\|_{C^\lambda} & \leq C_1\left\|\left(\eta_s u\right) \circ \varphi_s^{-1}\right\|_{C^\lambda} \\
\left\|\left(\eta_s u\right) \circ \varphi_s^{-1}\right\|_{H_1^q} & \leq C_2\left\|\eta_s u\right\|_{H_1^q}
\end{aligned}
$$
where the norms in the right-hand side of the first inequality, and in the lefthand side of the second inequality, are with respect to the Euclidean space. Since $H_1^q\left(\mathbb{R}^n\right) \subset C_B^\lambda\left(\mathbb{R}^n\right)$, one gets from the above inequalities that there exists $C_3>0$ such that for any $s$ and any $u \in C^{\infty}(M)$,
$$
\left\|\eta_s u\right\|_{C^{\lambda }} \leq C_3\left\|\eta_s u\right\|_{H_1^q}
$$
Independently, one clearly has that there exists $B>0$ such that for any $u \in$ $C^{\infty}(M)$
$$
\sum_{s=1}^N\left\|\eta_s u\right\|_{H_1^q} \leq B\|u\|_{H_1^q}
$$
For instance, one can take
$$
B=\sum_{s=1}^N\left(\max _{x \in M} \eta_s+\max _{x \in M}\left|\nabla \eta_s\right|\right)
$$
Hence, for any $u \in C^{\infty}(M)$,
$$
\|u\|_{C^\lambda} \leq \sum_{s=1}^N\left\|\eta_s u\right\|_{C^\lambda} \leq B C_3\|u\|_{H_1^q}
$$
This ends the proof of the theorem.
\end{Proof}

\subsection{Compactness of Embeddings}

We discuss in this section compactness properties of Sobolev embeddings. Given $\left(E,\|\cdot\|_E\right)$ and $\left(F,\|\cdot\|_F\right)$ two normed vector spaces, $E$ being a subspace of $F$, recall that the embedding of $E$ in $F$ is said to be compact if bounded subsets of $\left(E,\|\cdot\|_E\right)$ are relatively compact in $\left(F,\|\cdot\|_F\right)$. This means, again, that bounded sequences in $\left(E,\|\cdot\|_E\right.$ ) possess convergent subsequences in $\left(F,\|\cdot\|_F\right)$. Clearly, if the embedding of $E$ in $F$ is compact, it is also continuous. We prove here the following result. On what concerns its second part, we restrict once more our attention to the case $k=1$ and $m=0$.

\begin{Theorem}
    Let $(M, g)$ be a smooth, compact Riemannian ${n}$-manifold.
    
    \vskip 3pt
    \noindent
    (i) For any integers ${j \geq 0}$ and ${m \geq 1}$, any real number ${q \geq 1}$, and any real number ${p}$ such that \textbf{$\bm{1 \leq p<n q /(n-m q)}$, the embedding of $\bm{H_{j+m}^q(M)}$ in $\bm{H_j^p(M)}$ is compact.} In particular, for any ${q \in[1, n)}$ real and any ${p \geq 1}$ such that ${1 / p>1 / q-1 / n}$, the embedding of ${H_1^q(M)}$ in ${L^p(M)}$ is compact.
    
    \vskip 3pt
    \noindent
    (ii) \textbf{For $\bm{q>n}$, the embedding of $\bm{H_1^q(M)}$ in $\bm{C^\lambda(M)}$ is compact for any $\bm{\lambda \in$ $(0,1)}$ such that $\bm{(1-\lambda) q>n}$.} In particular, the embedding of ${H_1^q(M)}$ in ${C^0(M)}$ is compact.
\label{thm38}
\end{Theorem}
In order to prove \autoref{thm38}, we need first the following lemma. Such a lemma can be seen as the analogue of the Ascoli theorem. Given $A$ and $B$ two subsets of $\mathbb{R}^n$, $\operatorname{dist}(A, B)$ denotes the distance from $A$ to $B$.

\begin{Lemma}
Let $\Omega$ be an open subset of $\mathbb{R}^n, p \geq 1$ real, and $\mathcal{H}$ a bounded subset of $L^p(\Omega)$. Then $\mathcal{H}$ is relatively compact in $L^p(\Omega)$ if and only if for any $\varepsilon>0$. there exists a compact subset $K \subset \Omega$, and there exists $0<\delta<\operatorname{dist}(K, \partial \Omega)$ such that
$$
\int_{\Omega \backslash K}|u(x)|^p d x<\varepsilon \quad \text { and } \quad \int_K|u(x+y)-u(x)|^p d x<\varepsilon
$$
for any $u \in \mathcal{H}$ and any $y$ such that $|y|<\delta$.
\label{lem34}
\end{Lemma}
The proof of the result can be found in Adams \cite{A}, Theorem 2.21. 
Given $\Omega$ an open subset of $\mathbb{R}^n$, and $q \geq 1$ real, we denote by $H_{0,1}^q(\Omega)$ the closure of $\mathcal{D}(\Omega)$ in $H_1^q(\Omega)$. Then we prove the following:

\begin{Lemma}
    Let $\Omega$ be a bounded, open subset of $\mathbb{R}^n, q \in[1, n)$ real, and $p \geq 1$ real such that $1 / p>1 / q-1 / n$. Then the embedding of $H_{0.1}^q(\Omega)$ in $L^p(\Omega)$ is compact.
\label{lem35}
\end{Lemma}

\begin{Proof}
    Let $q \in[1, n)$ and $p \geq 1$ such that $1 / p>1 / q-1 / n$ be given. Let also $p_0$ be defined by $p_0=n q /(n-q)$. By \autoref{lem33}, one has that $H_{0,1}^q(\Omega) \subset$
    $L^{p_0}(\Omega)$. Moreover, there exists $A>0$ such that for any $u \in H_{0.1}^q(\Omega)$,
$$
\left(\int_{\Omega}|u|^{p_0} d x\right)^{1 / p_0} \leq A\left(\int_{\Omega}|\nabla u|^q d x\right)^{1 / q}
$$
Noting that $\Omega$ has finite volume, one clearly gets that $H_{0.1}^q(\Omega) \subset L^p(\Omega)$. There is still to prove that this embedding is compact. Let $\mathcal{H}$ be a bounded subset of $H_{0,1}^q(\Omega)$. There exists $C>0$ such that for any $u \in \mathcal{H}$,
$$
\int_{\Omega}|\nabla u|^q d x+\int_{\Omega}|u|^q d x \leq C
$$
For $j$ integer, set
$$
K_j=\left\{x \in \Omega \text { s.t. } \operatorname{dist}(x, \partial \Omega) \geq \frac{2}{j}\right\}
$$
Given $u \in \mathcal{H}$, and according to Hölder's inequality,
$$
\begin{aligned}
\int_{\Omega \backslash K_j}|u| d x & \leq\left(\int_{\Omega \backslash K_j}|u|^{p_0} d x\right)^{1 / p_0}\left(\int_{\Omega \backslash K_j} d x\right)^{1-\frac{1}{p_0}} \\
& \leq A C^{1 / q}\left(\int_{\Omega \backslash K_j} d x\right)^{1-\frac{1}{p_0}}
\end{aligned}
$$
Let $\varepsilon>0$ be given. One then gets that for $j$ big enough, and any $u \in \mathcal{H}$,
$$
\int_{\Omega \backslash K_j}|u| d x<\varepsilon
$$
From now on, let $y$ be such that $|y|<1 / j$. If $x \in K_j$, then $x+y \in K_{2 j}$. For $u \in \mathcal{D}(\Omega)$ one can then write that
$$
\begin{aligned}
\int_{K_j}|u(x+y)-u(x)| d x & \leq \int_{K_j} d x \int_0^1\left|\frac{d}{d t} u(x+t y)\right| d t \\
& \leq|y| \int_{K_{2 j}}|\nabla u| d x \\
& \leq|y| \int_{\Omega}|\nabla u| d x
\end{aligned}
$$
Since $\mathcal{D}(\Omega)$ is dense in $H_{0.1}^q(\Omega)$, one gets that for any $u \in H_{0.1}^q(\Omega)$, and any $y$ such that $|y|<1 / j$,
$$
\int_{K_j}|u(x+y)-u(x)| d x \leq|y| \int_{\Omega}|\nabla u| d x
$$
By Hölder's inequality
$$
\int_{\Omega}|\nabla u| d x \leq\left(\int_{\Omega}|\nabla u|^q d x\right)^{1 / q}\left(\int_{\Omega} d x\right)^{1-\frac{1}{q}}
$$
One then gets that there exists $B>0$ such that for any $u \in \mathcal{H}$,
$$
\int_{\Omega}|\nabla u| d x \leq B
$$
Hence, for any $u \in \mathcal{H}$, and any $y$ such that $|y|<\min \left(\frac{\varepsilon}{B}, \frac{1}{j}\right)$,
$$
\int_{K_j}|u(x+y)-u(x)| d x<\varepsilon
$$
By \autoref{lem34} this implies that $\mathcal{H}$ is relatively compact in $L^1(\Omega)$. One gets that $\mathcal{H}$ is relatively compact in $L^p(\Omega)$ as follows: If $\left(u_m\right)$ is a sequence in $\mathcal{H}$, then, by Hölder's inequality,
$$
\begin{aligned}
\int_{\Omega}\left|u_{m_1}-u_{m_2}\right|^p d x & \leq\left(\int_{\Omega}\left|u_{m_1}-u_{m_2}\right| d x\right)^k\left(\int_{\Omega}\left|u_{m_1}-u_{m_2}\right|^{p_0} d x\right)^{1-k} \\
& \leq\left(2 A C^{1 / q}\right)^{(1-k) p_0}\left(\int_{\Omega}\left|u_{m_1}-u_{m_2}\right| d x\right)^k
\end{aligned}
$$
where $k=\frac{p_0-p}{p_0-1}$. From such an inequality, and from the fact that $\mathcal{H}$ is relatively compact in $L^1(\Omega)$, one easily gets that $\mathcal{H}$ is also relatively compact in $L^p(\Omega)$. This ends the proof of the lemma.
\end{Proof}

Now that such results have been stated, we prove \autoref{thm38}. For the sake of clarity, concerning point (i), we restrict ourselves to the case $j=0$ and $m=1$. In other words, we prove that for any smooth, compact Riemannian $n$-manifold, any $q \in[1, n)$, and any $p \geq 1$ such that $p<n q /(n-q)$, the embedding of $H_1^q(M)$ in $L^p(M)$ is compact. 
The proof that the other embeddings are also compact can be found in Aubin \cite{Aubin}.

\begin{Proof1}
    (i) Since $M$ is compact, $M$ can be covered by a finite number of charts
$$
\left(\Omega_s, \varphi_s\right)_{s=1 \ldots . . N}
$$
such that for any $s$ the components $g_{i j}^s$ of $g$ in $\left(\Omega_s, \varphi_s\right)$ satisfy
$$
\frac{1}{2} \delta_{i j} \leq g_{i j}^s \leq 2 \delta_{i j}
$$
as bilinear forms. Let $\left(\eta_s\right)$ be a smooth partition of unity subordinate to the covering $\left(\Omega_s\right)$. Given $\left(u_m\right)$ a bounded sequence in $H_1^q(M)$, and for any $s$, we let
$$
u_m^s=\left(\eta_s u_m\right) \circ \varphi_s^{-1}
$$
Clearly, $\left(u_m^s\right)$ is a bounded sequence in $H_{0.1}^q\left(\varphi_s\left(\Omega_s\right)\right)$ for any $s$. By \autoref{lem35} one then gets that a subsequence $\left(u_m^s\right)$ of $\left(u_m^s\right)$ is a Cauchy sequence in $L^p\left(\varphi_s\left(\Omega_s\right)\right.$ ). Let $\left(u_m\right)$ be a subsequence of $\left(u_m\right)$ chosen such that for any $s,\left(u_m^s\right)$ is a Cauchy sequence in $L^p\left(\varphi_s\left(\Omega_s\right)\right)$. Coming back to the inequalities satisfied by the $g_{i j}^s$ 's, one easily gets that for any $s,\left(\eta_s u_m\right)$ is a Cauchy sequence in $L^p(M)$. But for any $m_1$ and $m_2$,
$$
\left\|u_{m_2}-u_{m_1}\right\|_p \leq \sum_{s=1}^N\left\|\eta_s u_{m_2}-\eta_s u_{m_1}\right\|_p
$$
where $\|\cdot\|_p$ stands for the $L^p$-norm. Hence, $\left(u_m\right)$ is a Cauchy sequence in $L^p(M)$. 

\vskip 3pt
(ii) Let $\lambda \in(0,1)$ be such that $(1-\lambda) q>n$, and let $\alpha \in(0,1)$ be such that $\lambda<\alpha$ and $(1-\alpha) q>n$. 
By \autoref{thm37}, one has that $H_1^q(M) \subset C^\alpha(M)$. Given $\mathcal{H}$ a bounded subset in $H_1^q(M)$, one then gets that there exists $C>0$ such that for any $u \in \mathcal{H},\|u\|_{C^\alpha} \leq C$. By Ascoli's theorem, $\mathcal{H}$ is relatively compact in $C^0(M)$. From now on, let $\left(u_m\right)$ be a sequence in $\mathcal{H}$. Up to the extraction of a subsequence, $\left(u_m\right)$ converges to some $u$ in $C^0(M)$. Clearly, $u \in C^\alpha(M)$ and $\|u\|_{C^\alpha} \leq C$. Setting $v_m=u_m-u$, one then gets that $\left\|v_m\right\|_{C^\alpha} \leq 2 C$, and that for all $x$ and $y$ in $M$, $x \neq y$,
$$
\begin{aligned}
\frac{\left|v_m(y)-v_m(x)\right|}{d_g(x, y)^\lambda} & =\left(\frac{\left|v_m(y)-v_m(x)\right|}{d_g(x, y)^\alpha}\right)^{\frac{\lambda}{\alpha}}\left|v_m(y)-v_m(x)\right|^{1-\frac{\lambda}{\alpha}} \\
& \leq(2 C)^{\frac{\lambda}{\alpha}}\left|v_m(y)-v_m(x)\right|^{1-\frac{\lambda}{\alpha}} \\
& \leq(2 C)^{\frac{\lambda}{\alpha}}\left(2\left\|v_m\right\|_{C^0}\right)^{1-\frac{\lambda}{\alpha}}
\end{aligned}
$$
Since ($v_m$) converges to 0 in $C^0(M)$, one gets from such inequalities that ($v_m$) converges to 0 in $C^\lambda(M)$. This proves the theorem.
\end{Proof1}

\newpage



\section{Complete Manifolds without Boundary}

We mainly discuss in this section the validity of Sobolev embeddings and Sobolev inequalities for complete manifolds. Density problems are first discussed in Section 4.1. Sobolev embeddings and Sobolev inequalities are then studied in Sections 4.2. 
Given $(M, g)$ a smooth, complete Riemannian manifold, $k$ an integer, and $p \geq$ 1 real, we define
$$
H_{0, k}^q(M)=\text { closure of } \mathcal{D}(M) \text { in } H_k^q(M)
$$
where $\mathcal{D}(M)$ is the space of smooth functions with compact support in $M$. As already mentioned, we start in this section with density problems for Sobolev spaces.

\subsection{Density Problems}
Let $(M, g)$ be a smooth, complete Riemannian manifold. For $H_{0, k}^q(M)$ defined as above, we discuss in this section the case of equality $H_{0, k}^q(M)=H_k^q(M)$. In other words, we try to find for which complete manifolds $(M, g)$ one has that
$\mathcal{D}(M)$ is dense in $H_k^q(M)$. The completion for such a study is necessary, in the sense that one can construct many noncomplete manifolds for which $H_{0, k}^q \neq H_k^q$. For instance, set $\Omega$ a bounded, open subset of $\mathbb{R}^n$ endowed with the Euclidean metric $e$. One easily checks that in such a situation, $H_{0,1}^2(\Omega) \neq H_1^2(\Omega)$. Consider for this purpose the scalar product $\langle\cdot, \cdot\rangle$ of \autoref{prp31} (with $g=e$ and $k=1$ ), and let $u \in C^{\infty}(\Omega) \cap H_1^2(\Omega)$ be such that $\Delta_e u+u=0, u \not \equiv 0$, where $\Delta_e$ is the Laplacian of $e$ (with the minus sign convention). For instance, one can take $u=\sinh x_1, x_1$ the first coordinate in $\mathbb{R}^n$. Then for any $v \in \mathcal{D}(\Omega)$,

$$
\langle u, v\rangle=\int_{\Omega}\left(\Delta_e u+u\right) v d x=0
$$
so that $u \notin H_{0.1}^2(\Omega)$. This proves the above claim. On the contrary, one has the following result:


\subsubsection{Relations between $H_{0 . k}^q\left(\mathbb{R}^n\right)$ and $H_k^q\left(\mathbb{R}^n\right)$}
\begin{Proposition}
    For any $k$ an integer and any $q \geq 1$ real, $H_{0 . k}^q\left(\mathbb{R}^n\right)=H_k^q\left(\mathbb{R}^n\right)$.
\label{prp41}
\end{Proposition}

\begin{Proof}
    Let $f: \mathbb{R} \rightarrow \mathbb{R}$ be a smooth decreasing function such that
$$
f(t)=1 \text { if } t \leq 0 \quad\text { and }\quad f(t)=0 \text { if } t \geq 1
$$
As one can easily check, it is sufficient to prove that any $u \in C^{\infty}\left(\mathbb{R}^n\right) \cap H_k^q\left(\mathbb{R}^n\right)$ can be approximated in $H_k^q\left(\mathbb{R}^n\right)$ by functions of $\mathcal{D}\left(\mathbb{R}^n\right)$. For $m$ an integer and $u$ some smooth function in $H_k^q\left(\mathbb{R}^n\right)$, set
$$
u_m(x)=u(x) f(r-m)
$$
where $r$ denotes the distance from 0 to $x$. Clearly, $u_m \in \mathcal{D}\left(\mathbb{R}^n\right)$ for any $m$. On the one hand, one has by Leibnitz's formula that for any $s$ integer and any $m$,
$$
\left|\nabla^s\left(u_m-u\right)\right| \leq C_1 \sum_{j=0}^s\left|\nabla^j u\right| \cdot\left|\nabla^{s-j}\left(f_m-1\right)\right|
$$
where $C_1>0$ is independent of $m$, and $f_m(x)=f(r-m)$. In particular, noting that $\left|\nabla^s r\right|$ is bounded for $s \geq 1$ and $r \geq 1$, one gets that for any $s$ integer, and any $m \geq 1$,
$$
\left|\nabla^s\left(u_m-u\right)\right| \leq C_2 \sum_{j=0}^s\left|\nabla^j u\right|
$$
where $C_2>0$ is independent of $m$. On the other hand, one clearly has that for any $s$ integer, and with respect to the pointwise convergence,
$$
\lim _{m \rightarrow+\infty} \nabla^s u_m=\nabla^s u
$$
Since for any $s \leq k,\left|\nabla^s u\right| \in L^p\left(\mathbb{R}^n\right)$, the proposition easily follows from the Lebesgue dominated convergence theorem.
\end{Proof}

\subsubsection{Relations between $H_{0 . k}^q(M)$ and $H_k^q(M)$}
When dealing with arbitrary, complete Riemannian manifolds $(M, g)$, one can hope that the equality $H_{0 . k}^q(M)=H_k^q(M)$ still holds. As surprising as it may seem, such a question is open for $k \geq 2$. For $k=1$, this is the content of \autoref{thm34} of section 2; things work for the best and one has the following result of Aubin \cite{Aubin-}.

\begin{Theorem}
    Given $(M, g)$ a smooth, complete Riemannian manifold, $H_{0.1}^q(M)$ $=H_1^q(M)$ for any $q \geq 1$ real.
\label{thm41}
\end{Theorem}

The situation for $k \geq 2$ seems to be more complicated, and assumptions on the manifolds are now needed (at least at the present state of the field). Aubin \cite{Aubin-} proved that for any $q \geq 1$ and $k \geq 2, \mathcal{D}(M)$ is dense in $H_k^q(M)$ provided that $(M, g)$ has a positive injectivity radius and that the Riemann curvature of $(M, g)$ is bounded up to the order $k-2$. 
Hebey \cite{H2} proved that the above result still holds if the assumptions on the Riemann curvature are replaced by similar assumptions on the Ricci curvature. Moreover, thanks to the Bochner-Lichnerowicz-Weitzenböck formula, something special happens in the case $k=p=2$, where only a lower bound on the Ricci curvature is needed instead of a global bound. This is what we are going to discuss now. Let us start with the general case.

\begin{Proposition}
    Let $(M, g)$ be a smooth, complete Riemannian manifold with positive injectivity radius, and let $k \geq 2$ be an integer. We assume that for $j=$ $0, \ldots, k-2,\left|\nabla^j \operatorname{Rc}_{(M . g)}\right|$ is bounded. Then for any $q \geq 1$ real, $H_{0 . k}^q(M)=$ $H_k^q(M)$
\label{prp42}
\end{Proposition}

\begin{Proof}
Suppose that the injectivity radius $\operatorname{inj}_{(M . g)}$ of $(M, g)$ is positive, and that there exists $C>0$ such that for any $j=0, \ldots, k-2,\left|\nabla^j \operatorname{Rc}_{(M . g)}\right| \leq C$. By \autoref{thm22} one has that for any real numbers $Q>1$ and $\alpha \in(0,1)$, the $C^{k-1 . \alpha}$ harmonic radius $r_H=r_H(Q, k-1, \alpha)$ is positive. Fix, for instance, $Q=4$ and $\alpha=1 / 2$. (As one will see, $\alpha$ plays no role in the following of the proof). For any $x \in M$ one then has that there exists some harmonic chart $\varphi: B_x\left(r_H\right) \rightarrow \mathbb{R}^n$ such that the points 1 and 2 of \autoref{def21} are satisfied with $Q=4$ and $\alpha=1 / 2$. (Without loss of generality, we can also assume that $\varphi(x)=0$ ). In particular, we get that for any $r \leq r_H$
$$
B_0(r / 2) \subset \varphi\left(B_x(r)\right) \subset B_0(2 r)
$$
where for $t \geq 0$ real $B_0(t)$ denotes the Euclidean ball of center 0 and radius $t$. Let $\beta \in \mathcal{D}\left(\mathbb{R}^n\right)$ be such that
$$
0 \leq \beta \leq 1, \quad \beta=1 \quad \text { on } B_0\left(\frac{r_H}{8}\right), \quad \beta=0 \quad \text { on } \mathbb{R}^n \backslash B_0\left(\frac{r_H}{4}\right)
$$
As a consequence of the above inclusions, we get that $\beta \circ \varphi \in \mathcal{D}(M)$ satisfies
$$
0 \leq \beta \circ \varphi \leq 1, \quad \beta \circ \varphi=1 \quad \text { on } B_x\left(\frac{r_H}{16}\right), 
\quad\beta \circ \varphi=0  \text { on } M \backslash B_x\left(\frac{r_H}{2}\right)
$$
From now on, let $\left(x_i\right)$ be a sequence of points of $M$ such that

$$M=\bigcup_i B_{x_i}\left(\frac{r_H}{16}\right), \quad\left(B_{x_i}\left(\frac{r_H}{2}\right)\right)_i \;\text{is uniformly locally finite}$$
The existence of such a sequence is given by \autoref{lem21}. Let $\varphi_i: B_{x_i}\left(r_H\right) \rightarrow \mathbb{R}^n$ be as above and set $\beta_i=\beta \circ \varphi_i$. Since the components of the metric tensor are
$C^{k-1}$-controlled in the charts $\left(B_{x_i}\left(r_H\right), \varphi_i\right)$, one easily gets that there exists $C>0$ such that for any $i$ and any $m=0, \ldots, k,\left|\nabla^m \beta_i\right| \leq C$. Let us now set
$$
\eta_i=\frac{\beta_i}{\sum_j \beta_j}
$$
As a consequence of what we have said above, $\left(\eta_i\right)$ is a smooth partition of unity subordinate to the covering ($B_{x_i}\left(\frac{r_H}{2}\right)$), and since this covering is uniformly locally finite, one easily obtains that there exists some constant $\tilde{C} \geq 1$ such that for any $m=0, \ldots, k, \sum_i\left|\nabla^m \eta_i\right|$ $\leq \tilde{C}$. Now fix $u \in \mathcal{C}_k^p(M)$ where $p \geq 1$ is some given real number. The proposition will obviously be proved if we show that for any $\varepsilon>0$ there exists $u_0 \in D(M)$ such that $\left\|u-u_0\right\|_{H_k^p}<\varepsilon$. Fix $\varepsilon>0$ and let $\Omega \subset M$ be some bounded subset of $M$ such that
$$
\sum_{m=0}^k C_{k+1}^{m+1}\left(\int_{M \backslash \Omega}\left|\nabla^m u\right|^p d v(g)\right)^{1 / p}<\varepsilon / \tilde{C}
$$
where $\tilde{C}$ is as above and
$$
C_{k+1}^{m+1}=\frac{(k+1)!}{(m+1)!(k-m)!}
$$
Since the covering ($B_{x_i}\left(\frac{r_H}{2}\right)$) is uniformly locally finite, one easily obtains that there exists some integer $N$ such that for any $i \geq N+1, B_{x_i}\left(\frac{r_H}{2}\right) \cap \Omega=\emptyset$. Set $u_0=(1-\eta) u$ where $1-\eta=\sum_{i=1}^N \eta_i$. Then $u_0 \in \mathcal{D}(M)$ and
$$
\left\|u-u_0\right\|_{H_k^p} \leq \sum_{m=0}^k\left\|\nabla^m(\eta u)\right\|_p
$$
where $\|\cdot\|_p$ stands for the norm of $L^p(M)$. But
$$
\left|\nabla^m(\eta u)\right| \leq \sum_{j=0}^m C_m^j\left|\nabla^j \eta \| \nabla^{m-j} u\right|
$$
and since $\operatorname{supp} \eta \subset M \backslash \Omega$ and $\sum_i\left|\nabla^j \eta_i\right| \leq \tilde{C}$ for any $j=0, \ldots, k$, we get that
$$
\left\|\nabla^m(\eta u)\right\|_p \leq \tilde{C} \sum_{j=0}^m C_m^j\left(\int_{M \backslash \Omega}\left|\nabla^j u\right|^p d v(g)\right)^{1 / p}
$$
As a consequence, noting that for any $0 \leq m \leq k, \sum_{j=m}^k C_j^m=C_{k+1}^{m+1}$, we get that
$$
\begin{aligned}
\left\|u-u_0\right\|_{H_k^p} & \leq \tilde{C} \sum_{m=0}^k \sum_{j=0}^m C_m^j\left(\int_{M \backslash \Omega}\left|\nabla^j u\right|^p d v(g)\right)^{1 / p} \\
& =\tilde{C} \sum_{m=0}^k\left(\sum_{j=m}^k C_j^m\right)\left(\int_{M \backslash \Omega}\left|\nabla^m u\right|^p d v(g)\right)^{1 / p} \\
& =\tilde{C} \sum_{m=0}^k C_{k+1}^{m+1}\left(\int_{M \backslash \Omega}\left|\nabla^m u\right|^p d v(g)\right)^{1 / p}
\end{aligned}
$$
Since
$$
\sum_{m=0}^k C_{k+1}^{m+1}\left(\int_{M \backslash \Omega}\left|\nabla^m u\right|^p d v(g)\right)^{1 / p}<\varepsilon / \tilde{C}
$$
we have shown that for any $\varepsilon>0$ and any $u \in \mathcal{C}_k^p(M)$ there exists $u_0 \in \mathcal{D}(M)$ such that $\left\|u-u_0\right\|_{H_k^p}<\varepsilon$. As already mentioned, this ends the proof of the proposition.
\end{Proof}

\vskip 3pt
As a straightforward corollary to \autoref{prp42} one gets the following:

\begin{Corollary}
    For any Riemannian covering $(\tilde{M}, \tilde{g})$ of a compact Riemannian manifold $(M, g)$, for any $k$ integer, and any $q \geq 1$ real, $H_{0 . k}^q(M)=H_k^q(M)$.
\label{cor41}
\end{Corollary}


\vskip 3pt
\noindent
As already mentioned, thanks to the so-called Bochner-Lichnerowicz-Weitzenböck formula, something special happens in the case $k=p=2$. Here, one can replace the global bound on the Ricci curvature by a lower bound on the Ricci curvature.

\begin{Proposition}
    For any smooth, complete Riemannian manifold $(M, g)$ with positive injectivity radius and Ricci curvature bounded from below, $H_{0.2}^2(M)=$ $H_2^2(M)$.
\label{prp43}
\end{Proposition}

\begin{Proof}
Let $K_2^2(M)$ be the completion of
$$
\tilde{\mathcal{C}}_2^2(M)=\left\{u \in C^{\infty}(M) \,,\, u,|\nabla u|, \Delta_g u \in L^2(M)\right\}
$$
with respect to
$$
\|u\|_{K_2^2}=\left(\int_M u^2 d v(g)\right)^{1 / 2}+\left(\int_M|\nabla u|^2 d v(g)\right)^{1 / 2}+\left(\int_M\left|\Delta_g u\right|^2 d v(g)\right)^{1 / 2}
$$
Let also $K_{0,2}^2(M)$ be the closure of $\mathcal{D}(M)$ in $K_2^2(M)$. We assume that the Ricci curvature of $(M, g)$ is bounded from below by some $\lambda$, and that the injectivity radius of $(M, g)$ is positive. By \autoref{thm22}, one then gets that for any $Q>1$ real, and any $\alpha \in(0,1)$ real, the $C^{0 . \alpha}$-harmonic radius $r_H=r_H(Q, 0, \alpha)$ is positive. Noting that in a harmonic coordinate chart,
$$
\Delta_g u=-g^{i j} \partial_{i j} u
$$
for any $u \in C^{\infty}(M)$, similar arguments to those used in the proof of \autoref{prp42} prove that
$$
K_{0.2}^2(M)=K_2^2(M)
$$
Independently, one clearly has that for any $u \in C^{\infty}(M),\left|\Delta_g u\right|^2 \leq n\left|\nabla^2 u\right|^2$. Hence,
$$
H_2^2(M) \subset K_2^2(M)
$$
with the property that this embedding is continuous. Recall now that by the Bochner-Lichnerowicz-Weitzenböck formula, for any smooth function $u$ on $M$,
$$
\left\langle\Delta_g(d u), d u\right\rangle=\frac{1}{2} \Delta_g\left(|d u|^2\right)+|\nabla(d u)|^2+\operatorname{Rc}_{(M, g)}(\nabla u, \nabla u)
$$
Integrating this formula, one then gets that for any $u \in$ $\mathcal{D}(M)$,
$$
\begin{aligned}
\int_M\left|\nabla^2 u\right|^2 d v(g) & =\int_M\left|\Delta_g u\right|^2 d v(g)-\int_M \operatorname{Rc}_{(M, g)}(\nabla u, \nabla u) d v(g) \\
& \leq \int_M\left|\Delta_g u\right|^2 d v(g)+|\lambda| \int_M|\nabla u|^2 d v(g)
\end{aligned}
$$
Hence,
$$
\|u\|_{H_2^2} \leq(1+\sqrt{|\lambda|})\|u\|_{K_2^2}
$$
for any $u \in \mathcal{D}(M)$, and according to what we have just said, we get that
$$
H_{0.2}^2(M)=K_{0,2}^2(M)
$$
As a consequence,
$$
H_{0.2}^2(M) \subset H_2^2(M) \subset K_2^2(M)=K_{0.2}^2(M)=H_{0.2}^2(M)
$$
and this ends the proof of the proposition.
\end{Proof}

\subsection{Embeddings for Complete Manifolds without Boundary} 
\subsubsection{When 1/q-(k-m)/n>0}
Let us now discuss results where Sobolev embeddings in their first part do hold. As one will see, the situation is well understood when dealing with manifolds having the property that their Ricci curvature is bounded from below. 
In the 1970s, Aubin and Cantor proved that Sobolev embeddings were valid for complete manifolds with bounded sectional curvature and positive injectivity radius. About ten years later, Varopoulos proved that Sobolev embeddings do hold if the Ricci curvature of the manifold is bounded from below and if one has a lower bound for the volume of small balls which is uniform with respect to their center. By Croke's result a lower bound on the injectivity radius implies a lower bound on the volume of small balls which is uniform with respect to their center. One then has the following generalization of the result of Aubin and Cantor. The assumption that there is a bound on the sectional curvature is here replaced by the weaker assumption that there is a lower bound for the Ricci curvature.

\begin{Proposition}
    The Sobolev embeddings in their first part are valid for any smooth, complete Riemannian manifold with Ricci curvature bounded from below and positive injectivity radius. In particular, given $(M, g)$ a smooth, complete Riemannian n-manifold with Ricci curvature bounded from below and positive injectivity radius, and for any $q \in[1, n)$ real, $H_1^q(M) \subset L^p(M)$ where $1 / p=1 / q-1 / n$
    \label{prp44}
\end{Proposition}

Let us now state and prove the more general result of Varopoulos mentioned above. The original proof of this result was based on rather intricate semigroup techniques. The proof we present here is somehow more natural. 

\begin{Theorem}
Let ${(M, g)}$ be a smooth, complete Riemannian n-manifold with Ricci curvature bounded from below. Assume that
$$
\inf_{x \in M} \operatorname{Vol}_g\left(B_x(1)\right)>0
$$
where $\operatorname{Vol}_g\left(B_x(1)\right)$ stands for the volume of ${B_x(1)}$ with respect to ${g}$. \textbf{Then the Sobolev embeddings in their first part are valid for $\bm{(M, g)}$.} In particular, for any ${q \in[1, n)}$ real, ${H_1^q(M) \subset L^p(M)}$ where ${1 / p=1 / q-1 / n}$.
\label{thm42}
\end{Theorem}

As a remark on the statement of \autoref{thm42}, note that the assumption
$$
\inf _{x \in M} \operatorname{Vol}_g\left(B_x(1)\right)>0
$$
implies that for any $r>0$, there exists $v_r>0$ such that for any $x \in M$, $\operatorname{Vol}_g\left(B_x(r)\right) \geq v_r$. Such a claim is a straightforward consequence of Gromov's result, \autoref{thm21}. Now, the proof of \autoref{thm42} proceeds in several steps. As a starting point, we prove the following:

\begin{Lemma}
Let $(M, g)$ be a smooth, complete Riemannian $n$-manifold such that its Ricci curvature satisfies $\operatorname{Rc}_{(M, g)} \geq k g$ for some $k \in \mathbb{R}$. Let also $R>0$ be some positive real number. There exists a positive constant $C=C(n, k, R)$, depending only on $n, k$, and $R$, such that for any $r \in(0, R)$, and any $u \in \mathcal{D}(M)$,
$$
\int_M\left|u-\bar{u}_r\right| d v(g) \leq C r \int_M|\nabla u| d v(g)
$$
where $\bar{u}_r(x)=\frac{1}{\operatorname{Vol}_g\left(B_x(r)\right)} \int_{B_x(r)} u d v(g), x \in M$.
\label{lem41}
\end{Lemma}

\begin{Proof}
Let $(M, g)$ be a smooth, complete Riemannian $n$-manifold such that $\operatorname{Rc}_{(M . g)} \geq k g$ for some $k \in \mathbb{R}$, and let $R>0$. By the work of Buser \cite{Buser}, there exists a positive constant $C=C(n, k, R)$, depending only on $n, k$, and $R$, such that for any $x \in M$, any $r \in(0,2 R)$, and any $u \in C^{\infty}\left(B_x(r)\right)$,
\begin{equation}
\int_{B_x(r)}\left|u-\bar{u}_r(x)\right| d v(g) \leq C r \int_{B_{x}(r)}|\nabla u| d v(g)
\label{eq9}
\end{equation}
Let $r \in(0, R)$ be given and let $\left(x_i\right)_{i \in I}$ be a sequence of points of $M$ such that simultaneously
$$
M=\bigcup_i B_{x_i}(r) \quad \text { and } \quad B_{x_i}\left(\frac{r}{2}\right) \cap B_{x_j}\left(\frac{r}{2}\right)=\emptyset \quad \text { if } i \neq j
$$
With the same arguments as used in the proof of \autoref{lem21}, one gets that
$$
\operatorname{Card}\left\{i \in I \,,\, x \in B_{x_i}(2 r)\right\} \leq N=N(n, k, R)=(16)^n e^{8 \sqrt{(n-1)|k|} R}
$$
where Card stands for the cardinality. Let $u \in \mathcal{D}(M)$. We have
$$
\begin{aligned}
\int_M\left|u-\bar{u}_r\right| d v(g) &\leq  \sum_i \int_{B_{x_i}(r)}\left|u-\bar{u}_r\right| d v(g) \\
&\leq  \sum_i \int_{B_{x_i}(r)}\left|u-\bar{u}_r\left(x_i\right)\right| d v(g) \\
& +\sum_i \int_{B_{x_i}(r)}\left|\bar{u}_r\left(x_i\right)-\bar{u}_{2 r}\left(x_i\right)\right| d v(g) \\
& +\sum_i \int_{B_{x_i}(r)}\left|\bar{u}_r-\bar{u}_{2 r}\left(x_i\right)\right| d v(g)
\end{aligned}
$$
By \autoref{eq9}, we get that
$$
\begin{aligned}
\sum_i \int_{B_{x_i}(r)}\left|u-\bar{u}_r\left(x_i\right)\right| d v(g) & \leq C r \sum_i \int_{B_{x_i}(r)}|\nabla u| d v(g) \\
& \leq N C r \int_M|\nabla u| d v(g)
\end{aligned}
$$
while
$$
\begin{aligned}
\sum_i \int_{B_{x_i}(r)}\left|\bar{u}_r\left(x_i\right)-\bar{u}_{2 r}\left(x_i\right)\right| d v(g) & =\sum_i \operatorname{Vol}_g\left(B_{x_i}(r)\right)\left|\bar{u}_r\left(x_i\right)-\bar{u}_{2 r}\left(x_i\right)\right| \\
& \leq \sum_i \int_{B_{x_i}(r)}\left|u-\bar{u}_{2 r}\left(x_i\right)\right| d v(g) \\
& \leq \sum_i \int_{B_{x_i}(2 r)}\left|u-\bar{u}_{2 r}\left(x_i\right)\right| d v(g) \\
& \leq 2 N C r \int_M|\nabla u| d v(g)
\end{aligned}
$$
Independently, we have
$$
\begin{aligned}
&\quad\,\sum_i \int_{B_{x_i}(r)}\left|\bar{u}_r-\bar{u}_{2 r}\left(x_i\right)\right| d v(g) \\
&\leq \sum_i \int_{x \in B_{x_i}(r)}\left\{\frac{1}{\operatorname{Vol}_g\left(B_x(r)\right)} \int_{y \in B_x(r)}\left|u(y)-\bar{u}_{2 r}\left(x_i\right)\right| d v_g(y)\right\} d v_g(x) \\
&\leq \sum_i \int_{x \in B_{x_i}(r)}\left\{\frac{1}{\operatorname{Vol}_g\left(B_x(r)\right)} \int_{y \in B_{x_i}(2 r)}\left|u(y)-\bar{u}_{2 r}\left(x_i\right)\right| d v_g(y)\right\} d v_g(x) \\
&\leq \sum_i \int_{B_{x_i}(2 r)}\left|u(y)-\bar{u}_{2 r}\left(x_i\right)\right| d v_g(y) \int_{B_{x_i}(r)} \frac{1}{\operatorname{Vol}_g\left(B_x(r)\right)} d v_g(x)\\
\end{aligned}
$$
But, by \autoref{eq9},
$$
\int_{B_{x_i}(2 r)}\left|u(y)-\bar{u}_{2 r}\left(x_i\right)\right| d v_g(y) \leq 2 C r \int_{B_{x_i(2 r)}}|\nabla u| d v(g)
$$
while by Gromov's result,
$$
\frac{1}{\operatorname{Vol}_g\left(B_x(r)\right)} \leq \frac{K}{\operatorname{Vol}_g\left(B_x(2 r)\right)}
$$
where $K=K(n, k, R)=2^n e^{2 \sqrt{(n-1)|k| }R}$. Since $x \in B_{x_i}(r)$ implies that $B_{x_i}(r)$ is a subset of $B_x(2 r)$, we get that
$$
\int_{B_{x_i}(r)} \frac{1}{\operatorname{Vol}_g\left(B_x(r)\right)} d v_g(x) \leq K
$$
Hence,
$$
\sum_i \int_{B_{x_i}(r)}\left|\bar{u}_r-\bar{u}_{2 r}\left(x_i\right)\right| d v(g) \leq 2 K C N r \int_M|\nabla u| d v(g)
$$
and for any $u \in \mathcal{D}(M)$,
$$
\int_M\left|u-\bar{u}_r\right| d v(g) \leq 3(1+K) N C r \int_M|\nabla u| d v(g)
$$
This ends the proof of the lemma.
\end{Proof}

We now prove the following lemma: 

\begin{Lemma}
    Let $(M, g)$ be a smooth, complete Riemannian n-manifold. Suppose that its Ricci curvature satisfies $\operatorname{Rc}_{(M . g)} \geq k g$ for some $k \in \mathbb{R}$, and suppose that there exists $v>0$ such that $\operatorname{Vol}_g\left(B_x(1)\right) \geq v$ for any $x \in M$. There exist two positive constants $C=C(n, k, v)$ and $\eta=\eta(n, k, v)$, depending only on $n, k$, and $v$, such that for any open subset $\Omega$ of $M$ with smooth boundary and compact closure, if $\operatorname{Vol}_g(\Omega) \leq \eta$, then $\operatorname{Vol}_g(\Omega)^{(n-1) / n} \leq C \operatorname{Area}_g(\partial \Omega)$.
\label{lem42}
\end{Lemma}

\begin{Proof}
    By \autoref{thm21} and the remark following this theorem, we have that for any $x \in M$ and any $0<r<R$,
$$
\operatorname{Vol}_g\left(B_x(r)\right) \geq\left(\frac{1}{R^n} e^{-\sqrt{(n-1)|k|} R} \operatorname{Vol}_g\left(B_x(R)\right)\right) r^{n}
$$
Fix $R=1$. Then we get that for any $x \in M$ and any $r \in(0,1)$,
$$
\operatorname{Vol}_g\left(B_x(r)\right) \geq\left(e^{-\sqrt{(n-i)|k|})} v\right) r^n
$$
Set
$$
\eta=\frac{1}{16} e^{-\sqrt{(n-1)|k|}} v \quad \text { and } \quad C_1=e^{-\sqrt{(n-1)|k|}} v
$$
Let $\Omega$ be some open subset of $M$ with smooth boundary, compact closure, and such that $\operatorname{Vol}_g(\Omega) \leq \eta$. For sufficiently small $\varepsilon>0$, consider the function
$$
u_{\varepsilon}(x)=\left\{\begin{array}{ll}
1 & \text { if } x \in \Omega \\
1-\frac{1}{\varepsilon} d_g(x, \partial \Omega) & \text { if } x \in M \backslash \Omega \text { and } d_g(x, \partial \Omega) \leq \varepsilon \\
0 & \text { if } x \in M \backslash \Omega \text { and } d_g(x, \partial \Omega) \geq \varepsilon
\end{array}\right.
$$
Then $u_{\varepsilon}$ is Lipschitz for every $\varepsilon$ and one easily sees that
$$
\lim _{\varepsilon \rightarrow 0} \int_M u_{\varepsilon} d v(g)=\operatorname{Vol}_g(\Omega)
$$
while
$$
\left|\nabla u_{\varepsilon}\right|=\left\{\begin{array}{ll}
\frac{1}{\varepsilon} & \text { if } x \in M \backslash \bar{\Omega} \text { and } d_g(x, \partial \Omega)<\varepsilon \\
0 & \text { otherwise }
\end{array}\right.
$$
which implies that
$$
\lim _{\varepsilon \rightarrow 0} \int_M\left|\nabla u_{\varepsilon}\right| d v(g)=\operatorname{Area}_g(\partial \Omega)
$$
Furthermore, for every $\varepsilon>0$,
$$
\operatorname{Vol}_g(\Omega)=\operatorname{Vol}_g\left(\left\{x \in M \,,\, u_{\varepsilon}(x) \geq 1\right\}\right)
$$
and for any $\varepsilon>0$ and any $r>0$,
$$
\begin{aligned}
\operatorname{Vol}_g\left(\left\{x \in M \,,\, u_{\varepsilon}(x) \geq 1\right\}\right) &\leq  \operatorname{Vol}_g\left(\left\{x \in M \,,\,\left|u_{\varepsilon}(x)-\bar{u}_{\varepsilon, r}(x)\right| \geq \frac{1}{2}\right\}\right) \\
& +\operatorname{Vol}_g\left(\left\{x \in M \,,\, \bar{u}_{\varepsilon, r}(x) \geq \frac{1}{2}\right\}\right)
\end{aligned}
$$
where
$$
\bar{u}_{\varepsilon, r}(x)=\frac{1}{\operatorname{Vol}_g\left(B_x(r)\right)} \int_{B_x(r)} u_{\varepsilon} d v(g)
$$
Now note that for $r>0$ and $\varepsilon \ll 1$,
$$
\bar{u}_{\varepsilon, r}(x) \leq \frac{2 \operatorname{Vol}_g(\Omega)}{\operatorname{Vol}_g\left(B_x(r)\right)}
$$
Fix $r=\left(\frac{8 \operatorname{Vol}_g(\Omega)}{C_1}\right)^{1 / n}$. Since $\operatorname{Vol}_g(\Omega) \leq \eta=\frac{c_1}{16}$, we get that $r \in(0,1)$ and that
$$
\frac{2 \operatorname{Vol}_g(\Omega)}{\operatorname{Vol}_g\left(B_x(r)\right)} \leq \frac{1}{4}
$$
(according to what we have said above). Hence,
$$
\left\{x \in M \,,\, \bar{u}_{\varepsilon, r}(x) \geq \frac{1}{2}\right\}=\emptyset
$$
and for every $0<\varepsilon \ll 1$,
$$
\operatorname{Vol}_g(\Omega) \leq \operatorname{Vol}_g\left(\left\{x \in M \,,\,\left|u_{\varepsilon}(x)-\bar{u}_{\varepsilon, r}(x)\right| \geq \frac{1}{2}\right\}\right)
$$
But
$$
\operatorname{Vol}_g\left(\left\{x \in M \,,\,\left|u_{\varepsilon}(x)-\bar{u}_{\varepsilon, r}(x)\right| \geq \frac{1}{2}\right\}\right) \leq 2 \int_M\left|u_{\varepsilon}-\bar{u}_{\varepsilon, r}\right| d v(g)
$$
and by \autoref{lem41} there exists a positive constant $C_2=C_2(n, k)$ such that
$$
\int_M\left|u_{\varepsilon}-\bar{u}_{\varepsilon, r}\right| d v(g) \leq C_2 r \int_M\left|\nabla u_{\varepsilon}\right| d v(g)
$$
Hence,
$$
\begin{aligned}
\operatorname{Vol}_g(\Omega) & \leq 2 C_2\left(\frac{8 \operatorname{Vol}_g(\Omega)}{C_1}\right)^{1 / n} \lim _{\varepsilon \rightarrow 0} \int_M\left|\nabla u_{\varepsilon}\right| d v(g) \\
& \leq C_3 \operatorname{Vol}_g(\Omega)^{1 / n} \operatorname{Area}_g(\partial \Omega)
\end{aligned}
$$
where $C_3$ depends only on $n, k$, and $v$. Clearly, this ends the proof of the lemma.
\end{Proof}

\vskip 3pt
\autoref{lem42} has the following consequence: The ideas used in the proof of \autoref{lem43} are by now standard

\begin{Lemma}
Let $(M, g)$ be a smooth, complete Riemannian n-manifold. Suppose that its Ricci curvature satisfies $\operatorname{Rc}_{(M . g)} \geq k g$ for some $k \in \mathbb{R}$ and suppose that there exists $v>0$ such that $\operatorname{Vol}_g\left(B_x(1)\right) \geq v$ for any $x \in M$. There exist two positive constants $\delta=\delta(n, k, v)$ and $A=A(n, k, v)$, depending only on $n, k$, and $v$, such that
$$
\left(\int_M|u|^{n /(n-1)} d v(g)\right)^{(n-1) / n} \leq A \int_M|\nabla u| d v(g)
$$
for any $x \in M$ and any $u \in \mathcal{D}\left(B_x(\delta)\right)$.
\label{lem43}
\end{Lemma}

\begin{Proof}
Let $\eta=\eta(n, k, v)$ be as in \autoref{lem42}. By \autoref{thm21} there exists $\delta=\delta(n, k, v)$ such that for any $x \in M, \operatorname{Vol}_g\left(B_x(\delta)\right) \leq \eta$. Let $x \in M$ and let $u \in \mathcal{D}\left(B_{x}(\delta)\right)$. For $t \geq 0$, let
$$
\Omega(t)=\{x \in M \,,\,|u(x)|>t\}\quad \text { and } \quad V(t)=\operatorname{Vol}_g(\Omega(t))
$$
Clearly, $V(t) \leq \eta$ for any $t \geq 0$. Then the co-area formula and \autoref{lem42} imply that
$$
\int_M|\nabla u| d v(g) \geq \frac{1}{C} \int_0^{\infty} V(t)^{1-1 / n} d t
$$
where $C$ is the constant given by \autoref{lem42}. Independently,
$$
\int_M|u|^{n /(n-1)} d v(g)=\frac{n}{n-1} \int_0^{\infty} t^{1 /(n-1)} V(t) d t
$$
Noting that
$$
\int_0^{\infty} V(t)^{1-1 / n} d t \geq\left(\frac{n}{n-1} \int_0^{\infty} t^{1 /(n-1)} V(t) d t\right)^{1-1 / n}
$$
we end the proof of the lemma.
\end{Proof}

\vskip 3pt
With \autoref{lem43} we are now in position to prove \autoref{thm42}.

\vskip 3pt
\begin{Proof2}
Let $(M, g)$ be a smooth, complete Riemannian $n$ manifold such that $\operatorname{Rc}_{(M, g)} \geq k g$ for some $k \in \mathbb{R}$ and such that there exists $v>0$ with the property that $\operatorname{Vol}_g\left(B_{\mathrm{r}}(1)\right) \geq v$ for any $x \in M$. We want to prove that the Sobolev embeddings are valid on $M$. By \autoref{lem31} we just have to prove that $H_1^{1}(M) \subset L^{n /(n-1)}(M)$. Let $\delta=\delta(n, k, v)$ be as in \autoref{lem43} and let $\left(x_i\right)$ be a sequence of points of $M$ such that

1. $M=\bigcup_i B_{x_i}\left(\frac{\delta}{2}\right)$

2. $B_{x_i}\left(\frac{\delta}{4}\right) \cap B_{x_j}\left(\frac{\delta}{4}\right)=\emptyset$ if $i \neq j$, and

3. there exists $N=N(n, k, v)$ depending only on $n, k$, and $v$, such that each point 

$\quad\;$of $M$ has a neighborhood that intersects at most $N$ of the $B_{x_i}(\delta)$ 's.

\vskip 3pt
\noindent
The existence of such a sequence is given by \autoref{lem21}. Let also
$$
\rho:[0, \infty) \rightarrow[0,1]
$$
be defined by
$$
\rho(t)=\left\{\begin{array}{ll}
1 & \text { if } 0 \leq t \leq \frac{\delta}{2} \\
3-\frac{4}{\delta} t & \text { if } \frac{\delta}{2} \leq t \leq \frac{3 \delta}{4} \\
0 & \text { if } t \geq \frac{3 \delta}{4}
\end{array}\right.
$$
and let
$$
\alpha_i(x)=\rho\left(d_g\left(x_i, x\right)\right)
$$
where $d_g$ denotes the distance associated to $g$ and $x \in M$. Clearly, $\alpha_i$ is Lipschitz with compact support. Hence, by \autoref{prp34}, $\alpha_i$ belongs to $H_1^1(M)$. Furthermore, since $\operatorname{supp} \alpha_i \subset B_{x_i}\left(\frac{3 \delta}{4}\right)$, we get without any difficulty that $\alpha_i \in$ $H_{0.1}^1\left(B_{x_i}(\delta)\right)$. Let
$$
\eta_i=\frac{\alpha_i}{\sum_m \alpha_m}
$$
Then, since $\left|\nabla \alpha_i\right| \leq 4 / \delta$ a.e., we get by (3) that for any $i, \eta_i \in H_{0.1}^1\left(B_{x_i}(\delta)\right)$, $\left(\eta_i\right)$ is a partition of unity subordinate to the covering ($\left.B_{x_i}(\delta)\right), \nabla \eta_i$ exists almost everywhere, and there exists a positive constant $H=H(n, k, v)$ such that $\left|\nabla \eta_i\right| \leq$ $H$ a.e. Let $u \in \mathcal{D}(M)$. We have
$$
\begin{aligned}
\left(\int_M|u|^{n /(n-1)} d v(g)\right)^{(n-1) / n} & \leq \sum_i\left(\int_M\left|\eta_i u\right|^{n /(n-1)} d v(g)\right)^{(n-1) / n} \\
& \leq A \sum_i \int_M\left|\nabla\left(\eta_i u\right)\right| d v(g)
\end{aligned}
$$
where $A$ is the constant of \autoref{lem43}. Hence,
$$
\begin{aligned}
\left(\int_M|u|^{n /(n-1)} d v(g)\right)^{(n-1) / n} &\leq A \sum_i \int_M \eta_i|\nabla u| d v(g)+A \sum_i \int_M|u|\left|\nabla \eta_i\right| d v(g) \\
&\leq A \int_M|\nabla u| d v(g)+A N H \int_M|u| d v(g) \\
&\leq A(1+N H)\left(\int_M|\nabla u| d v(g)+\int_M|u| d v(g)\right)
\end{aligned}
$$
and there exists $\tilde{A}>0$ such that for any $u \in \mathcal{D}(M)$,
$$
\left(\int_M|u|^{n /(n-1)} d v(g)\right)^{(n-1) / n} \leq \tilde{A}\left(\int_M|\nabla u| d v(g)+\int_M|u| d v(g)\right)
$$
By \autoref{thm41} we then get that $H_1^1(M) \subset L^{n /(n-1)}(M)$. As already mentioned, this ends the proof of the theorem.
\end{Proof2}

\vskip 3pt
Given $(M, g)$ a smooth, complete Riemannian $n$-manifold, we refer to the scale of Sobolev embeddings when considering the embeddings $H_1^q(M) \subset L^p(M), q \in$ $[1, n), 1 / p=1 / q-1 / n$. As already mentioned in Section 3.3, the validity of one of these embeddings implies the validity of the ones after: if
$H_1^{q_0}(M) \subset L^{p_0}(M)$ for some $q_0 \in[1, n)$ and $1 / p_0=1 / q_0-1 / n$, then $H_1^q(M) \subset$ $L^p(M)$ for any $q \in\left[q_0, n\right)$ and $1 / p=1 / q-1 / n$. A natural question is to know if such a scale is coherent, that is, if the validity of one of these embeddings implies the validity of all the other ones. 
In other words, if the validity of one of the embeddings $H_1^q(M) \subset L^p(M), q \in[1, n)$, implies the validity of the embedding $H_1^1(M) \subset L^{n /(n-1)}(M)$. Combining \autoref{thm42} and \autoref{lem32}, one gets that the scale of Sobolev embeddings is coherent for complete manifolds with Ricci curvature bounded from below. More precisely, one has the following:

\begin{Theorem}
Let $(M, g)$ be a smooth, complete Riemannian n-manifold with Ricci curvature bounded from below.

(i) Suppose that for some $q_0 \in[1, n), H_1^{q_0}(M) \subset L^{p_0}(M)$ where $1 / p_0=$ $1 / q_0-1 / n$. 

$\quad\;$Then for any $q \in[1, n), H_1^q(M) \subset L^p(M)$ where $1 / p=$ $1 / q-1 / n$. In particular, 

$\quad\;$one has that $H_1^{1}(M) \subset L^{n /(n-1)}(M)$.

(ii) Given $q \in[1, n)$, one has that $H_1^q(M) \subset L^p(M)$, where $1 / p=1 / q-1 / n$ if and 

$\quad\;$only if there exists a lower bound for the volume of small balls which is uniform 

$\quad\;$with respect to their center.
\label{thm43}
\end{Theorem}

Point (ii) in such a theorem means that for any $r>0$ there exists $v_r>0$ such that for any $x \in M, \operatorname{Vol}_g\left(B_x(r)\right) \geq v_r$. By Gromov's result, \autoref{thm21}, since the manifolds considered have their Ricci curvature bounded from below, it is sufficient to have such a lower bound for one $r_0>0$. Independently, let $(M, g)$ be a smooth, complete Riemannian $n$-manifold satisfying the assumptions of \autoref{thm42}. Namely, its Ricci curvature satisfies that $\operatorname{Rc}_{(M . g)} \geq k g$ for some $k \in \mathbb{R}$, and there exists $v>0$ such that for any $x \in M, \operatorname{Vol}_g\left(B_x(1)\right) \geq v$. By \autoref{thm42}, for any $q \in[1, n)$, there exists $A>0$ such that for any $u \in H_1^q(M)$,
$$
\left(\int_M|u|^p d v(g)\right)^{1 / p} \leq A\left(\left(\int_M|\nabla u|^q d v(g)\right)^{1 / q}+\left(\int_M|u|^q d v(g)\right)^{1 / q}\right)
$$
Note here that the proof of \autoref{thm42} gives the exact dependence of $A$: it depends only on $n, q, k$, and $v$. Finally, we have seen in Section 3 that for compact $n$-manifolds, $H_1^q \subset L^p$ for any $q \in[1, n)$ and any $p \geq 1$ such that $p \leq n q /(n-q)$, that is, for any $p$ such that $1 / p \geq 1 / q-1 / n$. One can ask here if such a result still holds for complete manifolds. As a first remark, one can note that for complete, noncompact manifolds, one must have that $p \geq q$. Indeed, given $\left(\mathbb{R}^n, e\right)$ the Euclidean space, let $u_\alpha \in C^{\infty}\left(\mathbb{R}^n\right)$ be some smooth function such that $u_\alpha(x)=1 /|x|^\alpha$ if $|x| \geq 1$. As one can easily check, for $p \in[1, q), u_{n / p} \in H_1^q\left(\mathbb{R}^n\right)$ while $u_{n / p} \notin L^p\left(\mathbb{R}^n\right)$. This proves the above claim. On the contrary, one can prove that the embeddings $H_1^q \subset L^p$ do hold for complete $n$-manifolds as soon as $p \geq q$. This is the subject of the following result:

\begin{Proposition}
Let $(M, g)$ be a smooth, complete $n$-dimensional Riemannian manifold such that its Ricci curvature is bounded from below and such that there exists $v>0$ with the property that for any $x \in M, \operatorname{Vol}_g\left(B_x(1)\right) \geq v$. For any $q \in[1, n)$ real and any $p \in[q,$ $n q /(n-q)], H_1^q(M) \subset L^p(M)$.
\label{prp45}
\end{Proposition}

\begin{Proof}
Set $q^*=n q /(n-q)$, and let $p \in\left[q, q^*\right]$. As a simple application of Hölder's inequality, one gets that for any $u \in \mathcal{D}(M)$,
$$
\left(\int_M|u|^p d v(g)\right)^{1 / p} \leq\left(\int_M|u|^q d v(g)\right)^{\alpha / q}\left(\int_M|u|^{q^*} d v(g)\right)^{(1-\alpha) / q^*}
$$
where $\alpha \in[0,1]$ is given by
$$
\alpha=\frac{1 / p-1 / q^*}{1 / q-1 / q^{\star}}
$$
By \autoref{thm42}, there exists $A=A(n, q, k, v)$ such that for any $u \in \mathcal{D}(M)$,
$$
\left(\int_M|u|^{q^*} d v(g)\right)^{1 / q^*} \leq A\left(\int_M|\nabla u|^q d v(g)\right)^{1 / q}+A\left(\int_M|u|^q d v(g)\right)^{1 / q}
$$
Since for any $x$ and $y$ nonnegative, and any $\alpha \in[0,1], x^\alpha y^{1-\alpha} \leq x+y$, one gets that for any $p \in\left[q, q^{\star}\right]$, and any $u \in \mathcal{D}(M)$,
$$
\begin{aligned}
\quad \left(\int_M|u|^p d v(g)\right)^{1 / p} &\leq\left(\int_M|u|^q d v(g)\right)^{1 / q}+\left(\int_M|u|^{q^*} d v(g)\right)^{1 / q^*} \\
&\leq A\left(\int_M|\nabla u|^q d v(g)\right)^{1 / q}+(A+1)\left(\int_M|u|^q d v(g)\right)^{1 / q}\\
\end{aligned}
$$
Clearly, this proves the proposition.
\end{Proof}

\vskip 3pt
In the end of this part, let us now make some remarks. 
As a first remark, note that \textbf{the assumption we made till now on the Ricci curvature is satisfactory but certainly not necessary.} Indeed, there exist complete manifolds for which the whole scale of Sobolev embeddings $H_1^q \subset L^p$ is valid, but for which the Ricci curvature is not bounded from below. Just consider the space $\mathbb{R}^n$ with a conformal metric $g=e^u e$ to the Euclidean metric $e$, the conformal factor $u$ being bounded and chosen such that the Ricci curvature of $g$ is not bounded from below. With such a choice, one gets examples of the kind mentioned above. 
As a second remark, recall that we have seen in Section 3 that for compact manifolds, the embeddings $H_1^q \subset L^p$ with $p<n q /(n-q)$ are compact. \textbf{One can ask here if such a property still holds for complete manifolds. The answer is negative.} Just think to $\mathbb{R}^n$ with its Euclidean metric $e$, and let $u \in \mathcal{D}\left(\mathbb{R}^n\right)$ be such that $0 \leq u \leq 1, u=1$ in $B_0(1)$, and $u=0$ in $\mathbb{R}^n \backslash B_0(2)$. For $m$ an integer, set $u_m(x)=u\left(x-x_m\right)$ where $x_m \in \mathbb{R}^n$ is such that $\left|x_m\right|=m$. 
Clearly, $\left(u_m\right)$ is bounded in $H_1^q\left(\mathbb{R}^n\right)$ by $\|u\|_{H_1^q}$, while for any $m$, $\left\|u_m\right\|_p=\|u\|_p>0$. Since $\left(u_m\right.$) converges to 0 for the pointwise convergence, one gets as a consequence of what has been said that ($u_m$) does not converge in $L^p(M)$. This proves the above claim. 

\subsubsection{When 1/q-(k-m)/n<0}

We briefly discuss in this section the validity of Sobolev embeddings in their second part for complete manifolds. Recall that by Sobolev embeddings in their second part, we refer to embeddings such as $H_h^q \subset C^m$. In the 1970s, Aubin and Cantor proved that such embeddings were valid for complete manifolds with bounded sectional curvature and positive injectivity radius. We prove here that the result still holds under the weaker assumption that the Ricci curvature is bounded from below and that the injectivity radius is positive. Extensions will be discussed at the end of the section. Given $(M, g)$ a smooth, complete manifold and $m$ an integer, we denote by $C_B^m(M)$ the space of functions $u: M \rightarrow \mathbb{R}$ of class $C^m$ for which the norm
$$
\|u\|_{C^m}=\sum_{j=0}^m \sup _{x \in M}\left|\left(\nabla^j u\right)(x)\right|
$$
is finite. In the same order of ideas, given $\lambda \in(0,1)$, we denote by $C_B^\lambda(M)$ the space of continuous functions $u: M \rightarrow \mathbb{R}$ for which the norm
$$
\|u\|_{C^\lambda}=\sup _{x \in M}|u(x)|+\sup _{x \neq y \in M} \frac{|u(y)-u(x)|}{d_g(x, y)^\lambda}
$$
is finite, where $d_g$ denotes the distance associated to $g$. The first result we prove is the following:

\begin{Theorem}
    Let $(M, g)$ be a smooth, complete Riemannian n-manifold with Ricci curvature bounded from below and positive injectivity radius. For $q \geq 1$ real and $m<k$ two integers, \textbf{if $\bm{1 / q<(k-m) / n}$, then $\bm{H_k^q(M) \subset C_B^m(M)}$.}
\label{thm44}
\end{Theorem}

\begin{Proof}
First we prove that for $q>n, H_1^q(M) \subset C_B^0(M)$. By \autoref{thm22}, one has that for any $Q>1$ and $\alpha \in(0,1)$, the $C^{0 . \alpha}$-harmonic radius $r_H=$ $r_H(Q, 0, \alpha)$ is positive. Fix, for instance, $Q=2$ and $\alpha=1 / 2$. For any $x \in M$ one then has that there exists some harmonic chart $\varphi_x: B_x\left(r_H\right) \rightarrow \mathbb{R}^n$ such that the components $g_{i j}$ of $g$ in this chart satisfy
$$
\frac{1}{2} \delta_{i j} \leq g_{i j} \leq 2 \delta_{i j}
$$
as bilinear forms. Let $\left(x_i\right)$ be a sequence of points of $M$ such that

1. $M=\bigcup_i B_{x_i}\left(\frac{r_H}{2}\right)$ and

2. there exists $N$ such that each point of $M$ has a neighborhood which intersects at 

$\quad\;$most $N$ of the $B_{x_i}\left(r_H\right)$ 's.

\vskip 3pt
\noindent
The existence of such a sequence is given by \autoref{lem21}. Let also
$$
\rho:[0, \infty) \rightarrow[0,1]
$$
be defined by
$$
\rho(t)=\left\{\begin{array}{ll}
1 & \text { if } 0 \leq t \leq \frac{r_H}{2} \\
3-\frac{4}{r_H} t & \text { if } \frac{r_H}{2} \leq t \leq \frac{3 r_H}{4} \\
0 & \text { if } t \geq \frac{3 r_H}{4}
\end{array}\right.
$$
and let
$$
\alpha_i(x)=\rho\left(d_g\left(x_i, x\right)\right)
$$
where $d_g$ denotes the distance associated to $g$ and $x \in M$. Clearly, $\alpha_i$ is Lipschitz and bounded, with compact support in $B_{x_i}\left(r_h\right)$. Set
$$
\eta_i=\frac{\alpha_i^{[q]+1}}{\sum_m \alpha_m^{[q]+1}}
$$
where $[q]$ is the greatest integer not exceeding $q$. As one can easily check, $\eta_i$ and $\eta_i^{1 / q}$ are also Lipschitz with compact support in $B_{x_i}\left(r_H\right)$. In particular, one gets by \autoref{prp34} that $\eta_i^{1 / q} \in H_{0.1}^q\left(B_{x_i}\left(r_H\right)\right)$. Moreover, one has that $\left(\eta_i\right)$ is a partition of unity subordinate to the covering $\left(B_{x_i}\left(r_H\right)\right.$ ), that $\nabla \eta_i^{1 / q}$ exists almost everywhere, and that there exists a positive constant $H$ such that for all $i,\left|\nabla \eta_i^{1 / q}\right| \leq$ $H$ a.e. Given $u \in \mathcal{D}(M)$, one clearly has that
$$
\left\|\eta_i^{1 / q} u\right\|_{C^0}=\left\|\left(\eta_i^{1 / q} u\right) \circ \varphi_{x_i}^{-1}\right\|_{C^0}
$$
for all $i$. Independently, starting from the inequalities satisfied by the $g_{i j}$ 's, one easily gets that there exists $C>0$ such that for any $i$ and any $u \in \mathcal{D}(M)$,
$$
\left\|\left(\eta_i^{1 / q} u\right) \circ \varphi_{x_i}^{-1}\right\|_{H_1^q} \leq C\left\|\eta_i^{1 / q} u\right\|_{H_1^q}
$$
where the norm in the left-hand side of this inequality is with respect to the Euclidean metric. Since $H_1^q\left(\mathbb{R}^n\right) \subset C_B^0\left(\mathbb{R}^n\right)$, this leads to the existence of some $A>0$ such that for any $i$ and any $u \in \mathcal{D}(M)$,
$$
\left\|\eta_i^{1 / q} u\right\|_{C^0} \leq A\left\|\eta_i^{1 / q} u\right\|_{H_1^q}
$$
Given $u \in \mathcal{D}(M)$ one can write that
$$
\|u\|_{C^0}^q=\left\|\sum_i \eta_i|u|^q\right\|_{C^0} \leq \sum_i\left\|\eta_i|u|^q\right\|_{C^0}  =\sum_i\left\|\eta_i^{1 / q} u\right\|_{C^0}^q \leq A^q \sum_i\left\|\eta_i^{1 / q} u\right\|_{H_i^q}^q
$$
Let $\mu=\mu(q)$ be such that for $x \geq 0$ and $y \geq 0,(x+y)^q \leq \mu\left(x^q+y^q\right)$. Then, for $u \in \mathcal{D}(M)$,
$$
\|u\|_{C^0}^q \leq A^q \mu \sum_i\left(\int_M\left|\nabla\left(\eta_i^{1 / q} u\right)\right|^q d v(g)+\int_M \eta_i|u|^q d v(g)\right)
$$
Here, one has that
$$
\begin{aligned}
\sum_i \int_M\left|\nabla\left(\eta_i^{1 / q} u\right)\right|^q d v(g) &\leq \mu \sum_i \int_M\left|\nabla \eta_i^{1 / q}\right|^q|u|^q d v(g)+\mu \sum_i \int_M \eta_i|\nabla u|^q d v(g)\\
&\leq \mu N H^q \int_M|u|^q d v(g)+\mu \int_M|\nabla u|^q d v(g) \\
&\leq \mu\left(N H^q+1\right)\left(\int_M|\nabla u|^q d v(g)+\int_M|u|^q d v(g)\right)\\
\end{aligned}
$$
Hence, there exists $B>0$ such that for any $u \in \mathcal{D}(M)$,
$$
\|u\|_{C^0}^q \leq B\|u\|_{H_1^4}^q
$$
Clearly, by \autoref{thm41}, this proves that $H_1^q(M) \subset C_B^0(M)$. Let us now prove that for $q, k$, and $m$ as in the theorem, $H_k^q(M) \subset C_B^m(M)$. Given $u \in \mathcal{C}_k^q(M)$, one has by Kato's inequality that for any integer $s$,
$$
|\nabla| \nabla^s u|| \leq\left|\nabla^{s+1} u\right|
$$
Let $s \in\{0, \ldots, m\}$. By \autoref{prp44} one has that $H_{k-s}^q(M) \subset H_1^{p_s}(M)$ where
$$
\frac{1}{p_{\mathrm{s}}}=\frac{1}{q}-\frac{k-s-1}{n}
$$
In particular, $p_s>n$. Hence, according to what has been said above, $H_1^{p_s}(M) \subset$ $C^0(M)$. Given $s \in\{0, \ldots, m\}$ and $u \in \mathcal{C}_k^q(M)$ one then gets that
$$
\left\|\nabla^{s} u\right\|_{C^0} \leq C_1(s)\left\|\nabla^s u\right\|_{H_1^{p_s}} \leq C_2(s)\left\|\nabla^s u\right\|_{H_{k-s}^q} \leq C_2(s)\|u\|_{H_k^q}
$$
by Kato's inequality, and where $C_1(s)$ and $C_2(s)$ do not depend on $u$. As an immediate consequence of such inequalities, one gets that $H_k^q(M) \subset C_B^m(M)$ for $q, k$, and $m$ as above. This ends the proof of the theorem.
\end{Proof}

\vskip 3pt
Let us now prove the following result:

\begin{Theorem}
Let $(M, g)$ be a smooth, complete Riemannian $n$-manifold with Ricci curvature bounded from below and positive injectivity radius. For $q \geq 1$ real and $\lambda \in(0,1)$ real, if $1 / q \leq(1-\lambda) / n$, then $H_1^q(M) \subset C_B^\lambda(M)$.
\label{thm45}
\end{Theorem}

\begin{Proof}
Here again, given $Q>1$ and $\alpha \in(0,1)$, one has by \autoref{thm22} that the $C^{0, \alpha}$-harmonic radius $r_H=r_H(Q, 0, \alpha)$ is positive. Fix, for instance, $Q=2$ and $\alpha=1 / 2$. For any $x \in M$ one then has that there exists some harmonic chart $\varphi_x: B_x\left(r_H\right) \rightarrow \mathbb{R}^n$ such that the components $g_{i j}$ of $g$ in this chart satisfy
\begin{equation}
\frac{1}{2} \delta_{i j} \leq g_{i j} \leq 2 \delta_{i j}
\label{eq10}
\end{equation}
as bilinear forms. Let also $r \in\left(0, r_H\right)$ sufficiently small, for instance, $r<r_H / 3$, such that for any $x \in M$, the minimizing geodesic joining two points in $B_x(r)$ lies in $B_x\left(r_H\right)$. We use in what follows that for $\Omega$, a regular, bounded, open subset of $\mathbb{R}^n$, and $q, \lambda$ as in the theorem, $H_1^q(\Omega) \subset C_B^\lambda(\Omega)$. The proof of this assertion can be found in Adams \cite{A}. Given $q$ and $\lambda$ as in the theorem, let $x$ and $y$ be two points of $M$ such that $x \neq y$.

\vskip 3pt
Suppose first that $d_g(x, y) \geq r$. Then for any $u \in \mathcal{D}(M)$,
$$
\frac{|u(y)-u(x)|}{d_g(x, y)^\lambda} \leq \frac{2}{r^\lambda}\|u\|_{C^0}
$$
By \autoref{thm44}, this leads to the existence of $C_1>0$ such that for any $u \in \mathcal{D}(M)$,
$$
\frac{|u(y)-u(x)|}{d_g(x, y)^\lambda} \leq C_1\|u\|_{H_1^q}
$$
Suppose now that $d_g(x, y)<r$. By \autoref{eq10} one easily gets that
$$
\left|\varphi_x(y)-\varphi_x(x)\right| \leq \sqrt{2} d_g(x, y)
$$
Hence,
$$
\frac{|u(y)-u(x)|}{d_g(x, y)^\lambda} \leq 2^{\frac{\lambda}{2}} \frac{\left|\left(u \circ \varphi_x^{-1}\right)\left(\varphi_x(y)\right)-\left(u \circ \varphi_x^{-1}\right)\left(\varphi_x(x)\right)\right|}{\left|\varphi_x(y)-\varphi_x(x)\right|^\lambda}
$$
Similarly, one easily gets from \autoref{eq10} that there exists $C_2>0$ such that for any $u \in \mathcal{D}(M)$,
$$
\begin{aligned}
\int_{\Omega}\left|\nabla\left(u \circ \varphi_x^{-1}\right)\right|^q d x &\leq C_2 \int_{B_x(r)}|\nabla u|^q d v(g) \\
\int_{\Omega}\left|u \circ \varphi_x^{-1}\right|^q d x &\leq C_2 \int_{B_x(r)}|u|^q d v(g)\\
\end{aligned}
$$
where $\Omega=\varphi_x\left(B_x(r)\right)$, and $d x$ stands for the Euclidean volume element. Since $H_1^q(\Omega) \subset C_B^\lambda(\Omega)$, such inequalities lead to the existence of $C_3>0$ such that for any $u \in \mathcal{D}(M)$,
$$
\frac{|u(y)-u(x)|}{d_g(x, y)^\lambda} \leq C_3\|u\|_{H_1^q}
$$
Take $C_4=\max \left(C_1, C_3\right)$. Then, for any $x$ and $y$ in $M$, with the property that $x \neq y$, and for any $u \in \mathcal{D}(M)$,
$$
\frac{|u(y)-u(x)|}{d_g(x, y)^\lambda} \leq C_4\|u\|_{H_1^q}
$$
Such an inequality, combined with \autoref{thm44}, leads to the existence of $C_5>0$ such that for any $u \in \mathcal{D}(M)$,
$$
\|u\|_{C^\lambda} \leq C_5\|u\|_{H_1^q}
$$
By \autoref{thm41}, one then gets that $H_1^q(M) \subset C_B^\lambda(M)$. This ends the proof.
\end{Proof}

\vskip 3pt
\autoref{thm45} has been generalized by Coulhon \cite{Coulhon} in the spirit of what has been said in the preceding section. More precisely, it is proved in \cite{Coulhon} that for $(M, g)$ a smooth, complete Riemannian $n$-manifold, for $q \geq 1$ real and $\lambda \in(0,1)$ real, if $1 / q \leq(1-\lambda) / n$, then the embedding of $H_1^q(M)$ in $C_B^\lambda(M)$ does hold as soon as the Ricci curvature of $(M, g)$ is bounded from below and that for any $r_0>0$, there exists $C\left(r_0\right)>1$, such that for any $x \in M$ and any $r \in\left(0, r_0\right)$,
$$
C\left(r_0\right)^{-1} r^n \leq \operatorname{Vol}_g\left(B_x(r)\right) \leq C\left(r_0\right) r^n
$$
Under the assumption that
$$
\inf _{x \in M} \operatorname{Vol}_g\left(B_x(1)\right)>0
$$
this last property is an easy consequence of Gromov's theorem, \autoref{thm21}. One then gets the following generalization of \autoref{thm45}. The proof can be found in Coulhon \cite{Coulhon}.

\begin{Theorem}
    Let $(M, g)$ be a smooth, complete Riemannian $n$-manifold with Ricci curvature bounded from below. Assume that
$$
\inf _{\mathrm{r} \in \mathrm{M}} \operatorname{Vol}_g\left(B_{x}(1)\right)>0
$$
where $\operatorname{Vol}_g\left(B_x(1)\right)$ stands for the volume of $B_x(1)$ with respect to $g$. For $q \geq 1$ real and $\lambda \in(0,1)$ real, if ${1 / q \leq(1-\lambda) / n}$, then ${H_1^q(M) \subset C_B^\lambda(M)}$.
\label{thm46}
\end{Theorem}

As a remark, note that with the same arguments as the ones used in the second part of the proof of \autoref{thm44}, one gets from \autoref{thm46} that for $(M, g)$ as in the statement of \autoref{thm46}, for $q \geq 1$ real and for $m<k$ two integers, if $1 / q<(k-m) / n$, then $H_k^q(M) \subset C_B^m(M)$.

\newpage
\section{Compact Manifolds with Boundary}

We now turn to the study of Sobolev embeddings on compact manifolds with boundary. 
In Section 5.1, We first recall some basic definitions and results on manifolds with boundary. 
Then we introduce the general results of embeddings and the proof of validity in Section 5.2. Finally, we discuss the compactness of the embeddings in Section 5.3.
\subsection{Background Materials}

\subsubsection{Manifolds with Boundary}

\begin{Definition}
Let $E$ be the half-space of $\mathbb{R}^n\left(x^1<0\right), x^1$ the first coordinate of $\mathbb{R}^n$. Consider $\bar{E} \subset \mathbb{R}^n$ with the induced topology. We identify the hyperplane of $\mathbb{R}^n, x^1=0$, with $\mathbb{R}^{n-1}$.
$M_n$ is a manifold of dimension $n$ with boundary if each point of $M_n$ has a neighborhood homeomorphic to an open set of $\bar{E}$.
The points of $M_n$ which have a neighborhood homeomorphic to $\mathbb{R}^n$ are called interior points. They form the inside of $M_n$. The other points are called boundary points. We denote the set of boundary points by $\partial M$.
\end{Definition}

\begin{Theorem}
Let $M_n$ be a $\left(C^k\right.$-differentiable) manifold with boundary. If $\partial M$ is not empty, then $\partial M$ is a ( $C^k$-differentiable) manifold of dimension $(n-1)$, without boundary: $\partial(\partial M)=\varnothing$.
\label{thm51}
\end{Theorem}

\begin{Proof}
If $Q \in \partial M$, there exists a neighborhood $\Omega$ of $Q$ homeomorphic by $\varphi$, to an open set $\Theta \subset \bar{E}$. The restriction $\tilde{\varphi}$ of $\varphi$ to $\tilde{\Omega}=\Omega \cap \partial M$ is a homeomorphism of a neighborhood $\widetilde{\Omega}$ of $Q \in \partial M$ onto an open set $\widetilde{\Theta} \subset \mathbb{R}^{n-1}$. Thus $\partial M$ is a manifold (without boundary) of dimension $(n-1)$. If $M_n$ is $C^k$-differentiable, let $\left(\Omega_i, \varphi_i\right)_{i \in I}$ be a $C^k$-atlas. Clearly, $\left(\widetilde{\Omega}_i, \tilde{\varphi}_i\right)_{i \in I}$ form a $C^k$-atlas for $\partial M$.
\end{Proof}

\begin{Definition}
    By $\bar{W}_n$ a compact Riemannian manifold with boundary of class $C^k$, we understand the following: $\bar{W}_n$ is a $C^k$-differentiable manifold with boundary and $\bar{W}_n$ is a compact subset of $M_n$, a $C^{\infty}$ Riemannian manifold. We set $W=\bar{W}$. We always suppose that the boundary is $C^1$, or at least Lipschitzian.
\label{def52}
\end{Definition}

\subsubsection{Oriented Manifolds with Boundary}
\begin{Theorem}
If $M_n$ is a $C^k$-differentiable oriented manifold with boundary, $\partial M$ is orientable. An orientation of $M_n$ induces a natural orientation of $\partial M$.
\label{thm52}
\end{Theorem}

\begin{Proof}
    Let $\left(\Omega_j, \varphi_j\right)_{j \in I}$ be an allowable atlas with the orientation of $M_n$, and $\left(\widetilde{\Omega}_j, \tilde{\varphi}_j\right)_{j \in I}$ the corresponding atlas of $\partial M$, as above. Set $\mathrm{i}: \partial M \rightarrow M$, the canonical embedding of $\partial M$ into $M$. We identify $Q$ with $\mathrm{i}(Q)$, and $X \in T_Q(\partial M)$ with $\mathrm{i}_*(X) \in T_Q(M)$. 
    
    \vskip 3pt
    Given $Q \in \partial M$, pick $e_1 \in T_Q(M), e_1 \notin T_Q(\partial M)$, $e_1$ being oriented to the outside, namely, $e_1(f) \geq 0$ for all functions differentiable on a neighborhood of $Q$, which satisfy $f \leq 0$ in $M_n, f(Q)=0$. We choose a basis of $T_Q(\partial M)=\left\{e_2, e_3, \ldots, e_n\right\}$, such that the basis of $T_Q(M):\left\{e_1, e_2, \ldots, e_n\right\}$, belongs to the positive orientation given on $M_n$.
\end{Proof}

\vskip 3pt
This procedure defines a canonical orientation on $\partial M$, as one can see.

\subsubsection{Properties of Compact Manifolds with Boundary}

\begin{Theorem}
Let $\bar{W}_n$ be a compact Riemannian manifold with boundary of class $C^r$. Then $C^r(\bar{W})$ is dense in $H_k^p(W)$ for $k \leq r$.
\label{thm53}
\end{Theorem}

\begin{Proof}
Let $\left(\Omega_i, \varphi_i\right)$ be a finite $C^r$ atlas of $\bar{W}$, each $\Omega_i$ being homeomorphic either to a ball $B$ of $\mathbb{R}^n$, or to a half ball $D \subset \bar{E}(D=B \cap \bar{E})$. $C^r(\bar{W})$ is the set of functions belonging to $C^r(W) \cap C^0(\bar{W})$, whose derivatives of order $\leq r$, in each $\Omega_i$, can be extended to continuous functions on $\bar{W} \cap \Omega_i$.

\vskip 3pt
Consider a $C^{\infty}$ partition of unity $\left\{\alpha_i\right\}$ subordinate to the covering $\left\{\Omega_i\right\}$ of $\bar{W}$. Let $f \in H_k^p(W) \cap C^{\infty}(W)$. We have to prove that each function $\alpha_i f$ can be approximated in $H_k^p(W)$ by functions of $C^r(\bar{W})$. There is only a problem for the $\Omega_i$ homeomorphic to $D$. Let $\Omega_i$ be one of them.
The sequence of functions $h_m$ defined, for $m$ sufficiently large, as the restriction to $D$ of $\left[\left(\alpha_i f\right) \circ \varphi_i^{-1}\right]\left(x_1-1 / m, x_2, \ldots, x_n\right)$ converges to $\left(\alpha_i f\right) \circ \varphi_i^{-1}$ in $H_k^p(D)$, where $D$ has the Euclidean metric. Since the metric tensor, and all its derivatives are bounded on $\Omega_i$ (by a proper choice of the $\Omega_i$, without loss of generality), $h_m \circ \varphi_i \in C^r(\bar{W})$ and converges to $\alpha_i f$ in $H_k^p(W)$ for $k \leq r$, when $m \rightarrow \infty$.
\end{Proof}


\subsection{Embeddings for Compact Manifolds with Boundary}

\begin{Theorem}
    \label{thm54}
    For the compact manifolds ${\bar{W}_n}$ with ${C^r}$-boundary, ${(r \geq 1)}$, the Sobolev embedding theorem holds. More precisely: \textbf{First part. The embedding $\bm{H_k^q(W) \subset H_l^p(W)}$ is continuous with $\bm{1 / p=1 / q-$ $(k-\ell) / n>0}$. }
    Moreover, for any $\varepsilon>0$, there exists a constant $A_q(\varepsilon)$ such
    that every $\varphi \in H_1^q\left(W_n\right)$ satisfies 
    \begin{equation}
    \|\varphi\|_p \leq[\mathrm{K}(n, q)+\varepsilon]\|\nabla \varphi\|_q+A_q(\varepsilon)\|\varphi\|_q \text {,   with } 1 / p=1 / q-1 / n>0 \text {, }
    \label{eq11}
    \end{equation}
    and such that every $\varphi \in H_1^q\left(W_n\right)$ satisfies:
    \begin{equation}
    \|\varphi\|_p \leq\left[2^{1 / n} \mathrm{~K}(n, q)+\varepsilon\right]\|\nabla \varphi\|_q+A_q(\varepsilon)\|\varphi\|_q .
    \label{eq13}
    \end{equation}
    \textbf{Second part. The following embeddings are continuous:}
    
    \textbf{(a) $\bm{H_k^q(W) \subset C_B^s(W)}$, if $\bm{k-n / q>s \geq 0}$, s being an integer,}
    
    \textbf{(b) $\bm{H_k^q(W) \subset C^s(\bar{W})}$, if in addition $\bm{s<r}$;}
    
    \textbf{(c) $\bm{H_k^q(W) \subset C^\alpha(\bar{W})}$, if $\bm{\alpha}$ satisfies $\bm{0<\alpha<1}$ and $\bm{\alpha \leq k-n / q}$ instead.}
\end{Theorem}

\subsubsection{Proof of the Fisrt Part}
From Section 3 we know that for a compact manifold $M$, the Sobolev embedding theorem holds. Moreover $H_k^q$ does not depend on the Riemannian metric.
Let $\left\{\Omega_i\right\}$ be a finite covering of $M$, $(i=1,2, \ldots, N)$, and $\left(\Omega_i, \varphi_i\right)$ the corresponding charts. Consider $\left\{\alpha_i\right\}$ a $C^{\infty}$ partition of unity subordinate to the covering $\left\{\Omega_i\right\}$. 
Particularly there exist constants $C_i$ such that every $C^{\infty}$ function $f$ on $M$ satisfies:
\begin{equation}
\left\|\alpha_i f\right\|_p \leq C_i\left\|\alpha_i f\right\|_{H_i^q} .
\label{eq12}
\end{equation}

\noindent
Now let $\left(\Omega_i, \varphi_i\right)$ be a finite $C^r$-atlas of $\bar{W}_n$, each $\Omega_i$ being homeomorphic either to a ball of $\mathbb{R}^n$ or to a half ball $D \subset \bar{E}$. We have only to prove \autoref{eq12} for all $f \in H_1^q(W) \cap$ $C^{\infty}(W), \alpha_i$ being a $C^r$ partition of unity subordinate to the covering $\Omega_i$. When $\Omega_i$ is homeomorphic to a ball, the proof is that of \autoref{thm35}. When $\Omega_i$ is homeomorphic to a half ball, the proof is similar. But one applies the following lemma:

\begin{Lemma}
    Let $\psi$ be $C^1$-function on $\bar{E}$, whose support belongs to $D$, then $\psi$ satisfies:
$$
\|\psi\|_p \leq 2^{1 / n} \mathrm{~K}(n, q)\|\nabla \psi\|_q, \quad with  \;1 / p=1 / q-1 / n>0
$$
\label{lem51}
\end{Lemma}

\vskip -23pt
\begin{Proof}
    Recall that $E$ is the half-space of $\mathbb{R}^n$ and $D=B \cap \bar{E}$, where $B$ is the open ball with center 0 and radius 1. 
    Consider $\tilde{\psi}$ defined, for $x \in \bar{E}$, by $\tilde{\psi}(x)=\psi(x)$ and $\tilde{\psi}(\tilde{x})=\psi(x)$, when $\tilde{x}=\left(-x_1, x_2, \ldots, x_n\right),\left(x_1, x_2, \ldots, x_n\right)$ being the coordinates of $x . \tilde{\psi}$ is a Lipschitzian function with compact support, thus $\tilde{\psi} \in H_1^q\left(\mathbb{R}^n\right)$ and according to the Sobolev embeddings for $\mathbb{R}^n$:
    $$
    \|\tilde{\psi}\|_p \leq \mathrm{K}(n, q)\|\nabla \tilde{\psi}\|_q .
    $$
    The lemma follows, since
    $$
    2 \int_{\bar{E}}|\psi|^p d E=\int_{\mathbb{R}^n}|\tilde{\psi}|^p d E \quad \text { and } \quad 2 \int_{\bar{E}}|\nabla \psi|^q d E=\int_{\mathbb{R}^n}|\nabla \tilde{\psi}|^q d E \text {. }
    $$\end{Proof}

The proof that every $\varphi \in \hat{H}_1^q\left(W_n\right)$ satisfies \autoref{eq11} can be found in Aubin \cite{Aubin}, using the condition that the covering is finite.
And by \autoref{lem51}, we can prove that all $\varphi \in H_1^q\left(W_n\right)$ satisfy \autoref{eq13}; for a complete proof see Cherrier \cite{Cherrier}.

\subsubsection{Proof of the Second Part}

\begin{Proofa}
There exist constants $C_i(\tilde{q})$ such that for all $f \in H_1^{\tilde{q}}(W) \cap C^{\infty}(W)$
    \begin{equation}
    \sup \left|\alpha_i f\right| \leq C_i(\tilde{q})\|f\|_{H_i^{\tilde{q}}} \;\text {, if } \tilde{q}>n \text {. }
    \label{eq14}
\end{equation}
Set $h(x)=0$ for $x \notin D$, and $h(x)=\left(\alpha_i f\right) \circ \varphi_i^{-1}(x)$ for $x \in D$.
Consider a half straight line through $x$, defined by $\theta \in \mathbb{S}_{n-1}(1)$, entirely included in $E$. We have
$$
|h(x)| \leq \int_0^1|\nabla h(r, \theta)| d r .
$$
From Aubin \cite{Aubin}, Lemma 2.22, we can get: 
$$
|h(x)| \leq\left(\omega_{n-1} / 2\right)^{-1}\left(\int_E|\nabla h|^{\tilde{q}} d x\right)^{1 / \tilde{q}}\left(\frac{\omega_{n-1}}{2} \int_0^1 r^{(n-1)\left(1-q^{\prime}\right)} d r\right)^{1 / q^{\prime}}
$$
where $1 / q^{\prime}=1-1 / \tilde{q}$.
Since the metric tensor is bounded on $\Omega_i$ (by proper choice of the $\left(\Omega_i, \varphi_i\right)$, without loss of generality), for some constant $\widetilde{C}_i(\tilde{q})$ we obtain:
$$
\sup \left|\alpha_i f\right| \leq \tilde{C}_i(\tilde{q})\left\|\nabla\left(\alpha_i f\right)\right\|_{\tilde{q}} .
$$
\autoref{eq14} follows; thus, recalling that $I$ is finite, we have
$$
\sup |f| \leq C(\tilde{q})\|f\|_{H_1^{\tilde{q}}}, \quad \text { with } C(\tilde{q})=\sum_{i \in I} C_i(\tilde{q}) .
$$
Since $k>n / q$, there exists $\tilde{q}>n$, such that the embedding $H_k^q(W) \subset$ $H_1^{\tilde{q}}(W)$ is continuous; we have only to choose $1 / \tilde{q} \geq 1 / q-(k-1) / n$. So there exists a constant $C$, such that every $f \in H_k^q(W) \cap C^{\infty}(W)$ satisfies:
$$
\sup |f| \leq C\|f\|_{H_k^q} .
$$
Thus a Cauchy sequence of $C^{\infty}$ functions in $H_k^q(W)$ is a Cauchy sequence in $C^0(W)$ and the preceding inequality holds for all $f \in H_k^q(W)$.
For $s>0$, apply the preceding result to $\left|\nabla^{\ell} f\right|, 0 \leq \ell \leq s$, and the continuous embedding $H_k^q(W) \subset C_B^s(W)$ is established.
\end{Proofa}

\vskip 10pt
\begin{Proofb}
    Instead of taking $f \in C^{\infty}(W)$, we establish an inequality of the type:
    \begin{equation}
    \|f\|_{{C}^s} \leq A\|f\|_{{H}_{s+1}^{\tilde{q}}}, \quad \text { when } 0 \leq s<r,
\label{eq15}    
\end{equation}
    for the functions $f \in C^r(\bar{W})$, with $A$ a constant and $\tilde{q}>n$. According to \autoref{thm53}, $C^r(\bar{W})$ is dense in $H_{s+1}^{\tilde{q}}(W)$. Thus $H_{s+1}^{\tilde{q}}(W) \subset$ $C^s(\bar{W})$ and \autoref{eq15} holds for all $f \in H_{s+1}^{\tilde{q}}(W)$. When $k-n / q>s$, we may choose $\tilde{q}>n$, such that $1 / \tilde{q} \geq 1 / q-(k-s-1) / n$. In this case the embedding $H_k^q(W) \subset H_{s+1}^{\tilde{q}}(W)$ is continuous and so $H_k^q(W) \subset C^s(\bar{W})$.
\end{Proofb}

\vskip 10pt
\begin{Proofc}
    Let $f \in H_k^q(W)$; according to the preceding result $f \in C^0(\bar{W})$ because $k-n / q>0$. Consider the function on $D$, defined by $h(x)=\left(\alpha_i f\right) \circ \varphi_i^{-1}(x)$, for a given $i \in I$. According to Aubin \cite{Aubin} Theorem 2.23 , we can establish the existence of a constant $B$ such that every $f \in H_1^{\tilde{q}}(W)$ satisfies:
    $$
    |h(x)-h(y)|\|x-y\|^{-\alpha} \leq B\left(\int_D|\nabla h|^{\tilde{q}} d x\right)^{1 / \tilde{q}},
    $$
    where $\tilde{q}=n /(1-\alpha)$. Instead of considering a ball of radius $\|x-y\| / 2$, we must integrate over a cube $K$ with edge $\|x-y\|$, included in $\bar{E}$, with $x$ and $y$ belonging to $K$ (see Adams \cite{A} p109).
    
    \vskip 3pt
    \noindent
    Then, since the metric tensor is bounded on $\Omega_i$, there exists a constant $B_i$ such that, for every pair $(P, Q)$ of points of $\bar{W}$, any $f \in H_1^{\bar{q}}$ satisfies:
    $$
    \left|\alpha_i(P) f(P)-\alpha_i(Q) f(Q)\right|[d(P, Q)]^{-\alpha} \leq B_i\|f\|_{H_1^{\tilde{q}}} .
    $$
    Thus we establish the desired inequality:
    $$
    |f(P)-f(Q)||d(P, Q)|^{-\alpha} \leq\left(\sum_{i \in I} B_i\right)\|f\|_{H_1^{\tilde{q}}} \leq \text { Const } \times\|f\|_{H_k^{q}},
    $$
    where the last inequality follows from the first part of the Sobolev embedding theorem, since $\tilde{q}=n /(1-\alpha)$ satisfies $1 / \tilde{q} \geq 1 / q-(k-1) / n$.
\end{Proofc}


\subsection{Compactness of Embeddings}

\subsubsection{The Kondrakov Theorem for $\mathbb{R}^n$}

\begin{Theorem}
Let $k \geq 0$ be an integer, $p$ and $q$ two real numbers satisfying $1 \geq 1 / p>$ $1 / q-k / n>0$. The Kondrakov Theorem asserts that, if $\Omega$, a bounded open set of $\mathbb{R}^n$, has a sufficiently regular boundary $\partial \Omega$ ( $\partial \Omega$ of class $C^1$, or only Lipschitzian):

(a) the embedding $H_k^q(\Omega) \subset L_p(\Omega)$ is compact.

(b) With the same assumptions for $\Omega$, the embedding $H_k^q(\Omega) \subset C^\alpha(\bar{\Omega})$ is compact, if 

$\quad\;\;\,$$k-\alpha>n / q$, with $0 \leq \alpha<1$.

(c) For $\Omega$ a bounded open set of $\mathbb{R}^n$, the following embeddings are compact:
$$
\stackrel{\circ}{H_k^q}(\Omega) \subset L_p(\Omega),\; \stackrel{\circ}{H^q_k}(\Omega) \subset C^\alpha(\bar{\Omega}) .
$$
\label{thm55}
\end{Theorem}

\vskip -23pt
\begin{Proof}
Roughly, the proof consists in proving that if the Sobolev embedding theorem holds for a bounded domain $\Omega$, then the Kondrakov theorem is true for $\Omega$.

\vskip 10pt
\textbf{(a)} According to the Sobolev embedding \autoref{thm54}, the embedding $H_k^q \subset H_1^{\tilde{q}}$ is continuous with $1 / \tilde{q}=1 / q-(k-1) / n$. Thus we have only to
prove that the embedding of $H_1^{\tilde{q}} \subset L_p$ is compact when $1 \geq 1 / p>1 / \tilde{q}-1 / n>0$, since the composition of two continuous embeddings is compact if one of them is compact.
Let $\mathcal{A}$ be a bounded subset of $H_1^{\tilde{q}}(\Omega)$, so if $f \in \mathcal{A}$,
$$
\|f\|_{H^{\tilde{q}}_1} \leq C, \text { a constant. }
$$
By hypothesis $H_1^{\tilde{q}}(\Omega) \subset L_r$, with $1 / r=1 / \tilde{q}-1 / n$, and there exists a constant $A$ such that for $f \in H_1^{\tilde{q}}(\Omega)$,
$$
\|f\|_r \leq A\|f\|_{H_1^{\bar{q}}} .
$$
Set $K_j=\{x \in \Omega \,,\, \operatorname{dist}(x, \partial \Omega) \geq 2 / j\}, j \in \mathbb{N}$. For $f \in \mathcal{A}$, by Hölder's inequality:
$$
\int_{\Omega-K_j}|f| d x \leq\left(\int_{\Omega-K_j}|f|^r d x\right)^{1 / r}\left(\int_{\Omega-K_j} d x\right)^{1-1 / r} \leq A C\left(\int_{\Omega-K_j} d x\right)^{1-1 / r}
$$
which goes to zero, when $j \rightarrow \infty$. Thus, given $\varepsilon>0$, there exists $j_0 \in \mathbb{N}$, such that $\left[\operatorname{vol}\left(\Omega-K_{j_0}\right)\right]^{1-1 / r} \leq \varepsilon / A C$. Now, by Fubini's theorem:
$$
\int_{K_{j_0}}|f(x+y)-f(x)| d x  \leq \int_{K_{j_0}} d x \int_0^1\left|\frac{d}{d t} f(x+t y)\right| d t \leq\|y\| \int_{K_{2 j_0}}|\nabla f| d x \leq\|y\|\|\nabla f\|_1
$$
for $\|y\|<1 / j_0$ since $x+y \in K_{2j_0}$, if $x \in K_{j_0}$. Since $C^{\infty}(\Omega)$ is dense in $H_1^{\tilde{q}}(\Omega)$, the preceding inequality holds for any $f \in H_1^{\tilde{q}}(\Omega)$. Moreover, by Hölder's inequality, $\|\nabla f\|_1 \leq\|\nabla f\|_r(\operatorname{vol} \Omega)^{1-1 / r} \leq B$, a constant.

\vskip  3pt
\noindent
Take $\delta=\varepsilon / B$ and we can imply that $\mathcal{A}$ is precompact in $L_1(\Omega)$. Hence $\mathcal{A}$ is precompact in $L_p(\Omega)$, because if $f_m \in \mathcal{A}$ is a Cauchy sequence in $L_1$, it is a Cauchy sequence in $L_p$ :
$$
\left\|f_m-f_{\ell}\right\|_p \leq\left\|f_m-f_{\ell}\right\|_1^\mu\left\|f_m-f_{\ell}\right\|_r^{1-\mu} \leq(2 A C)^{1-\mu}\left\|f_m-f_{\ell}\right\|_1^\mu,
$$
by Hölder's inequality, with $\mu=[(r / p)-1] /(r-1)$.

\vskip 10pt
\textbf{(b)} Let $\lambda$ satisfy $\alpha<\lambda<$ $\inf (1, k-n / q)$. Then by the Sobolev embedding theorem \autoref{thm54}, $H_k^q(\Omega)$ is included in $C^\lambda(\bar{\Omega})$, and there exists a constant $A$ such that $\|f\|_{C^\lambda} \leq A\|f\|_{H_k^q}$.
Let $\mathcal{A}$ be a bounded subset of $H_k^q(\Omega)$; if $f \in \mathcal{A},\|f\|_{H_k^q} \leq C$, a constant, and $\|f\|_{C^\lambda} \leq A C$.
Thus we can apply Ascoli's Theorem, $\mathcal{A}$ is a bounded subset of equicontinuous functions of $C^0(\bar{\Omega})$, and $\bar{\Omega}$ is compact. So $\mathcal{A}$ is precompact in $C^0(\bar{\Omega})$.

\noindent
Then, since
$$
|f(x)-f(y)|\|x-y\|^{-\alpha}=\left(|f(x)-f(y)|\|x-y\|^{-\lambda}\right)^{\alpha / \lambda}|f(x)-f(y)|^{1-\alpha / \lambda}
$$
if a sequence $f_m \in \mathcal{A}$ converges to $f$ in $C^0(\bar{\Omega}),\|f\|_{C^\lambda} \leq A C$ and
$$
\left\|f-f_m\right\|_{C^\alpha} \leq(2 A C)^{\alpha / \lambda}\left(\left\|f-f_m\right\|_{C^0}\right)^{1-\alpha / \lambda}+\left\|f-f_m\right\|_{C^0} .
$$
Thus $\mathcal{A}$ is precompact in $C^\alpha(\bar{\Omega})$.

\vskip 10pt
\textbf{(c)} $\mathcal{D}(\Omega)$ is included in $\mathcal{D}\left(\mathbb{R}^n\right)$, so we can apply the Sobolev theorem for $\mathbb{R}^n$ to the space $\stackrel{\circ}{H_k^q}(\Omega)$. A proof similar to those of (a) and (b) gives the desired result.
\end{Proof}

\subsubsection{The Kondrakov Theorem for Compact Manifolds}

\begin{Theorem}
The Kondrakov theorem, \autoref{thm55}, holds for the compact Riemannian manifolds ${M_n}$, and the compact Riemannian manifolds ${\bar{W}_n}$ with ${C^1}$-boundary. Namely, the following embeddings are compact:

\vskip 5pt
\noindent
(a) ${H_k^q\left(M_n\right) \subset L_p\left(M_n\right)}$ and \textbf{$\bm{H_k^q\left(W_n\right) \subset L_p\left(W_n\right)}$, with $\bm{1 \geq 1 / p>1 / q-$ $k / n>0}$.}

\vskip 5pt
\noindent
(b) ${H_k^q\left(M_n\right) \subset C^\alpha\left(M_n\right)}$ and \textbf{$\bm{H_k^q\left(W_n\right) \subset C^\alpha\left(\bar{W}_n\right)}$, if $\bm{k-\alpha>n / q}$, with $\bm{0 \leq \alpha<1}$}
\end{Theorem}

\begin{Proof}
    Let $\left(\Omega_i, \varphi_i\right),(i=1,2, \ldots, N)$ be a finite atlas of $M_n$ (respectively, $C^1$- atlas of $\bar{W}_n$ ), each $\Omega_i$ being homeomorphic either to a ball of $\mathbb{R}^n$ or to a half ball $D \subset \bar{E}$. We choose the atlas so that in each chart the metric tensor is bounded. Consider a $C^{\infty}$ partition of unity $\left\{\alpha_i\right\}$ subordinate to the covering $\left\{\Omega_i\right\}$. It is sufficient to prove the theorem in the special case $k=1$ for the same reason as in the preceding proof of \autoref{thm55}.

    \vskip 10pt
    \textbf{(a)} Let $\left\{f_m\right\}$ be a bounded sequence in $H_1^q$. Consider the functions defined on $B$ (or on $D$), $i$ being given:
$$
h_m(x)=\left(\alpha_i f_m\right) \circ \varphi_i^{-1}(x) \text {. }
$$
Since the metric tensor is bounded on $\Omega_i$, the set $\mathcal{A}_i$ of these functions is bounded in $H_1^q(\Omega)$ with $\Omega=B$ or $D$. (The boundary of $D$ is only Lipschitzian, but, since $\operatorname{supp}\left(\alpha_i \circ \varphi_i^{-1}\right)$ is included in $B$, we may consider a bounded open set $\Omega$ with smooth boundary which satisfies $D \subset \Omega \subset E$).

\noindent
According to \autoref{thm55}, $\mathcal{A}_i$ is precompact. Thus there exists a subsequence which is a Cauchy sequence in $L_p$. Repeating this operation successively for $i=1,2, \ldots, N$, we may select a subsequence $\left\{\tilde{f}_m\right\}$ of the sequence $\left\{f_m\right\}$, such that $\alpha_i \tilde{f}_m$ is a Cauchy sequence in $L_p$ for each $i$. Thus $\left\{\tilde{f}_m\right\}$ is a Cauchy sequence in $L_p$, since
$$
\left|\tilde{f}_m-\tilde{f}_{\ell}\right| \leq \sum_{i=1}^N\left|\alpha_i \tilde{f}_m-\alpha_i \tilde{f}_{\ell}\right| .
$$

\vskip 10pt
\textbf{(b)} Let $\lambda$ satisfy $\alpha<\lambda<\inf (1, k-n / q)$. According to \autoref{thm54} and \autoref{thm55}, the embeddings $H_k^q\left(M_n\right) \subset C^\lambda\left(M_n\right)$ and $H_k^q\left(W_n\right) \subset C^\lambda\left(\bar{W}_n\right)$ are continuous. Thus the same proof used to show \autoref{thm55} $\mathrm{(b)}$  establishes the result.
\end{Proof}

\vskip 10pt
\textbf{Remark 1.} We have given only the main results concerning the theorems of Sobolev and Kondrakov. These theorems are proved for the compact manifolds with Lipschitzian boundary in Aubin \cite{Aubin-}. To obtain complete results for domains of $\mathbb{R}^n$ see Adams \cite{A}.

\vskip 3pt
\textbf{Remark 2.} Instead of the spaces $H_k^p\left(M_n\right)$, it is possible to introduce the spaces $H_k^{\prime p}\left(M_n\right)$, which are the completion of $\mathcal{E}_k^{\prime p}$ with respect to the norm
$$
\|\varphi\|_{H_k^{\prime p} }=\sum_{0=\ell \leq k / 2}\left\|\Delta^{\ell} \varphi\right\|_p+\sum_{0=\ell \leq(k-1) / 2}\left\|\nabla \Delta^{\ell} \varphi\right\|_p,
$$
with $\mathcal{E}_k^{\prime p}$ the vector space of the functions $\varphi \in C^{\infty}\left(M_n\right)$, such that $\Delta^{\ell} \varphi \in L_p\left(M_n\right)$ for $0 \leq \ell \leq k / 2$ and such that $\left|\nabla \Delta^{\ell} \varphi\right| \in L_p\left(M_n\right)$ for $0 \leq \ell \leq(k-1) / 2$. For these spaces, the Kondrakov theorem holds, as well as the Sobolev embedding theorem when $p>1$ (see Aubin \cite{Aubin-}).








































\newpage
\section{Summary and Perspectives}
\label{Sec6}
\subsection{Main Results}

\subsubsection{Compact Manifolds without Boundary}
\vskip -3pt
Let $(M, g)$ be a smooth, compact Riemannian n-manifold.
Let $q\geq 1$ be real and $0 \leq m<k$ be two integers.
    
\vskip 2pt
(A1) If ${1/p=1 / q-(k-m) / n > 0}$, then ${H_k^q(M) \subset H_m^p(M)}$.
    
\vskip 2pt
(A2) If ${1/p=1 / q-(k-m) / n < 0}$, then ${H_k^q(M) \subset C^m(M)}$.
    
\vskip 2pt
(B1)  For any real ${p}$ such that ${1 \leq p<\frac{n q}{(n-(k-m) q)}}$, ${H_{k}^q(M)\subset \subset H_m^p(M)}$.
    
\vskip 2pt
(B2) For ${q>n}$ and for any ${\lambda \in (0,1)}$, such that ${(1-\lambda) q>n}$, ${H_1^q(M)\subset \subset C^\lambda(M)}$. 


\subsubsection{Complete Manifolds without Boundary}
\vskip -3pt
Let $(M, g)$ be a smooth, complete Riemannian $n$-manifold with Ricci curvature bounded from below. Assume that
$\inf _{x \in M} \operatorname{Vol}_g\left(B_x(1)\right)>0$
where $\operatorname{Vol}_g\left(B_x(1)\right)$ stands for the volume of $B_x(1)$ with respect to $g$.

\vskip 2pt
(A1) If ${1/p=1 / q-(k-m) / n > 0}$, then ${H_k^q(M) \subset H_m^p(M)}$.

\vskip 7pt
Let $(M, g)$ be a smooth, complete Riemannian $n$-manifold with Ricci curvature bounded from below and positive injectivity radius. 

\vskip 2pt
(A2) If ${1/p=1 / q-(k-m) / n < 0}$, then ${H_k^q(M) \subset C_B^m(M)}$.
  

\subsubsection{Compact Manifolds with Boundary}
\vskip -3pt
For the compact manifolds ${\bar{W}_n}$ with ${C^r}$-boundary, ${(r \geq 1)}$.

\vskip 2pt
(A1) If ${1/p=1 / q-(k-m) / n > 0}$, then ${H_k^q(M) \subset H_m^p(M)}$.
   
\vskip 2pt
(A2-a) If ${1/p=1 / q-(k-m) / n < 0}$, ${k-n / q>s \geq 0}$, then ${H_k^q(W) \subset C_B^s(W)}$.
    
\vskip 2pt
(A2-b) If in addition ${s<r}$, then ${H_k^q(W) \subset C^s(\bar{W})}$.

\vskip 2pt
(A2-c) If ${\alpha}$ satisfies ${0<\alpha<1}$ and ${\alpha \leq k-n / q}$ instead, then ${H_k^q(W) \subset C^\alpha(\bar{W})}$

\vskip 7pt
For the compact Riemannian manifolds ${\bar{W}_n}$ with ${C^1}$-boundary. 

\vskip 2pt
(B1) If ${1 \geq 1 / p>1 / q- k / n>0}$, then ${H_k^q\left(W_n\right) \subset\subset L_p\left(W_n\right)}$.

\vskip 2pt
(B2) If $1/q-k/n<0, {k-\alpha>n / q}$, with ${0 \leq \alpha<1}$, then ${H_k^q\left(W_n\right) \subset\subset C^\alpha\left(\bar{W}_n\right)}$. 


\subsection{Perspectives}
As one can see above, the compactness of embeddings for complete manifolds without boundary is still to be studied. 
How to find the best constant in the embeddings is also an interesting problem.


%%------------------------正文页从这里开始-------------------%






%%----------- 参考文献 -------------------%%
%在reference.bib文件中填写参考文献,此处自动生成
\newpage
\reference






































\end{document}